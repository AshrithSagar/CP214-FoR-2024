\section*{Problem 5}

\textbf{Drone Orientation Error}:
Recall Question 6 from Homework 3.

For any ``planner/controller/AI'' aiming to follow a trajectory for the drone, it needs to assess how poorly it is currently performing or how poorly it will perform.
One of the quantities it can examine is the error between the desired orientation \( R_{d} \in SO(3) \) and the current orientation \( R_{c} \in SO(3) \), or the \( k^{th} \) predicted orientation \( R_{k} \in SO(3) \).
Now, answer the following questions for calculating this error.
\begin{enumerate}[label= (\alph*)]
      \item Simply computing \( \left \| R_{d}-R_{c}\right \| \) or \( \left \| R_{d}-R_{k}\right \| \) would yield incorrect results.
            Why is this the case?
      \item  Show that computing \( \left \| \log {\left(R_{d}^{T} R_{c}\right)}^{\vee}\right \| \) or \( \left \| \log {\left(R_{d}^{T} R_{k}\right)}^{\vee}\right \| \) is more appropriate, and provide a proper explanation.
            Here, \( \log (\cdot) \) is the same as what was defined in Question 4 (Hint: you don't need to use the exact formula to answer this question), and the vee operator \( { }^{\vee} \) maps from so \( (3) \) to \( \mathbb{R}^{3} \) as the inverse of the hat operator.
            For example, \( \hat{a}^{\vee}=a \), which extracts the vector from the corresponding skew-symmetric matrix.
\end{enumerate}

\subsection*{Solution}

\subsubsection*{(a) Computing \( \left \| R_{d}-R_{c}\right \| \) or \( \left \| R_{d}-R_{k}\right \| \) is incorrect}

We will assume the norm to be the Frobenius norm.
We can note that \( R_{d} - R_{c} \) is not a rotation matrix.
We can argue that even though we can compute the norm of \( R_{d} - R_{c} \), it does not give us the correct orientation error.
We can illustrate by considering 2D rotations.
The 2D rotation group \( SO(2) \) is bijective to rotations in the complex plane, and we can observe the following.
Consider \( R_d = e^{i \theta} \) and \( R_c = e^{i \phi} \).
We have by Euler's formula that \( e^{i \theta} = \cos \theta + i \sin \theta \).
We can compute the error as \( \left \| R_d - R_c \right \| = \left \| e^{i \theta} - e^{i \phi} \right \| = \left \| \cos \theta + i \sin \theta - \cos \phi - i \sin \phi \right \| = \left \| \cos \theta - \cos \phi + i (\sin \theta - \sin \phi) \right \| = \sqrt{(\cos \theta - \cos \phi)^2 + (\sin \theta - \sin \phi)^2} = \sqrt{2 - 2 \cos (\theta - \phi)} \neq \left \| \theta - \phi \right \| \).

In a rough sense, it can be used as a metric, but it is \underline{not appropriate}.

\subsubsection*{(b) Computing \( \left \| \log {\left(R_{d}^{T} R_{c}\right)}^{\vee}\right \| \) or \( \left \| \log {\left(R_{d}^{T} R_{k}\right)}^{\vee}\right \| \) is more appropriate}

The expression \( \left \| \log {\left(R_{d}^{T} R_{c}\right)}^{\vee}\right \| \) is more appropriate.
The \( \left(R_{d}^{T} R_{c}\right) \) term is a rotation matrix, and denotes the rotation by \( R_c \) followed by undoing the rotation by \( R_d \).
We can see that \( R_{err} \triangleq R_d^{T} R_c \in SO(3) \) is a rotation matrix, and thereby we can compute the logarithm of this matrix.
This yields the following
\begin{align*}
      \log R_{err}
       & =
      \hat{\omega} \theta_{err}
      \\
      \implies
      ({\log R_{err}}^{\vee})
       & =
      \theta_{err} \omega
\end{align*}
Since \( \omega \) is a unit vector, we have
\begin{align*}
      \left \| ({\log R_{err}}^{\vee}) \right \|
       & =
      \theta_{err}
\end{align*}
This function gives us the correct orientation error, and is thereby \underline{more appropriate}.

We can also find the orientation error by considering the trace of the rotation matrix \( R_{err} \), as seen in the logarithm map.
\begin{equation*}
      \theta_{err} = \arccos{\frac{\text{trace}(R_{err}) - 1 }{2}}
\end{equation*}

To illustrate the correctness of this approach, we can consider the 2D rotation group \( SO(2) \) as before.
We consider \( R_d = e^{i \theta} \) and \( R_c = e^{i \phi} \), thereby we have \( R_d^T = e^{-i \theta} \), and
\[
      R_d^T R_c = e^{-i \theta} e^{i \phi} = e^{i (\phi - \theta)}
      \implies
      \log (R_d^T R_c) = i (\phi - \theta)
      \implies
      \left \| \log (R_d^T R_c) \right \| = \left \| \phi - \theta \right \|
\]

Thereby, we can see that this expression appropriately computes the orientation error.

In general, this discussion comes under finding metrics on Lie groups, and the logarithm map here allows us to compute the orientation error in a more appropriate manner.
