\section*{Problem 5}

\textbf{Drone Orientation Error}:
Recall Question 6 from Homework 3.

For any ``planner/controller/AI'' aiming to follow a trajectory for the drone, it needs to assess how poorly it is currently performing or how poorly it will perform.
One of the quantities it can examine is the error between the desired orientation \( R_{d} \in SO(3) \) and the current orientation \( R_{c} \in SO(3) \), or the \( k^{th} \) predicted orientation \( R_{k} \in SO(3) \).
Now, answer the following questions for calculating this error.
\begin{enumerate}[label= (\alph*)]
    \item Simply computing \( \left \| R_{d}-R_{c}\right \| \) or \( \left \| R_{d}-R_{k}\right \| \) would yield incorrect results.
          Why is this the case?
    \item  Show that computing \( \left \| \log {\left(R_{d}^{T} R_{c}\right)}^{\vee}\right \| \) or \( \left \| \log {\left(R_{d}^{T} R_{k}\right)}^{\vee}\right \| \) is more appropriate, and provide a proper explanation.
          Here, \( \log (\cdot) \) is the same as what was defined in Question 4 (Hint: you don't need to use the exact formula to answer this question), and the vee operator \( { }^{\vee} \) maps from so \( (3) \) to \( \mathbb{R}^{3} \) as the inverse of the hat operator.
          For example, \( \hat{a}^{\vee}=a \), which extracts the vector from the corresponding skew-symmetric matrix.
\end{enumerate}

\subsection*{Solution}
