\section*{Problem 1}

Show that \( so(3) \) is a vector space.
Determine its dimension and give a basis for \( so(3) \).

\subsection*{Solution}

The vector space skew-symmetric matrix group \( so(3) \) is defined as the set of all skew-symmetric matrices of size \( 3 \times 3 \).
\begin{equation*}
    so(3) = \left \{ A \in \mathbb{R}^{3 \times 3} \mid A^{T}=-A \right \}
\end{equation*}

Let \( \hat u, \hat v, \hat w \in so(3) \).
Since these are skew-symmetric matrices, we have
\[
    \hat u^{T} = -\hat u, \quad \hat v^{T} = -\hat v, \quad \hat w^{T} = -\hat w
\]

For \( so(3) \) to be a vector space, we need to show that it satisfies the following properties.

\subsubsection*{Closure under addition: \( \hat v + \hat w \in so(3) \quad \forall \hat v, \hat w \in so(3) \)}


By the linearity of the transpose operation, we have
\begin{equation*}
    {(\hat v + \hat w)}^{T} = \hat v^{T} + \hat w^{T} = -\hat v - \hat w = -(\hat v + \hat w)
\end{equation*}

Hence, \( \hat v + \hat w \) is also skew-symmetric matrix and \( \hat v + \hat w \in so(3) \).

\subsubsection*{Closure under scalar multiplication: \( c \cdot \hat v \in so(3) \quad \forall c \in \mathbb{R}, \hat v \in so(3) \)}

\begin{equation*}
    \implies
    {(c \cdot \hat v)}^{T} = c \cdot \hat v^{T} = c \cdot (-\hat v) = -(c \cdot \hat v)
\end{equation*}

Hence, \( c \cdot \hat v \) is also skew-symmetric matrix and \( c \cdot \hat v \in so(3) \).

\subsubsection*{Additive identity: \( \hat 0 \in so(3) \)}

\begin{equation*}
    \hat 0 + \hat v = \hat v + \hat 0 = \hat v
\end{equation*}
The zero matrix is skew-symmetric, as \( \hat 0^{T} = -\hat 0 \).
Hence, \( \hat 0 \in so(3) \).

\subsubsection*{Additive inverse: \( -\hat v \in so(3) \quad \forall \hat v \in so(3) \)}

\begin{equation*}
    {(-\hat v)}^{T} = -\hat v^{T} = -(-\hat v) = \hat v
\end{equation*}
Since \( \hat v \in so(3) \), hence, we have \( -\hat v \in so(3) \).

Therefore, \underline{\( so(3) \) is a vector space}.
