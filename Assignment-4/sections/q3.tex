\section*{Problem 3}

Properties of the matrix exponential:
Let \( A \) be a matrix in \( \mathbb{R}^{n \times n} \).
The exponential of \( A \) is defined as
\begin{equation*}
    e^{A}=I+A+A^{2} / 2!+A^{3} / 3!+\cdots
\end{equation*}
\begin{enumerate}[label= (\alph*)]
    \item Let \( g \in \mathbb{R}^{n \times n} \) be an invertible matrix. Show that the following equality is true:
          \begin{equation*}
              g e^{A} g^{-1}=e^{g A g^{-1}}
          \end{equation*}

    \item Verify that for \( \theta \in \mathbb{R} \) below holds (Here \( \dot{\theta} \equiv \frac{d \theta}{d t} \) ).
          \begin{equation*}
              \frac{d}{d t} e^{A \theta}=(\dot{\theta}) e^{A \theta}=e^{A \theta}(A \dot{\theta})
          \end{equation*}
\end{enumerate}

\subsection*{Solution}

The Taylor series of the exponential function for a real \( x \) is given by
\begin{equation*}
    e^{x}
    =
    1+x+\frac{x^{2}}{2 !}+\frac{x^{3}}{3 !}+\cdots
    =
    \sum_{n=0}^{\infty} \frac{x^{n}}{n !}
\end{equation*}
which is an analytic function that converges for all \( x \), thereby we can then define the matrix exponential from it as
\begin{equation*}
    e^{A}
    =
    I+A+\frac{1}{2 !}A^{2}+\frac{1}{3 !}A^{3}+\cdots
    =
    \sum_{n=0}^{\infty} \frac{1}{n !}A^{n}
\end{equation*}

\subsubsection*{(a) \( g e^{A} g^{-1}=e^{g A g^{-1}} \)}

Given that \( g \) is an invertible matrix, we first observe the following property.

For any invertible matrix \( g \), we have
\begin{equation*}
    g A^{k} g^{-1} = {(g A g^{-1})}^{k} \quad \forall k \in \mathbb{N}
\end{equation*}

This follows from the fact that
\begin{align*}
    {(g A g^{-1})}^{k}
     & =
    (g A g^{-1})(g A g^{-1})\cdots(g A g^{-1})
    \\ & =
    g A (g^{-1} g) A (g^{-1} g) \cdots A g^{-1}
    \\  & =
    g A A \cdots A g^{-1}
    \\ & =
    g A^{k} g^{-1}
\end{align*}

Now, we have
\begin{align*}
    g e^{A} g^{-1}
     & =
    g \left( I+A+\frac{1}{2 !}A^{2}+\frac{1}{3 !}A^{3}+\cdots \right) g^{-1}
    \\ & =
    g I g^{-1}+g A g^{-1}+\frac{1}{2 !}g A^{2} g^{-1}+\frac{1}{3 !}g A^{3} g^{-1}+\cdots
    \\ & =
    I+(g A g^{-1})+\frac{1}{2 !}{(g A g^{-1})}^{2}+\frac{1}{3 !}{(g A g^{-1})}^{3}+\cdots
    \\ & =
    e^{g A g^{-1}}
\end{align*}

\subsubsection*{(b) \( \displaystyle \frac{d}{d t} e^{A \theta}=(A\dot{\theta}) e^{A \theta}=e^{A \theta}(A \dot{\theta}) \)}

Given that \( \theta \in \mathbb{R} \), we have
\begin{align*}
    \frac{d}{d t} e^{A \theta}
     & =
    \frac{d}{d t} \left( I+A \theta+\frac{\theta^{2}}{2 !}A^{2}+\frac{\theta^{3}}{3 !}A^{3}+\cdots \right)
    \\ & =
    0+A \frac{d \theta}{d t}+\theta \frac{d \theta}{d t} A^{2}+\frac{\theta^{2}}{2 !} \frac{d \theta}{d t} A^{3}+\cdots
    \\ & =
    A \dot{\theta}+\theta \dot{\theta} A^2+\frac{\theta^{2}}{2 !} \dot{\theta} A^3+\frac{\theta^{3}}{3 !} \dot{\theta} A^4+\cdots
    \\ & =
    (A \dot{\theta})+(A \dot{\theta}) A\theta + (A \dot{\theta}) \frac{\theta^{2}}{2 !} A^{2}+(A \dot{\theta}) \frac{\theta^{3}}{3 !} A^{3}+\cdots
    \\ & =
    (A \dot{\theta}) \left( I+A \theta+\frac{\theta^{2}}{2 !}A^{2}+\frac{\theta^{3}}{3 !}A^{3}+\cdots \right)
    \\ & =
    (A \dot{\theta}) e^{A \theta}
\end{align*}

Similarly, we can also show that this product is commutative,
\begin{align*}
    \implies
    \frac{d}{d t} e^{A \theta}
     & =
    A \dot{\theta}+\theta \dot{\theta} A^2+\frac{\theta^{2}}{2 !} \dot{\theta} A^3+\frac{\theta^{3}}{3 !} \dot{\theta} A^4+\cdots
    \\ & =
    (A \dot{\theta}) + A\theta (A \dot{\theta}) + \frac{\theta^{2}}{2 !} A^{2} (A \dot{\theta}) + \frac{\theta^{3}}{3 !} A^{3} (A \dot{\theta}) + \cdots
    \\ & =
    \left( I+A \theta+\frac{\theta^{2}}{2 !}A^{2}+\frac{\theta^{3}}{3 !}A^{3}+\cdots \right) (A \dot{\theta})
    \\ & =
    e^{A \theta}(A \dot{\theta})
\end{align*}
