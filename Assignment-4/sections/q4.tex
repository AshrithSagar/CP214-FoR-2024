\section*{Problem 4}

Show that the matrix logarithm operator \( \log (\cdot) \) from \( SO(3) \) to \(so(3)\), which is also the inverse of the matrix exponential operator (Rodrigues formula) is given by
\begin{equation*}
    \log (R) = \frac{\theta}{2 \sin \theta}\left(R-R^{T}\right)
\end{equation*}
Here \( R \in S O(3), \theta=\arccos ((\operatorname{tr}(R)-1) / 2) \), and \( \operatorname{tr}(\cdot) \) calculates the trace of a square matrix.

\subsection*{Solution}

The matrix exponential operator, given by the Rodrigues formula, maps a skew-symmetric matrix \( \hat \omega \) to a rotation matrix \( R \in SO(3) \), by
\begin{equation*}
    R
    = e^{\hat \omega \theta}
    = I + \sin \theta \hat \omega + (1 - \cos \theta) \hat \omega^2
\end{equation*}
where \( \lVert \omega \rVert = 1 \) is a unit vector and \( \theta \) is the angle of rotation about \( \omega \).
The matrix logarithm operator, which is the inverse of the matrix exponential operator, maps a rotation matrix \( R \in SO(3) \) to a skew-symmetric matrix \( \hat \omega \in so(3) \), thereby giving
\begin{equation*}
    \log \left( R \right)
    =
    \log \left( \exp (\hat \omega \theta) \right)
    =
    \hat \omega \theta
\end{equation*}
Now, to express \( \theta \) in terms of \( R \), we can see that, since the trace of a matrix is a linear operator, we have
\begin{align*}
    \operatorname{tr} \left( R \right)
     & =
    \operatorname{tr} \left( I + \sin \theta \hat \omega + (1 - \cos \theta) \hat \omega^2 \right)
    \\ & =
    \operatorname{tr} \left( I \right)
    + \operatorname{tr} \left( \sin \theta \hat \omega \right)
    + \operatorname{tr} \left( (1 - \cos \theta) \hat \omega^2 \right)
    \\ & =
    3
    + \sin \theta \operatorname{tr} \left( \hat \omega \right)
    + (1 - \cos \theta) \operatorname{tr} \left( \hat \omega^2 \right)
    =
    3 + 0 + (-2) (1 - \cos \theta)
    \\
    \implies
    \operatorname{tr} \left( R \right)
     & =
    1 + 2 \cos \theta
    \implies
    \theta
    =
    \arccos \left( \frac{\operatorname{tr} \left( R \right) - 1}{2} \right)
\end{align*}
To express \( \hat \omega \) in terms of \( R \), we have, since the transpose of a matrix is a linear operator,
\begin{align*}
    R^T
     & =
    I^T + \sin \theta \hat \omega^T + (1 - \cos \theta) {(\hat \omega^2)}^T
    =
    I - \sin \theta \hat \omega + (1 - \cos \theta) \hat \omega^2
\end{align*}
since \( \hat \omega^T = -\hat \omega \) and \( {(\hat \omega^2)}^T = {({\hat \omega}^T)}^2 = {(-\hat \omega)}^2 = \hat \omega^2 \).

\begin{equation*}
    \implies
    R - R^T
    =
    2 \sin \theta \hat \omega
    \implies
    \hat \omega
    =
    \frac{1}{2 \sin \theta} (R - R^T)
\end{equation*}
Thereby,
\begin{equation*}
    \log \left( R \right)
    =
    \hat \omega \theta
    =
    \frac{\theta}{2 \sin \theta} (R - R^T)
\end{equation*}
whenever \( \theta \neq 0 \), i.e., \( R \neq I \).
