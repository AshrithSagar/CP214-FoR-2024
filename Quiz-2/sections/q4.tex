\section*{Problem 4}

\textbf{Mobile Manipulator:}

\begin{figure}[h]
    \centering
    \includegraphics[width=0.7\textwidth]{figures/images/q4.jpg}
    \caption{
        Mobile manipulator
    }\label{fig:q4}
\end{figure}

In the given figure, a robot arm mounted on a wheeled mobile platform moving in a room, and a camera fixed to the ceiling.
Frames \( \{b\} \) and \( \{c\} \) are respectively attached to the wheeled platform and the end-effector (tip) of the robot arm, and frame \( \{\mathrm{d}\} \) is attached to the camera.
A fixed frame \( \{\mathrm{a}\} \) is considered on the ground, and the robot must pick up an object with body frame \( \{\mathrm{e}\} \).
Suppose that the transformations \( g_{d b} \) and \( g_{d e} \) can be calculated from measurements obtained with the camera.
The transformation \( g_{b c} \) can be calculated using the arm's joint-angle measurements.
The transformation \( g_{a d} \) is assumed to be known in advance.
Suppose these calculated and known transformations are given as follows:
\begin{align*}
     &
    g_{d b}
    =
    \begin{bmatrix}
        0  & 0  & -1 & 250  \\
        0  & -1 & 0  & -150 \\
        -1 & 0  & 0  & 200  \\
        0  & 0  & 0  & 1
    \end{bmatrix}
    , \quad
     &
    g_{d e}
    =
    \begin{bmatrix}
        0  & 0  & -1 & 300 \\
        0  & -1 & 0  & 100 \\
        -1 & 0  & 0  & 120 \\
        0  & 0  & 0  & 1
    \end{bmatrix},
    \\ &
    g_{a d}
    =
    \begin{bmatrix}
        0  & 0  & -1 & 400 \\
        0  & -1 & 0  & 50  \\
        -1 & 0  & 0  & 300 \\
        0  & 0  & 0  & 1
    \end{bmatrix}
    , \quad
     &
    g_{b c}
    =
    \begin{bmatrix}
        0 & -\frac{1}{\sqrt{2}} & -\frac{1}{\sqrt{2}} & 30 \\
        0 & \frac{1}{\sqrt{2}}  & -\frac{1}{\sqrt{2}} & 40 \\
        1 & 0                   & 0                   & 25 \\
        0 & 0                   & 0                   & 1
    \end{bmatrix}
    .
\end{align*}
Now determine \( g_{c e} \) which is the configuration of the object relative to the robot hand.
This is required in order to calculate how to move the robot arm so as to pick up the object.
You can keep the answers in terms of \( \sqrt{2} \).

\subsection*{Solution}
