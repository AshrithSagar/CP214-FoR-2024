\section*{Problem 6}

For each equation below state whether:
\begin{itemize}
    \item The equation is nonsense. If so, why?
    \item Is always true. Why?
    \item Is never true. Why?
    \item Is sometimes true. Why?
\end{itemize}

\begin{enumerate}[label = (\alph*)]
    \item \( \vec{b} \times \vec{c} = \vec{c} \times \vec{b} \)
    \item \( \vec{b} \times \vec{c} = \vec{c} \cdot \vec{b} \)
    \item \( \vec{c} \cdot (\vec{a} \times \vec{b}) = \vec{b} \cdot (\vec{c} \times \vec{a}) \)
    \item \( \vec{a} \times (\vec{b} \times \vec{c}) = (\vec{a} \cdot \vec{c})\ \vec{b} - (\vec{a} \cdot \vec{b})\ \vec{c} \)
\end{enumerate}
where: \( \cdot \) = dot product and \( \times \) = cross product by definition.

\subsection*{Solution}

\subsubsection*{(a) \( \vec{b} \times \vec{c} = \vec{c} \times \vec{b} \)}

The equation is sometimes true.

The cross product of two vectors is not commutative in general, i.e., \( \vec{a} \times \vec{b} \neq \vec{b} \times \vec{a} \).
In fact, the cross product of two vectors is anti-commutative, i.e., \( \vec{a} \times \vec{b} = -\vec{b} \times \vec{a} \).

In the case when the cross product of two vectors is commutative, it implies that the two vectors are parallel or anti-parallel.
Ins this case, the cross product evaluates to zero.
Therefore, the equation is sometimes true.

\subsubsection*{(b) \( \vec{b} \times \vec{c} = \vec{c} \cdot \vec{b} \)}

The equation is nonsense.

The LHS of the equation is a vector due to the cross product.
The RHS of the equation is a scalar due to the dot product.

The cross product of two vectors results in a vector that is orthogonal to the plane formed by the two vectors.
The dot product of two vectors results in a scalar that is the projection of one vector onto the other vector.

Therefore, it makes no sense to use them like this.
