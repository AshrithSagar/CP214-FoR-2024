\section*{Problem 4}

What is the distance \( d \) between the origin and the line \( AB \) shown?
(You may write your solution in terms of \( \vec{\mathbf{A}} \) and \( \vec{\mathbf{B}} \) before doing any arithmetic).

\begin{figure*}[h]
    \centering
    \includegraphics[width=0.5\linewidth]{sections/q4/figure.png}
\end{figure*}

\subsection*{Solution}

A simpler solution would be to realise that \( O, A, B \) form a triangle in the 3D space.
Each of the sides can be seen as being the hypotenuse for an isosceles right triangle with side length 1, thereby each of the segments \( OA, AB, BO \) are of length \( \sqrt{2} \).
Therefore, \( \triangle OAB \) is an equilateral triangle of side length \( \sqrt{2} \), whose altitude can be calculated using the following formula, valid for an equilateral triangle:

\[
    \text{Altitude} = \frac{\sqrt{3}}{2} \times \text{Side length} \\
\]

\[
    \implies d = \frac{\sqrt{3}}{2} \times \sqrt{2} = \frac{\sqrt{3}}{\sqrt{2}}
\]
