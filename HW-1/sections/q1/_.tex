\section*{Problem 1}

Find the percentage error in computing the moment of W about the pivot point O as a function of \( \theta \), if the weight is assumed to act normal to the arm OA (a good approximation when \( \theta \) is very small).

\begin{figure*}[h]
    \centering
    \includegraphics[width=0.5\linewidth]{sections/q1/figure.png}
\end{figure*}

\subsection*{Solution}

The moment of \( \vec{\mathbf{W}} \) about the pivot point O is given by
\[
    \vec{\tau} = \vec{\mathbf{OA}} \times \vec{\mathbf{W}}
    = l \times W \times \sin(90 + \theta)
    = W l \cos(\theta)
\]
The percentage error in computing \( \vec{\tau} \) is given by
\begin{align*}
    \text{Percentage error in } \vec{\tau}
     & =
    \frac{\text{Error in }\vec{\tau}}{\text{Actual }\vec{\tau}} \times 100\% \\
     & = \frac{W l - W l \cos(\theta)}{W l \cos(\theta)} \times 100\%
    = \frac{1 - \cos(\theta)}{\cos(\theta)} \times 100\%
\end{align*}
which is the percentage error in computing the moment of \( \vec{\mathbf{W}} \) about the pivot point O as a function of \( \theta \).
We can indeed verify that the percentage error is zero when \( \theta = 0 \) and increases as \( \theta \) deviates from 0.
