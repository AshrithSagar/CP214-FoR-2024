\section*{Problem 3}

\textbf{Planar Differential Drive Robot:}
Consider a mobile robot performing planar motion through a differential drive mechanism (see figure).
In a differential drive, we basically undergo some movement by varying the speed of a set of wheels (in this case \( W_{1} \) and \( W_{2} \)) on either side of the robot, to produces a linear and an angular velocity.
\begin{figure}[h]
    \centering
    \includegraphics[width=0.6\textwidth]{figures/images/q3.jpg}
\end{figure}

Any rigid body transformation \( g=(p, R) \in S E(2) \) in this case would consist of a translation \( p \in \mathbb{R}^{2} \) and a \( 2 \times 2 \) rotation matrix \( R \in S O(2) \).
We can represent this in homogeneous coordinates as a \( 3 \times 3 \) matrix:
\[
    g=\left[\begin{array}{ll}
            R & p \\
            0 & 1
        \end{array}\right]
\]
Moreover, any twist \( \hat{\xi} \in \operatorname{se}(2) \) can be represented by a \( 3 \times 3 \) matrix of the form:
\[
    \hat{\xi}=\left[\begin{array}{ll}
            \hat{\omega} & v \\
            0            & 0
        \end{array}\right], \hat{w}=\left[\begin{array}{cc}
            0      & -\omega \\
            \omega & 0
        \end{array}\right], \omega \in \mathbb{R}, v \in \mathbb{R}^{2}
\]
The twist coordinates for \( \hat{\xi} \) have the form \( \xi=(v, \omega) \in \mathbb{R}^{3} \).
These twists can be generated by controlling the speed of the wheels \( W_{1}, W_{2} \).
For this problem you don't have to worry about the exact formulation of this
(Don't need to find answers in terms of \( W_{1}, W_{2} \)).
Further, note that we can consider \( v \) as a vector in the plane and \( \omega \) as a scalar.

Now given this information, answer all the following questions.
\begin{enumerate}[label= (\alph*)]
    \item Show that the matrix exponential of a twist in \( s e(2) \) gives a rigid body transformation in \( S E(2) \).
          Consider both the pure translation case, \( \xi= \) \( (v, 0) \) and the general case, \( \xi=(v, \omega), \omega \neq 0 \).
    \item Show that the planar twists which correspond to pure rotation about a point \( r \) and pure translation in a direction \( v \) are given by,
          \[
              \xi=\left[\begin{array}{c}
                      r_{y}  \\
                      -r_{x} \\
                      1
                  \end{array}\right] \text { (pure rotation), } \xi=\left[\begin{array}{c}
                      v_{x} \\
                      v_{y} \\
                      0
                  \end{array}\right] \text { (pure translation) }
          \]
    \item Show that the matrices \( \hat{V}^{s}=\dot{g} g^{-1} \) and \( \hat{V}^{b}=g^{-1} \dot{g} \) are both twists.
          Define and interpret the spatial velocity \( V^{s} \in \mathbb{R}^{3} \) and the body velocity \( V^{b} \in \mathbb{R}^{3} \)
\end{enumerate}

\clearpage
\subsection*{Solution}
