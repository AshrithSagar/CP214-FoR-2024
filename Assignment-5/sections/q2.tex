\section*{Problem 2}

\textbf{Connection between \( s e(3) \) and \( S E(3) \)}:
\begin{enumerate}[label= (\alph*)]
      \item Show that given \( \hat{\xi} \in \operatorname{se}(3) \) and \( \theta \in \mathbb{R} \), the exponential of \( \hat{\xi \theta} \) is an element of \( S E(3) \), i.e., prove that \( e^{\widehat{\xi \theta}} \in S E(3) \).
      \item Let \( A=\left(I-e^{\hat{\omega} \theta}\right) \hat{\omega}+\omega \omega^{T} \theta \), where \( \theta \in(0,2 \pi), \hat{\omega} \in \operatorname{so}(3) \) and \( \omega \in \mathbb{R}^{3} \).
            Show that \( A: \mathbb{R}^{3} \mapsto \mathbb{R}^{3} \) is invertible.
      \item Given \( g \in S E(3) \), show that there exists \( \xi \in \operatorname{se}(3) \) and \( \theta \in \mathbb{R} \) such that \( g=\exp (\widehat{\xi \theta}) \)
            (You may require the result of part (b)).
\end{enumerate}

\subsection*{Solution}

\subsubsection*{(a) \( e^{\hat{\xi}\theta} \in SE(3) \)}

The special Euclidean group \( SE(3) \) is defined as
\begin{equation*}
      SE(3) = \left \{ (R, t) \; \middle| \; R \in SO(3), \, t \in \mathbb{R}^3 \right \}
\end{equation*}
or equivalently, in homogeneous coordinates as
\begin{equation*}
      SE(3) = \left \{ g = \begin{bmatrix}
            R & t \\
            0 & 1
      \end{bmatrix} \; \middle| \; R \in SO(3), \, t \in \mathbb{R}^3 \right \}
\end{equation*}
The group \( SE(3) \) is a Lie group, and it's Lie algebra \( se(3) \) is defined by
\begin{equation*}
      se(3) = \left \{ \hat \xi = \begin{bmatrix}
            \hat \omega & v \\
            0           & 0
      \end{bmatrix} \; \middle| \; \hat \omega \in so(3), \, v \in \mathbb{R}^3 \right \}
\end{equation*}
Now, given \( \hat{\xi} \in se(3) \) and \( \theta \in \mathbb{R} \), we can consider \( \lVert \omega \rVert = 1 \) without loss of generality.

Now, consider a matrix \( g \) defined as
\begin{equation*}
      g \triangleq
      \begin{bmatrix}
            I_{3\times 3} & \omega \times v \\
            0             & 1
      \end{bmatrix}_{4\times 4}
\end{equation*}
We can see that \( g \) is invertible, and is given by
\begin{equation*}
      \implies
      g^{-1} =
      \begin{bmatrix}
            I & - \omega \times v \\
            0 & 1
      \end{bmatrix}
\end{equation*}
Now, we can see that, for \( \hat{\xi}' \) defined through \( \hat \xi = g \hat{\xi}' g^{-1} \), we have
\begin{align*}
      \implies
      \hat{\xi}'
       & =
      g^{-1} \hat{\xi} g
      =
      \begin{bmatrix}
            I & - \hat \omega v \\
            0 & 1
      \end{bmatrix}
      \begin{bmatrix}
            \hat \omega & v \\
            0           & 0
      \end{bmatrix}
      \begin{bmatrix}
            I & \hat \omega v \\
            0 & 1
      \end{bmatrix}
      \\ & =
      \begin{bmatrix}
            I & - \hat \omega v \\
            0 & 1
      \end{bmatrix}
      \begin{bmatrix}
            \hat \omega & {\hat{\omega}}^2 v + v \\
            0           & 0
      \end{bmatrix}
      =
      \begin{bmatrix}
            \hat \omega & {\hat{\omega}}^2 v + v \\
            0           & 0
      \end{bmatrix}
\end{align*}
From the lemma which was proved earlier in previous assignments, we know that
\begin{equation*}
      {\hat{a}}^2 = a a^T - \lVert a \rVert I, \quad \forall a \in \mathbb{R}^3
\end{equation*}
which can now be applied for \( \hat \omega \) with \( \lVert \omega \rVert = 1 \), giving
\begin{equation*}
      {\hat{\omega}}^2
      =
      \omega \omega^T - I
      \implies
      {\hat{\omega}}^2 v + v
      =
      \omega \omega^T v - \cancel{v} + \cancel{v}
      =
      \omega \omega^T v
\end{equation*}
\begin{equation*}
      \implies
      \hat{\xi}'
      =
      \begin{bmatrix}
            \hat \omega & \omega \omega^T v \\
            0           & 0
      \end{bmatrix}
      =
      \begin{bmatrix}
            \hat \omega & h \omega \\
            0           & 0
      \end{bmatrix},
      \quad \text{where} \; h = (\omega^T v) \in \mathbb{R}
\end{equation*}
We have, by expanding through the matrix exponential, that
\begin{equation*}
      e^{\hat{\xi} \theta}
      =
      e^{g \hat{\xi}' g^{-1} \theta}
      =
      g e^{\hat{\xi}' \theta} g^{-1}
\end{equation*}
Now, \( e^{\hat{\xi}' \theta} \) can be obtained using the matrix exponential, by noticing that
\begin{equation*}
      \hat{\xi}^{'2}
      =
      \begin{bmatrix}
            \hat \omega & h \omega \\
            0           & 0
      \end{bmatrix}
      \begin{bmatrix}
            \hat \omega & h \omega \\
            0           & 0
      \end{bmatrix}
      =
      \begin{bmatrix}
            0 & 0 \\
            0 & 0
      \end{bmatrix}
\end{equation*}
\begin{align*}
      \implies
      e^{\hat{\xi}' \theta}
       & =
      I + \hat{\xi}' \theta + \cancel{\frac{1}{2!} \hat{\xi}^{'2} \theta^2 + \frac{1}{3!} \hat{\xi}^{'3} \theta^3 + \ldots}
      =
      I_{4\times 4} + \hat{\xi}' \theta
      \\ & =
      \begin{bmatrix}
            I_{3\times 3} & 0 \\
            0             & 1
      \end{bmatrix}
      +
      \begin{bmatrix}
            \hat \omega \theta & h \theta \omega \\
            0                  & 0
      \end{bmatrix}
      =
      \begin{bmatrix}
            I + \hat \omega \theta & h \theta \omega \\
            0                      & 1
      \end{bmatrix}
      \\
      \implies
      e^{\hat{\xi} \theta}
       & =
      g e^{\hat{\xi}' \theta} g^{-1}
      =
      \begin{bmatrix}
            I & \omega \times v \\
            0 & 1
      \end{bmatrix}
      \begin{bmatrix}
            I + \hat \omega \theta & h \theta \omega \\
            0                      & 1
      \end{bmatrix}
      \begin{bmatrix}
            I & - \omega \times v \\
            0 & 1
      \end{bmatrix}
      \\ & =
      \begin{bmatrix}
            I & \omega \times v \\
            0 & 1
      \end{bmatrix}
      \begin{bmatrix}
            I + \hat \omega \theta & -(I + \hat \omega \theta)(\omega \times v) + h \theta \omega \\
            0                      & 1
      \end{bmatrix}
      \\ & =
      \begin{bmatrix}
            I + \hat \omega \theta & \omega \times v - \omega \times v + h \theta \omega \\
            0                      & 1
      \end{bmatrix}
\end{align*}

\subsubsection*{(b) \( A = \left(I-e^{\hat{\omega} \theta}\right) \hat{\omega}+\omega \omega^{T} \theta \) is invertible}

\subsubsection*{(c) \( \exists \; \xi \in se(3), \, \theta \in \mathbb{R} \) such that \( g = \exp(\hat{\xi} \theta) \)}
