\section*{Problem 2}
\setcounter{section}{2}
\setcounter{equation}{0}

\textbf{Connection between \( s e(3) \) and \( S E(3) \)}:
\begin{enumerate}[label= (\alph*)]
      \item Show that given \( \hat{\xi} \in \operatorname{se}(3) \) and \( \theta \in \mathbb{R} \), the exponential of \( \hat{\xi \theta} \) is an element of \( S E(3) \), i.e., prove that \( e^{\widehat{\xi \theta}} \in S E(3) \).
      \item Let \( A=\left(I-e^{\hat{\omega} \theta}\right) \hat{\omega}+\omega \omega^{T} \theta \), where \( \theta \in(0,2 \pi), \hat{\omega} \in \operatorname{so}(3) \) and \( \omega \in \mathbb{R}^{3} \).
            Show that \( A: \mathbb{R}^{3} \mapsto \mathbb{R}^{3} \) is invertible.
      \item Given \( g \in S E(3) \), show that there exists \( \xi \in \operatorname{se}(3) \) and \( \theta \in \mathbb{R} \) such that \( g=\exp (\widehat{\xi \theta}) \)
            (You may require the result of part (b)).
\end{enumerate}

\subsection*{Solution}

\subsubsection*{(a) \( e^{\hat{\xi}\theta} \in SE(3) \)}

The special Euclidean group \( SE(3) \) is defined as
\begin{equation*}
      SE(3) = \left \{ (R, t) \; \middle| \; R \in SO(3), \, t \in \mathbb{R}^3 \right \}
\end{equation*}
or equivalently, in homogeneous coordinates as
\begin{equation}\label{eq:SE3-homogeneous}
      SE(3) = \left \{ g = \begin{bmatrix}
            R & t \\
            0 & 1
      \end{bmatrix} \; \middle| \; R \in SO(3), \, t \in \mathbb{R}^3 \right \}
\end{equation}
The group \( SE(3) \) is a Lie group, and it's Lie algebra \( se(3) \) is defined by
\begin{equation*}
      se(3) = \left \{ \hat \xi = \begin{bmatrix}
            \hat \omega & v \\
            0           & 0
      \end{bmatrix} \; \middle| \; \hat \omega \in so(3), \, v \in \mathbb{R}^3 \right \}
\end{equation*}
Now, given \( \hat{\xi} \in se(3) \) and \( \theta \in \mathbb{R} \), we can consider \( \lVert \omega \rVert = 1 \) without loss of generality.

Now, consider a matrix \( g \) defined as
\begin{equation*}
      g \triangleq
      \begin{bmatrix}
            I_{3\times 3} & \omega \times v \\
            0             & 1
      \end{bmatrix}_{4\times 4}
\end{equation*}
We can see that \( g \) is invertible, and is given by
\begin{equation*}
      \implies
      g^{-1} =
      \begin{bmatrix}
            I & - \omega \times v \\
            0 & 1
      \end{bmatrix}
\end{equation*}
We can see that both \( g \) and \( g^{-1} \) are in \(SE(3)\), by equation~\eqref{eq:SE3-homogeneous}.\\
Now, we define \( \hat{\xi}' \) through \( \hat \xi = g \hat{\xi}' g^{-1} \), thereby we have
\begin{align*}
      \implies
      \hat{\xi}'
       & =
      g^{-1} \hat{\xi} g
      =
      \begin{bmatrix}
            I & - \hat \omega v \\
            0 & 1
      \end{bmatrix}
      \begin{bmatrix}
            \hat \omega & v \\
            0           & 0
      \end{bmatrix}
      \begin{bmatrix}
            I & \hat \omega v \\
            0 & 1
      \end{bmatrix}
      \\ & =
      \begin{bmatrix}
            I & - \hat \omega v \\
            0 & 1
      \end{bmatrix}
      \begin{bmatrix}
            \hat \omega & {\hat{\omega}}^2 v + v \\
            0           & 0
      \end{bmatrix}
      =
      \begin{bmatrix}
            \hat \omega & {\hat{\omega}}^2 v + v \\
            0           & 0
      \end{bmatrix}
\end{align*}
from which we can see that \( \hat{\xi}' \) is in \( se(3) \).\\
From the lemma which was proved earlier in previous assignments, we know that
\begin{equation*}
      {\hat{a}}^2 = a a^T - \lVert a \rVert I, \quad \forall a \in \mathbb{R}^3
\end{equation*}
which can now be applied for \( \hat \omega \) with \( \lVert \omega \rVert = 1 \), giving
\begin{equation*}
      {\hat{\omega}}^2
      =
      \omega \omega^T - I
      \implies
      {\hat{\omega}}^2 v + v
      =
      \omega \omega^T v - \cancel{v} + \cancel{v}
      =
      \omega \omega^T v
\end{equation*}
\begin{equation*}
      \implies
      \hat{\xi}'
      =
      \begin{bmatrix}
            \hat \omega & \omega \omega^T v \\
            0           & 0
      \end{bmatrix}
      =
      \begin{bmatrix}
            \hat \omega & h \omega \\
            0           & 0
      \end{bmatrix},
      \quad \text{where} \; h = (\omega^T v) \in \mathbb{R}
\end{equation*}
We have, by expanding through the matrix exponential, that
\begin{equation*}
      e^{\hat{\xi} \theta}
      =
      e^{g \hat{\xi}' g^{-1} \theta}
      =
      g e^{\hat{\xi}' \theta} g^{-1}
\end{equation*}
Now, \( e^{\hat{\xi}' \theta} \) can be obtained using the matrix exponential, by noticing that
\begin{align*}
      \hat{\xi}^{'2}
       & =
      \begin{bmatrix}
            \hat \omega & h \omega \\
            0           & 0
      \end{bmatrix}
      \begin{bmatrix}
            \hat \omega & h \omega \\
            0           & 0
      \end{bmatrix}
      =
      \begin{bmatrix}
            \hat{\omega}^2 & h \cancel{\hat{\omega} \omega} \\
            0              & 0
      \end{bmatrix}
      =
      \begin{bmatrix}
            \hat{\omega}^2 & 0 \\
            0              & 0
      \end{bmatrix}
      \\
      \hat{\xi}^{'3}
       & =
      \begin{bmatrix}
            \hat{\omega}^2 & 0 \\
            0              & 0
      \end{bmatrix}
      \begin{bmatrix}
            \hat \omega & h \omega \\
            0           & 0
      \end{bmatrix}
      =
      \begin{bmatrix}
            \hat{\omega}^3 & h \cancel{\hat{\omega}^2 \omega} \\
            0              & 0
      \end{bmatrix}
      =
      \begin{bmatrix}
            \hat{\omega}^3 & 0 \\
            0              & 0
      \end{bmatrix}
      \\
      \implies
      \hat{\xi}^{'k}
       & =
      \begin{bmatrix}
            \hat{\omega}^k & 0 \\
            0              & 0
      \end{bmatrix}
      \quad \forall k \geq 2
      \\
      \implies
      e^{\hat{\xi}' \theta}
       & =
      I + \hat{\xi}' \theta + \frac{1}{2!} \hat{\xi}^{'2} \theta^2 + \frac{1}{3!} \hat{\xi}^{'3} \theta^3 + \ldots
      \\ & =
      \begin{bmatrix}
            I_{3\times 3} & 0 \\
            0             & 1
      \end{bmatrix}
      +
      \begin{bmatrix}
            \hat \omega \theta & h \theta \omega \\
            0                  & 0
      \end{bmatrix}
      +
      \frac{1}{2!}
      \begin{bmatrix}
            \hat \omega^2 \theta^2 & 0 \\
            0                      & 0
      \end{bmatrix}
      +
      \frac{1}{3!}
      \begin{bmatrix}
            \hat \omega^3 \theta^3 & 0 \\
            0                      & 0
      \end{bmatrix}
      + \ldots
      \\ & =
      \begin{bmatrix}
            \left( I_{3\times 3} + \hat \omega \theta + \frac{1}{2!} \hat \omega^2 \theta^2 + \frac{1}{3!} \hat \omega^3 \theta^3 + \ldots \right)
              &
            h \theta \omega
            \\
            0 & 1
      \end{bmatrix}
      =
      \begin{bmatrix}
            e^{\hat \omega \theta} & h \theta \omega \\
            0                      & 1
      \end{bmatrix}
      \\
      \implies
      e^{\hat{\xi} \theta}
       & =
      g e^{\hat{\xi}' \theta} g^{-1}
      =
      \begin{bmatrix}
            I & \hat \omega v \\
            0 & 1
      \end{bmatrix}
      \begin{bmatrix}
            e^{\hat \omega \theta} & h \theta \omega \\
            0                      & 1
      \end{bmatrix}
      \begin{bmatrix}
            I & - \hat \omega v \\
            0 & 1
      \end{bmatrix}
      \\ & =
      \begin{bmatrix}
            I & \hat \omega v \\
            0 & 1
      \end{bmatrix}
      \begin{bmatrix}
            e^{\hat \omega \theta}
              &
            -e^{\hat \omega \theta} \hat \omega v + h \theta \omega
            \\
            0 & 1
      \end{bmatrix}
      =
      \begin{bmatrix}
            e^{\hat \omega \theta}
              &
            -e^{\hat \omega \theta} \hat \omega v + h \theta \omega + \hat \omega v
            \\
            0 & 1
      \end{bmatrix}
      \\ & =
      \begin{bmatrix}
            e^{\hat \omega \theta}
              &
            (I - e^{\hat \omega \theta}) \hat \omega v + h \theta \omega
            \\
            0 & 1
      \end{bmatrix}
      =
      \begin{bmatrix}
            e^{\hat \omega \theta}
              &
            (I - e^{\hat \omega \theta}) \hat \omega v + (\omega^T v \theta) \omega
            \\
            0 & 1
      \end{bmatrix}
\end{align*}
\begin{equation}\label{eq:SE3-exponential}
      \implies
      e^{\hat{\xi} \theta}
      =
      \begin{bmatrix}
            e^{\hat \omega \theta}
              &
            (I - e^{\hat \omega \theta}) \hat \omega v + \omega \omega^T v \theta
            \\
            0 & 1
      \end{bmatrix}
\end{equation}
Since \( e^{\hat \omega \theta} \in SO(3) \) and \( \Big( (I - e^{\hat \omega \theta}) \hat \omega v + \omega \omega^T v \theta \Big) \in \mathbb{R}^3 \), we have that \( e^{\hat{\xi} \theta} \) satisfies equation~\eqref{eq:SE3-homogeneous}.

Note that for the case \( \omega = 0 \), we can see that \( \hat{xi}^k = 0, \forall k \geq 2 \implies e^{\hat{\xi} \theta} = I + \hat{\xi} \theta \).
\begin{align*}
      e^{\hat{\xi} \theta}
       & =
      \begin{bmatrix}
            I_{3\times 3} & v \theta \\
            0             & 1
      \end{bmatrix}
      \in
      SE(3)
\end{align*}

Thereby, we have that \underline{\( e^{\hat{\xi} \theta} \in SE(3) \)}.

\clearpage
\subsubsection*{(b) \( A = \left(I-e^{\hat{\omega} \theta}\right) \hat{\omega}+\omega \omega^{T} \theta \) is invertible}

Given \( A = \left(I-e^{\hat{\omega} \theta}\right) \hat{\omega}+\omega \omega^{T} \theta \), we can see that
\begin{align*}
      A
       & =
      \left(I-e^{\hat{\omega} \theta}\right) \hat{\omega}+\omega \omega^{T} \theta
      =
      \hat{\omega} - e^{\hat{\omega} \theta} \hat{\omega} + \omega \omega^{T} \theta
\end{align*}

\subsubsection*{(c) \( \exists \; \xi \in se(3), \, \theta \in \mathbb{R} \) such that \( g = \exp(\hat{\xi} \theta) \)}

Given \( g \in SE(3) \), we can see that \( g \) can be represented as
\begin{equation*}
      g
      =
      \begin{bmatrix}
            R & t \\
            0 & 1
      \end{bmatrix}
\end{equation*}

In the case of \( R = I \), we can see that with
\begin{equation*}
      \hat \xi =
      \begin{bmatrix}
            0 & \frac{t}{\lVert t \rVert} \\
            0 & 0
      \end{bmatrix}
\end{equation*}
and \( \theta = \lVert t \rVert \), we have that
\begin{align*}
      \exp(\hat{\xi} \theta)
       & =
      \exp\left(
      \begin{bmatrix}
                  0 & \frac{t}{\lVert t \rVert} \lVert t \rVert \\
                  0 & 0
            \end{bmatrix}
      \right)
      =
      \exp\left(
      \begin{bmatrix}
                  0 & t \\
                  0 & 0
            \end{bmatrix}
      \right)
      =
      \begin{bmatrix}
            I & t \\
            0 & 1
      \end{bmatrix}
      =
      g
\end{align*}

In the case of \( R \neq I \), we have, from equation~\eqref{eq:SE3-exponential}, that
\begin{equation*}
      g
      =
      \begin{bmatrix}
            e^{\hat \omega \theta}
              &
            (I - e^{\hat \omega \theta}) \hat \omega v + \omega \omega^T v \theta
            \\
            0 & 1
      \end{bmatrix}
\end{equation*}
Thereby, we can set \( R = e^{\hat \omega \theta} \) and \( t = (I - e^{\hat \omega \theta}) \hat \omega v + \omega \omega^T v \theta \), and thereby we can have \( g = \exp(\hat{\xi} \theta) \).
To find \( \omega \), we can use the matrix logarithm, as
\begin{equation*}
      \hat\omega = \frac{1}{\theta} \log(R) = \frac{1}{2 \sin(\theta)} \left( R - R^T \right)
\end{equation*}
Now, to express \( v \) in terms of \( t \), we use the fact that \( A = \left(I-e^{\hat{\omega} \theta}\right) \hat{\omega}+\omega \omega^{T} \theta \) is invertible,
\begin{align*}
      t
       & =
      (I - e^{\hat \omega \theta}) \hat \omega v + \omega \omega^T v \theta
      =
      \Big( (I - e^{\hat \omega \theta}) \hat \omega + \omega \omega^T \theta \Big) v
      =
      A v
      \implies
      v
      =
      A^{-1} t
\end{align*}
Thereby, \underline{given \( g \in SE(3) \), we can find \( \xi \in se(3) \) and \( \theta \in \mathbb{R} \) such that \( g = \exp(\hat{\xi} \theta) \)}.
