\section*{Problem 2}

\textbf{Connection between \( s e(3) \) and \( S E(3) \)}:
\begin{enumerate}[label= (\alph*)]
    \item Show that given \( \hat{\xi} \in \operatorname{se}(3) \) and \( \theta \in \mathbb{R} \), the exponential of \( \hat{\xi \theta} \) is an element of \( S E(3) \), i.e., prove that \( e^{\widehat{\xi \theta}} \in S E(3) \).
    \item Let \( A=\left(I-e^{\hat{\omega} \theta}\right) \hat{\omega}+\omega \omega^{T} \theta \), where \( \theta \in(0,2 \pi), \hat{\omega} \in \operatorname{so}(3) \) and \( \omega \in \mathbb{R}^{3} \).
          Show that \( A: \mathbb{R}^{3} \mapsto \mathbb{R}^{3} \) is invertible.
    \item Given \( g \in S E(3) \), show that there exists \( \xi \in \operatorname{se}(3) \) and \( \theta \in \mathbb{R} \) such that \( g=\exp (\widehat{\xi \theta}) \)
          (You may require the result of part (b)).
\end{enumerate}

\subsection*{Solution}

\subsubsection*{(a) \( e^{\hat{\xi}\theta} \in SE(3) \)}

\subsubsection*{(b) \( A = \left(I-e^{\hat{\omega} \theta}\right) \hat{\omega}+\omega \omega^{T} \theta \) is invertible}

\subsubsection*{(c) \( \exists \; \xi \in se(3), \, \theta \in \mathbb{R} \) such that \( g = \exp(\hat{\xi} \theta) \)}
