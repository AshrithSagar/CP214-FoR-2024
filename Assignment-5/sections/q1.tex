\section*{Problem 1}
\setcounter{section}{1}

Show that given \( R(t) \in SO(3) \), the matrices \( \dot{R}(t) {R(t)}^{-1} \in \mathbb{R}^{3 \times 3} \) and \( R(t) {\dot{R}(t)}^{-1} \in \mathbb{R}^{3 \times 3} \) are skew-symmetric.

\subsection*{Solution}

We know that the special orthogonal group \( SO(3) \) is defined as
\begin{equation*}
    SO(3) = \left \{ R \in \mathbb{R}^{3 \times 3} \; \middle| \; R R^{T} = I, \, \det(R) = 1 \right \}
\end{equation*}
thereby, for \( R(t) \in SO(3) \), we have that
\begin{equation}\label{eq:RRT}
    R(t) {R(t)}^{T} = I
\end{equation}
and
\begin{equation}\label{eq:RTR}
    {R(t)}^{T} R(t) = I
\end{equation}

Differentiating both sides of equation~\eqref{eq:RRT} with respect to \( t \), we get
\begin{equation*}
    \implies
    \dot{R}(t) {R(t)}^{T}
    + R(t) {\dot{R}(t)}^{T}
    = 0
\end{equation*}
Note the order of the terms, since matrix multiplication is not commutative, in general.
\begin{equation*}
    \implies
    \dot{R}(t) {R(t)}^{T}
    =
    -R(t) {\dot{R}(t)}^{T}
    =
    - {\left( \dot{R}(t) {R(t)}^{T} \right)}^{T}
\end{equation*}
Thereby, we have that \( \dot{R}(t) {R(t)}^{T} \) is skew-symmetric.
By equation~\eqref{eq:RRT} and~\eqref{eq:RTR}, we also have that \( {R(t)}^{T} = {R(t)}^{-1} \).
Thereby, we have that \underline{\( \dot{R}(t) {R(t)}^{-1} \) is skew-symmetric}.

Similarly, by differentiating the equation~\eqref{eq:RTR} with respect to \( t \), we get
\begin{equation*}
    \implies
    {\dot{R}(t)}^{T} R(t)
    + {R(t)}^{T} \dot{R}(t)
    = 0
\end{equation*}
\begin{equation*}
    \implies
    {\dot{R}(t)}^{T} R(t)
    =
    - {R(t)}^{T} \dot{R}(t)
    =
    - {\left( {\dot{R}(t)}^{T} R(t) \right)}^{T}
\end{equation*}
