\section*{Problem 1}
\setcounter{section}{1}
\setcounter{equation}{0}

\textbf{Cayley-Rodrigues Relationships}:
Recall the Cayley parametrization of \( S O(3) \) from earlier assignment.
We can generalize it to higher orders as follows:
\[
    R=(I-\widehat{a})^{k}(I+\widehat{a})^{-k} \in S O(3)
\]
Here, \( a \in \mathbb{R}^{3} \) are the Cayley parameters.
Now answer the following,
\begin{enumerate}[label= (\alph*)]
    \item (Case \( k=1 \))
          Show that given \( R=e^{\widehat{\omega} \theta} \) for some unit vector \( \omega \) along the axis of rotation and angle \( \theta \), then \( a=-\omega \tan (\theta / 2) \).

          (Hint: Plug in the value of \( \widehat{a} \) and \( e^{\widehat{w} \theta} \) in LHS of \( R(I+\widehat{a})=I-\widehat{a} \) and verify that it matches the RHS.\@
          You may need to recall the properties of \( \widehat{\omega} \in \operatorname{so}(3) \))

          Furthermore, also prove below relationships:
          \[
              R=\frac{\left(1-a^{T} a\right) I+2 a a^{T}+2 \widehat{a}}{1+a^{T} a}, \quad \widehat{a}=\frac{R-R^{T}}{1+\operatorname{trace}(R)} .
          \]

          (Hint: For the first relation, substitute the value of \( \widehat{a} \) and simplify the expression for it to match with \( e^{\widehat{\omega} \theta} \).
          As for the second relation, simply use the first relation in RHS and show that it matches LHS, also recall that \( \operatorname{trace}(A+B)=\operatorname{trace}(A)+\operatorname{trace}(B) \))

          Using these, can we know what the vectors ``\( -a \)'' and ``\( a=0 \)'' represent?

    \item (Case \( k=2 \)) Now show that the rotation \( R \) corresponding to \( a \) can be computed from the formula
          \[
              R=I-4 \frac{1-a^{T} a}{\left(1+a^{T} a\right)^{2}} \widehat{a}+\frac{8}{\left(1+a^{T} a\right)^{2}} \widehat{a}^{2}
          \]

          (Hint: Since we don't start off with knowing the form of \( \widehat{a} \), try to prove that LHS is equal to RHS for \( R(I+\widehat{a})^{2}=(I-\widehat{a})^{2} \) by using the formula of \( e^{\widehat{\omega} \theta} \))

          Conversely, prove that given a rotation matrix \( R \), there exists a vector \( a \) which satisfies the above and can be obtained as \( a=-\omega \tan (\theta / 4) \), where \( \omega \) is the unit vector along the axis of rotation for \( R \), and \( \theta \) is the corresponding rotation angle.
          Is this solution unique or not?

          Finally, show that the angular velocity in the body frame obeys the following relation:
          \[
              \dot{a}=\frac{1}{4}\left(\left(1-a^{T} a\right) I+2 \widehat{a}+2 a a^{T}\right) \omega_{b} .
          \]

    \item For \( k \geq 2 \), explain what happens to the singularity at \( \pi \) that exists for the standard Cayley parameters \( (k=1) \).
          Also, discuss the relative advantages and disadvantages of choosing different \( k \).

    \item An attractive feature of the Cayley parameters is the simple form for the composition of two rotation matrices.
          If \( a_{1} \) and \( a_{2} \) denote the Cayley parameters for two rotations \( R_{1} \) and \( R_{2} \) respectively, then show that for \( R_{3}=R_{1} R_{2} \) we would have the corresponding \( a_{3} \) given by
          \[
              a_{3}=\frac{a_{1}+a_{2}+\left(a_{1} \times a_{2}\right)}{1-a_{1}^{T} a_{2}}
          \]

          Compare the number of arithmetic operations needed for multiplying two rotation matrices or two Cayley representations.
          Which requires the fewest arithmetic operations?

          [Extra]:

          Lastly, show that we can even find body frame and spatial frame angular velocities in a simple form like
          \[
              \omega_{s}=\frac{2}{1+\|a\|^{2}}(a \times \dot{a}+\dot{a}), \quad \omega_{b}=\frac{2}{1+\|a\|^{2}}(-a \times \dot{a}+\dot{a})
          \]

          (What would happen in the case when trace \( (R)=-1 \) or \( a_{1}^{T} a_{2}=1 \equiv \operatorname{trace}\left(R_{1} R_{2}\right)=-1 \)?
          Does everything fail, or can we still work around it?
          Maybe try deriving everything in terms of a variable \( s=\frac{a}{\sqrt{1+a^{T} a}} \) and see where it leads to \( \ldots \))
\end{enumerate}

\subsection*{Solution}

We can see that the Cayley parametrization satisfies the definition of elements in \( SO(3) \).
\begin{align*}
    \implies
    R R^{\top}
     & =
    \left[
        {(I - \widehat{a})}^{k}
            {(I + \widehat{a})}^{-k}
        \right]
    {\left[
            {(I - \widehat{a})}^{k}
                {(I + \widehat{a})}^{-k}
            \right]}^{\top}
    \\ & =
    \left[
        {(I - \widehat{a})}^{k}
            {(I + \widehat{a})}^{-k}
        \right]
    \left[
        {\left({(I + \widehat{a})}^{-k}\right)}^{\top}
            {\left({(I - \widehat{a})}^{k}\right)}^{\top}
        \right]
    \\ & =
    \left[
        {(I - \widehat{a})}^{k}
            {(I + \widehat{a})}^{-k}
        \right]
    \left[
        {\left({(I + \widehat{a})}^{\top}\right)}^{-k}
            {\left({(I - \widehat{a})}^{\top}\right)}^{k}
        \right]
    \\ & =
    \left[
        {(I - \widehat{a})}^{k}
            {(I + \widehat{a})}^{-k}
        \right]
    \left[
        {(I - \widehat{a})}^{-k}
            {(I + \widehat{a})}^{k}
        \right]
    \\ & =
    {(I - \widehat{a})}^{k}
    \left[
        {(I + \widehat{a})}^{-k}
            {(I - \widehat{a})}^{-k}
        \right]
    {(I + \widehat{a})}^{k}
    \\ & =
    {(I - \widehat{a})}^{k}
    \left[
        {(I - \widehat{a}^{2})}^{-k}
        \right]
    {(I + \widehat{a})}^{k}
    \\ & =
    {(I - \widehat{a})}^{k}
    \left[
        {(I - \widehat{a})}^{-k}
            {(I + \widehat{a})}^{-k}
        \right]
    {(I + \widehat{a})}^{k}
    \\ & =
    \left[
        {(I - \widehat{a})}^{k} {(I - \widehat{a})}^{-k}
        \right]
    \left[
        {(I - \widehat{a})}^{-k} {(I - \widehat{a})}^{k}
        \right]
    =
    (I)(I)
    =
    I
\end{align*}
\begin{equation*}
    \implies
    \det R
    =
    \det
    \left[
        {(I - \widehat{a})}^{k}
            {(I + \widehat{a})}^{-k}
        \right]
    =
    {\left[
        \det (I - \widehat{a})
        {\det (I + \widehat{a})}^{-1}
        \right]}^{k}
    =
    1^{k}
    =
    1
\end{equation*}
\begin{equation}\label{eq:cayley-k}
    \implies
    R = {(I - \widehat{a})}^{k} {(I + \widehat{a})}^{-k} \in SO(3)
\end{equation}

\subsubsection*{(a) Case \( k=1 \)}

Given \( R = e^{\widehat{\omega} \theta} \in SO(3) \) for some unit vector \( \omega \) along the axis of rotation and angle \( \theta \), we have to show that \( a = -\omega \tan(\theta / 2) \), where \( a \) is the Cayley parameter for \( R \), i.e. \( R \) satisfies \( R = (I - \widehat{a}) {(I + \widehat{a})}^{-1} \).
For this, we will first find the value of \( \widehat{a} \) and then substitute it in the equation \( R(I + \widehat{a}) \) to verify that it matches the RHS, as given in the hint.

From the Rodrigues formula, we have
\begin{equation*}
    R = e^{\widehat{\omega} \theta} = I + \widehat{\omega} \sin \theta + \widehat{\omega}^{2} (1 - \cos \theta)
\end{equation*}
Now, with \( a = -\omega \tan(\theta / 2) \), we have \( \widehat{a} = -\widehat{\omega} \tan(\theta / 2) \), which is clear to see based on the matrix representation of the hat operator.
Thus, we have
\begin{align*}
    \implies
    R(I + \widehat{a})
     & =
    R(I - \widehat{\omega} \tan(\theta / 2))
    \\ & =
    (I + \widehat{\omega} \sin \theta + \widehat{\omega}^{2} (1 - \cos \theta))(I - \widehat{\omega} \tan(\theta / 2))
    \\ & =
    I + \widehat{\omega} \sin \theta + \widehat{\omega}^{2} (1 - \cos \theta) - \widehat{\omega} \tan(\theta / 2)
    \\ & \quad
    - \widehat{\omega}^{2} \sin \theta \tan(\theta / 2) - \widehat{\omega}^{3} (1 - \cos \theta) \tan(\theta / 2)
    \\ & =
    I + \widehat{\omega} \sin \theta + \widehat{\omega}^{2} (1 - \cos \theta) - \cancel{\widehat{\omega} \tan(\theta / 2)}
    \\ & \quad
    - \widehat{\omega}^{2} \sin \theta \tan(\theta / 2) + \cancel{\widehat{\omega} \tan(\theta / 2)} - \widehat{\omega} \cos \theta \tan(\theta / 2)
    \\ & =
    I + \widehat{\omega} \left[ \sin \theta - \cos \theta \tan(\theta / 2) \right] + \widehat{\omega}^{2} \left[(1 - \cos \theta) - \sin \theta \tan(\theta / 2) \right]
\end{align*}
where we've used the relation \( \widehat{\omega}^{3} = -\widehat{\omega} \), which was derived in the previous assignments.
[Seen in \( \hat{w}^{3}=-\|w\|^{2} \hat{w} \) for \( \hat{w} \in so(3) \), and here \( \|w\|=1 \)].

Now, from the half-angle formulas, we have
\begin{equation*}
    \tan (\theta / 2)
    =
    \frac{\sin \theta}{1 + \cos \theta}
    =
    \frac{1 - \cos \theta}{\sin \theta}
\end{equation*}
\begin{align*}
    \implies
    R(I + \widehat{a})
     & =
    I + \widehat{\omega} \left[ \sin \theta - \cos \theta \frac{\sin \theta}{1 + \cos \theta} \right] + \widehat{\omega}^{2} \left[\cancel{(1 - \cos \theta)} - \cancel{\sin \theta \tan(\theta / 2)} \right]
    \\ & =
    I + \widehat{\omega} \sin \theta \left[ 1 - \frac{\cos \theta}{1 + \cos \theta} \right]
    =
    I + \widehat{\omega} \sin \theta \left[ \frac{1 + \cancel{\cos \theta} - \cancel{\cos \theta}}{1 + \cos \theta} \right]
    \\ & =
    I + \widehat{\omega} \frac{\sin \theta}{1 + \cos \theta}
    =
    I + \widehat{\omega} \tan(\theta / 2)
    \\ & =
    I - (-\widehat{\omega} \tan(\theta / 2))
    =
    (I - \widehat{a})
\end{align*}

Thus, we have shown that \( R(I + \widehat{a}) = I - \widehat{a} \implies \boxed{a = -\omega \tan(\theta / 2)} \) holds.

Now, we want to prove the relationships,
\[
    R = \frac{(1 - a^{\top} a) I + 2 a a^{\top} + 2 \widehat{a}}{1 + a^{\top} a}, \quad \widehat{a} = \frac{R - R^{\top}}{1 + \operatorname{trace}(R)}
\]

With \( a = -\omega \tan(\theta / 2) \), we have
\begin{align*}
    a^\top
     & =
    {(-\omega \tan(\theta / 2))}^\top
    =
    -\tan(\theta / 2) \omega^\top
    \\
    \implies
    a^{\top} a
     & =
    \tan^{2}(\theta / 2) \omega^\top \omega
    =
    \tan^{2}(\theta / 2) \| \omega \|^{2}
    =
    \tan^{2}(\theta / 2)
    \\
    \implies
    1 + a^{\top} a
     & =
    1 + \tan^{2}(\theta / 2)
    =
    \sec^{2}(\theta / 2)
    =
    \frac{1}{\cos^{2}(\theta / 2)}
    \\
    \implies
    1 - a^{\top} a
     & =
    1 - \tan^{2}(\theta / 2)
    =
    1 - \frac{\sin^{2}(\theta / 2)}{\cos^{2}(\theta / 2)}
    =
    \frac{\cos^{2}(\theta / 2) - \sin^{2}(\theta / 2)}{\cos^{2}(\theta / 2)}
    \\
    \implies
    a a^{\top}
     & =
    \tan^{2}(\theta / 2) \omega \omega^{\top}
    =
    \tan^{2}(\theta / 2) (\widehat{\omega}^2 + I)
\end{align*}
\begin{align*}
    \implies
     &
    \frac{(1 - a^{\top} a) I + 2 a a^{\top} + 2 \widehat{a}}{1 + a^{\top} a}
    =
    \frac{\left( \frac{\cos^{2}(\theta / 2) - \sin^{2}(\theta / 2)}{\cos^{2}(\theta / 2)} \right) I + 2 \tan^{2}(\theta / 2) (\widehat{\omega}^2 + I) + 2 \widehat{a}}{\sec^{2}(\theta / 2)}
    \\ & =
    \left( \cos^{2}(\theta / 2) - \sin^{2}(\theta / 2) \right) I + 2 \cos^{2}(\theta / 2) \tan^{2}(\theta / 2) (\widehat{\omega}^2 + I) + 2 \cos^{2}(\theta / 2) \widehat{a}
    \\ & =
    \cos(\theta) I + 2 \sin^{2}(\theta / 2) (\widehat{\omega}^2 + I) - 2 \cos^{2}(\theta / 2) \tan(\theta / 2) \widehat{w}
    \\ & =
    \cos(\theta) I + (1 - \cos(\theta)) (\widehat{\omega}^2 + I) - \sin(\theta / 2) \widehat{w}
    \\ & =
    I - \sin(\theta / 2) \widehat{w} + (1 - \cos(\theta)) \widehat{\omega}^2
    =
    R
\end{align*}
by the Rodrigues formula.
Thus, we have shown that \( \boxed{R = \frac{(1 - a^{\top} a) I + 2 a a^{\top} + 2 \widehat{a}}{1 + a^{\top} a}} \).

Now, we have to show that \( \displaystyle \widehat{a} = \frac{R - R^{\top}}{1 + \operatorname{tr}(R)} \).
We have
\begin{align*}
    R - R^\top
     & =
    \frac{\cancel{(1 - a^{\top} a) I} + \cancel{2 a a^{\top}} + 2 \widehat{a}}{1 + a^{\top} a} - \frac{\cancel{(1 - a^{\top} a) I^\top} + \cancel{2 {(a a^{\top})}^\top} + 2 \widehat{a}^\top}{1 + a^{\top} a}
    =
    \frac{4 \widehat{a}}{1 + a^{\top} a}
    \\
    \operatorname{tr}(R)
     & =
    \operatorname{tr}\left( \frac{(1 - a^{\top} a) I + 2 a a^{\top} + 2 \widehat{a}}{1 + a^{\top} a} \right)
    =
    \frac{1 - a^{\top} a}{1 + a^{\top} a} \operatorname{tr}(I) + \frac{2 \operatorname{tr}(a a^{\top})}{1 + a^{\top} a} + \cancel{\frac{2 \operatorname{tr}(\widehat{a})}{1 + a^{\top} a}}
    \\ & =
    \frac{3 (1 - a^{\top} a)}{1 + a^{\top} a} + \frac{2 \operatorname{tr}(a a^{\top})}{1 + a^{\top} a}
    =
    \frac{3 (1 - a^{\top} a) + 2 a^{\top} a}{1 + a^{\top} a}
    =
    \frac{3 - a^{\top} a}{1 + a^{\top} a}
    \\
    \implies
    1 + \operatorname{tr}(R)
     & =
    1 + \frac{3 - a^{\top} a}{1 + a^{\top} a}
    =
    \frac{1 + \cancel{a^{\top} a} + 3 - \cancel{a^{\top} a}}{1 + a^{\top} a}
    =
    \frac{4}{1 + a^{\top} a}
    \\
    \implies
    R - R^\top
     & =
    \frac{4 \widehat{a}}{1 + a^{\top} a}
    =
    \widehat{a} (1 + \operatorname{tr}(R))
    \implies
    \boxed{
        \widehat{a}
        =
        \frac{R - R^\top}{1 + \operatorname{tr}(R)}
    }
\end{align*}
as required.

Now, we want to see what the vectors ``-a'' and ``a = 0'' represent.
We can see that \( -a = \omega \tan(\theta / 2) \) and thereby, this represents the axis of the rotation, but scaled by \( \tan(\theta / 2) \).
For \( a = 0 \), we can see that \( R = (I - \widehat{0}) {(I + \widehat{0})}^{-1} = I \), which is the identity matrix, and thereby, this represents no rotation, or in other words, the identity rotation.
\clearpage
\subsubsection*{(b) Case \( k=2 \)}

Given \( R \in SO(3) \), we want to show that \( R \) can be expressed as
\begin{equation}\label{eq:cayley-k2}
    R
    =
    {(I - \widehat{a})}^{2} {(I + \widehat{a})}^{-2}
    =
    I - 4 \frac{(1 - a^{\top} a)}{{(1 + a^{\top} a)}^{2}} \widehat{a} + \frac{8}{{(1 + a^{\top} a)}^{2}} \widehat{a}^{2}
\end{equation}
for some \( a \in \mathbb{R}^{3} \).
The first half of the equation in~\eqref{eq:cayley-k2} is valid from~\eqref{eq:cayley-k} with \( k = 2 \). \\
Now, we will show the second half of the equation in~\eqref{eq:cayley-k2} is valid.
\begin{align*}
    \text{Let }
    E_0
     & =
    \left[
        I
        - 4 \frac{(1 - a^{\top} a)}{{(1 + a^{\top} a)}^{2}} \widehat{a}
        + \frac{8}{{(1 + a^{\top} a)}^{2}} \widehat{a}^{2}
        \right]
    \text{and }
    E_0
    {(I + \widehat{a})}^{2}
    =
    E_1
    , \text{ say}
    \\
    \implies
    E_1
     & =
    \left[
        I
        - 4 \frac{(1 - a^{\top} a)}{{(1 + a^{\top} a)}^{2}} \widehat{a}
        + \frac{8}{{(1 + a^{\top} a)}^{2}} \widehat{a}^{2}
        \right]
    (I + 2 \widehat{a} + \widehat{a}^{2})
    \\ & =
    I
    + \widehat{a} \left( 2 - 4 \frac{(1 - a^{\top} a)}{{(1 + a^{\top} a)}^{2}} \right)
    + \widehat{a}^{2} \left( 1 - 8 \frac{(1 - a^{\top} a)}{{(1 + a^{\top} a)}^{2}} + \frac{8}{{(1 + a^{\top} a)}^{2}} \right)
    \\ & \qquad
    + \widehat{a}^{3} \left( -4 \frac{(1 - a^{\top} a)}{{(1 + a^{\top} a)}^{2}} + \frac{16}{{(1 + a^{\top} a)}^{2}} \right)
    + \widehat{a}^{4} \left( \frac{8}{{(1 + a^{\top} a)}^{2}} \right)
\end{align*}
Now, using the relation \( \widehat{a}^{3} = - \| a \|^{2} \widehat{a} = - (a^{\top} a) \widehat{a} \), which were proved in previous assignment, thereby \( \widehat{a}^{4} = - (a^{\top} a) \widehat{a}^{2} \) and we get
\begin{align*}
    \implies
    E_1
     & =
    I
    + \widehat{a} \left( 2 - 4 \frac{(1 - a^{\top} a)}{{(1 + a^{\top} a)}^{2}} + 4 (a^{\top} a) \frac{(1 - a^{\top} a)}{{(1 + a^{\top} a)}^{2}} - \frac{16 (a^{\top} a)}{{(1 + a^{\top} a)}^{2}} \right)
    \\ & \qquad
    + \widehat{a}^{2} \left( 1 - 8 \frac{(1 - a^{\top} a)}{{(1 + a^{\top} a)}^{2}} + \frac{8}{{(1 + a^{\top} a)}^{2}} - \frac{8(a^{\top} a)}{{(1 + a^{\top} a)}^{2}} \right)
    \\
    \implies
    E_1
     & =
    I
    + E_2 \widehat{a}
    + E_3 \widehat{a}^{2},
    \text{ say}
    \\
    \implies
    E_2
     & =
    2 - 4 \frac{1 - a^{\top} a}{{(1 + a^{\top} a)}^{2}} + 4 (a^{\top} a) \frac{(1 - a^{\top} a)}{{(1 + a^{\top} a)}^{2}} - \frac{16 (a^{\top} a)}{{(1 + a^{\top} a)}^{2}}
    \\ & =
    \frac{
        2 {(1 + a^{\top} a)}^{2} - 4 (1 - a^{\top} a) + 4 (a^{\top} a) (1 - a^{\top} a) - 16 (a^{\top} a)
    }{{(1 + a^{\top} a)}^{2}}
    \\ & =
    \frac{
        2 + 4 (a^{\top} a) + 2 {(a^{\top} a)}^{2} - 4 + 4 (a^{\top} a) + 4 (a^{\top} a) - 4 {(a^{\top} a)}^{2} - 16 (a^{\top} a)
    }{{(1 + a^{\top} a)}^{2}}
    \\ & =
    \frac{
        -2 - 4 (a^{\top} a) - 4 {(a^{\top} a)}^{2}
    }{{(1 + a^{\top} a)}^{2}}
    =
    -2 \frac{\cancel{{(1 + a^{\top} a)}^{2}}}{\cancel{{(1 + a^{\top} a)}^{2}}}
    =
    -2
    \\
    \implies
    E_3
     & =
    1 - 8 \frac{(1 - a^{\top} a)}{{(1 + a^{\top} a)}^{2}} + \frac{8}{{(1 + a^{\top} a)}^{2}} - \frac{8(a^{\top} a)}{{(1 + a^{\top} a)}^{2}}
    \\ & =
    1 +
    \frac{
        - 8 (1 - a^{\top} a) + 8 - 8 (a^{\top} a)
    }{{(1 + a^{\top} a)}^{2}}
    =
    1 +
    \frac{
        - \cancel{8} + \cancel{8 (a^{\top} a)} + \cancel{8} - \cancel{8 (a^{\top} a)}
    }{{(1 + a^{\top} a)}^{2}}
    =
    1
\end{align*}
\begin{equation*}
    \implies
    E_1
    =
    I - 2 \widehat{a} + \widehat{a}^{2}
    =
    {(I - \widehat{a})}^{2}
    =
    E_0
    {(I + \widehat{a})}^{2}
    \implies
    E_0
    =
    {(I - \widehat{a})}^{2} {(I + \widehat{a})}^{-2}
    =
    R
\end{equation*}
Thereby, we have shown that the second half of the equation in~\eqref{eq:cayley-k2} is valid.
\clearpage
\subsubsection*{(c) Singularity at \( \pi \)}

For \( k \geq 2 \), the singularity at \( \pi \) that exists for the standard Cayley parameters \( (k=1) \) is removed.
The singularity at \( \pi \) is due to the fact that for the standard Cayley parameters with \( k = 1 \), we have that \( {(I + \widehat{a})}^{-1} \) is singular when \( \theta = \pi \), i.e., \( a = -w \tan (\theta/2) \) blows to \( \pm \infty \) near \( \theta = \pi \).
At higher \( k \), say \( k = 2 \), we have \( a = -w \tan (\theta/4) \implies a = -w \tan (\pi/4) = -w \) at \( \theta = \pi \).
The singularity at \( \theta = \pi \) is removed in this case, but we have a singularity at \( \theta = 2\pi \) instead.
If we consider the principal values for \( \theta \in [0, 2\pi) \), the singularity at \( \theta = \pi \) is effectively removed for higher \( k \).

\subsubsection*{Advantages of higher \( k \)}

The advantages of using higher \( k \) are as follows:
\begin{itemize}[noitemsep]
    \item The singularity at \( \pi \) is removed.
    \item The Cayley parameters are more numerically stable. \\
          This is in part due to the behavior of the tangent function.
    \item The Cayley parameters are more robust to numerical errors.
\end{itemize}

\subsubsection*{Disadvantages of higher \( k \)}

The disadvantages of using higher \( k \) are as follows:
\begin{itemize}[noitemsep]
    \item Increased computational complexity. \\
          The Cayley parameters are harder to compute as \( k \) increases.
\end{itemize}

\subsubsection*{(d) Composition of rotations}

Let \( a_{1} \) and \( a_{2} \) denote the Cayley parameters for two rotations \( R_{1} \) and \( R_{2} \) respectively. \\
We need to show that for \( R_{3}=R_{1} R_{2} \), the corresponding Cayley parameter \( a_{3} \) is given by
\[
    a_{3}=\frac{a_{1}+a_{2}+\left(a_{1} \times a_{2}\right)}{1-a_{1}^{\top} a_{2}}
\]
With \( R_1 = (I - \widehat{a_1}) {(I + \widehat{a_1})}^{-1} \), \( R_2 = (I - \widehat{a_2}) {(I + \widehat{a_2})}^{-1} \), we have
\begin{align*}
    \widehat{a_1}
     & =
    \frac{R_1 - R_1^\top}{1+\operatorname{trace}(R_1)}
    , \quad
    \widehat{a_2}
    =
    \frac{R_2 - R_2^\top}{1+\operatorname{trace}(R_2)}
\end{align*}

\subsection*{Extra}

\subsubsection*{When \( \operatorname{tr}(R) = -1 \) or \( a_1^\top a_2 = 1 \equiv \operatorname{tr}(R_1 R_2) = -1 \)}

Defining \( s=\frac{a}{\sqrt{1+a^{T} a}} \), we can see that
\begin{align*}
    \implies
    a
     & =
    s \sqrt{1 + {\| a \|}^2}
    \implies
    a^{T} a
    =
    s^{T} s \left( 1 + {\| a \|}^2 \right)
    \implies
    {\| a \|}^2
    =
    {\| s \|}^2 \left( 1 + {\| a \|}^2 \right)
    \\
    \implies
    {\| s \|}^2
     & =
    \frac{{\| a \|}^2}{1 + {\| a \|}^2}
    \implies
    \frac{1}{{\| s \|}^2}
    =
    1 + \frac{1}{{\| a \|}^2}
    \\
    \implies
    \frac{1}{{\| a \|}^2}
     & =
    \frac{1}{{\| s \|}^2} - 1
    \implies
    {\| a \|}^2
    =
    \frac{{\| s \|}^2}{1 - {\| s \|}^2}
\end{align*}
\begin{align*}
    \implies
    R
     & =
    \frac{\left( 1 - {\| a \|}^2 \right) I + 2 a a^{T} + 2 \widehat{a}}{1 + {\| a \|}^2}
    \\
    \implies
    R
     & =
    \frac{
    \left( 1 - {\| s \|}^2 \left( 1 + {\| a \|}^2 \right) \right) I
    + 2 \left( s \sqrt{1 + {\| a \|}^2} \right) {\left( s \sqrt{1 + {\| a \|}^2} \right)}^{\top}
    + 2 \sqrt{1 + {\| a \|}^2} \widehat{s}
    }
    {1 + {\| a \|}^2}
    \\ & =
    \frac{
    I
    + 2 \sqrt{1 + {\| a \|}^2} \widehat{s}
    }
    {1 + {\| a \|}^2}
    - {\| s \|}^2 I
    + 2 s s^{\top}
\end{align*}

We can then see that
\begin{equation*}
    \implies
    R
    =
    \left(1 - 2\|s\|^2\right) I + 2 ss^{\top} + 2\widehat{s}(1 - \|s\|^2)
\end{equation*}
which does not suffer from the singularity problem.

