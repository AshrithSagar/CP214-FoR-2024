\section*{Problem 1}
\setcounter{section}{1}
\setcounter{equation}{0}

\textbf{Cayley-Rodrigues Relationships}:
Recall the Cayley parametrization of \( S O(3) \) from earlier assignment.
We can generalize it to higher orders as follows:
\[
    R=(I-\widehat{a})^{k}(I+\widehat{a})^{-k} \in S O(3)
\]
Here, \( a \in \mathbb{R}^{3} \) are the Cayley parameters.
Now answer the following,
\begin{enumerate}[label= (\alph*)]
    \item (Case \( k=1 \))
          Show that given \( R=e^{\widehat{\omega} \theta} \) for some unit vector \( \omega \) along the axis of rotation and angle \( \theta \), then \( a=\omega \tan (\theta / 2) \).
          Furthermore, also prove below relationships:
          \[
              R=\frac{\left(1-a^{T} a\right) I+2 a a^{T}+2 \widehat{a}}{1+a^{T} a}, \quad \widehat{a}=\frac{R-R^{T}}{1+\operatorname{trace}(R)} .
          \]
          Using these, can we know what the vectors ``\( -a \)'' and ``\( a=0 \)'' represent?
          An attractive feature of the Cayley parameters is the simple form for the composition of two rotation matrices.
          If \( a_{1} \) and \( a_{2} \) denote the Cayley parameters for two rotations \( R_{1} \) and \( R_{2} \) respectively, then show that for \( R_{3}=R_{1} R_{2} \) we would have the corresponding \( a_{3} \) given by
          \[
              a_{3}=\frac{a_{1}+a_{2}+\left(a_{1} \times a_{2}\right)}{1-a_{1}^{T} a_{2}}
          \]
          Lastly, show that we can even find body frame and spatial frame angular velocities in a simple form like
          \[
              \omega_{s}=\frac{2}{1+\|a\|^{2}}(a \times \dot{a}+\dot{a}), \quad \omega_{b}=\frac{2}{1+\|a\|^{2}}(-a \times \dot{a}+\dot{a})
          \]

          (What would happen in the case when trace \( (R)=-1 \) or \( a_{1}^{T} a_{2}=1 \equiv \operatorname{trace}\left(R_{1} R_{2}\right)=-1 \)?
          Does everything fail, or can we still work around it?
          Maybe try deriving everything in terms of a variable \( s=\frac{a}{\sqrt{1+a^{T} a}} \) and see where it leads to \( \ldots \))

    \item (Case \( k=2 \)) Now show that the rotation \( R \) corresponding to \( a \) can be computed from the formula
          \[
              R=I-4 \frac{1-a^{T} a}{\left(1+a^{T} a\right)^{2}} \widehat{a}+\frac{8}{\left(1+a^{T} a\right)^{2}} \widehat{a}^{2}
          \]
          Conversely, prove that given a rotation matrix \( R \), there exists a vector \( a \) which satisfies the above and can be obtained as \( a=-\omega \tan (\theta / 4) \), where \( \omega \) is the unit vector along the axis of rotation for \( R \), and \( \theta \) is the corresponding rotation angle.
          Is this solution unique or not?

          Finally, show that the angular velocity in the body frame obeys the following relation:
          \[
              \dot{a}=\frac{1}{4}\left(\left(1-a^{T} a\right) I+2 \widehat{a}+2 a a^{T}\right) \omega_{b} .
          \]

    \item For \( k \geq 2 \), explain what happens to the singularity at \( \pi \) that exists for the standard Cayley parameters \( (k=1) \).
          Also, discuss the relative advantages and disadvantages of choosing different \( k \).

    \item Compare the number of arithmetic operations needed for multiplying two rotation matrices or two Cayley representations.
          Which requires the fewest arithmetic operations?
\end{enumerate}

\subsection*{Solution}

\subsubsection*{(a) Case \( k=1 \)}

\begin{equation*}
    R = (I - \widehat{a}) {(I + \widehat{a})}^{-1} \in SO(3)
\end{equation*}
