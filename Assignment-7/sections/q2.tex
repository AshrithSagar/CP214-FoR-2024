\section*{Problem 2}
\setcounter{section}{2}
\setcounter{equation}{0}

\textbf{3R spatial open chain:}
Consider a manipulator shown in its home position (all joint variables set equal to zero).
Choose the fixed frame \( \{0\} \) and end-effector frame \( \{3\} \) as indicated in the figure.
Given all this information, express the end-effector frame configuration when the robot is in its zero position (i.e., \( g_{e e}(0) \)) and the twist axes \( \widehat{\xi}_{i} \) of joint \( i=\{1,2,3\} \).

\begin{figure}[h]
    \centering
    \includegraphics[width=0.5\textwidth]{figures/images/q2.jpg}
\end{figure}

\subsection*{Solution}

The end-effector frame configuration can be computed using the product of exponentials formula as
\begin{equation}
    g_{e e}(\theta)
    =
    e^{\widehat{\xi}_{1} \theta_{1}}
    e^{\widehat{\xi}_{2} \theta_{2}}
    e^{\widehat{\xi}_{3} \theta_{3}}
    g_{e e}(0)
\end{equation}
where \( \theta = \begin{bmatrix} \theta_1 & \theta_2 & \theta_3 \end{bmatrix}^T \) are the joint variables.
The twist coordinates for each of the three joints can be found as
\begin{align*}
    \implies
    \xi_{1}
     & =
    \begin{bmatrix}
        v_1 \\
        \omega_1
    \end{bmatrix},
    \quad
    \omega_1
    =
    \begin{bmatrix}
        0 \\
        0 \\
        1
    \end{bmatrix},
    \quad
    v_1
    =
    \begin{bmatrix}
        0 \\
        0 \\
        0
    \end{bmatrix}
    \\
    \implies
    \xi_{2}
     & =
    \begin{bmatrix}
        v_2 \\
        \omega_2
    \end{bmatrix},
    \quad
    \omega_2
    =
    \begin{bmatrix}
        0  \\
        -1 \\
        0
    \end{bmatrix},
    \quad
    v_2
    =
    - \omega_2 \times
    \begin{bmatrix}
        L_1 \\
        0   \\
        0
    \end{bmatrix}
    =
    \begin{bmatrix}
        0 \\
        0 \\
        -L_1
    \end{bmatrix}
    \\
    \implies
    \xi_{3}
     & =
    \begin{bmatrix}
        v_3 \\
        \omega_3
    \end{bmatrix},
    \quad
    \omega_3
    =
    \begin{bmatrix}
        0 \\
        1 \\
        0
    \end{bmatrix},
    \quad
    v_3
    =
    - \omega_3 \times
    \begin{bmatrix}
        L_2 \\
        0   \\
        0
    \end{bmatrix}
    =
    \begin{bmatrix}
        0 \\
        0 \\
        L_2
    \end{bmatrix}
\end{align*}
The corresponding twist axes are
\begin{equation*}
    \implies
    \xi_{1}
    =
    \begin{bmatrix}
        0 \\
        0 \\
        0 \\
        0 \\
        0 \\
        1
    \end{bmatrix}
    \implies
    \widehat{\xi}_{1}
    =
    \begin{bmatrix}
        \widehat{\omega}_{1} & v_1 \\
        0                    & 0
    \end{bmatrix}
    \implies
    \boxed{
        \widehat{\xi}_{1}
        =
        \begin{bmatrix}
            0 & -1 & 0 & 0 \\
            1 & 0  & 0 & 0 \\
            0 & 0  & 0 & 0 \\
            0 & 0  & 0 & 0
        \end{bmatrix}
    }
\end{equation*}

\begin{equation*}
    \implies
    \xi_{2}
    =
    \begin{bmatrix}
        0    \\
        0    \\
        -L_1 \\
        0    \\
        -1   \\
        0
    \end{bmatrix}
    \implies
    \widehat{\xi}_{2}
    =
    \begin{bmatrix}
        \widehat{\omega}_{2} & v_2 \\
        0                    & 0
    \end{bmatrix}
    \implies
    \boxed{
        \widehat{\xi}_{2}
        =
        \begin{bmatrix}
            0 & 0 & -1 & 0    \\
            0 & 0 & 0  & 0    \\
            1 & 0 & 0  & -L_1 \\
            0 & 0 & 0  & 0
        \end{bmatrix}
    }
\end{equation*}

\begin{equation*}
    \implies
    \xi_{3}
    =
    \begin{bmatrix}
        0   \\
        0   \\
        L_2 \\
        0   \\
        1   \\
        0
    \end{bmatrix}
    \implies
    \widehat{\xi}_{3}
    =
    \begin{bmatrix}
        \widehat{\omega}_{3} & v_3 \\
        0                    & 0
    \end{bmatrix}
    \implies
    \boxed{
        \widehat{\xi}_{3}
        =
        \begin{bmatrix}
            0  & 0 & 1 & 0   \\
            0  & 0 & 0 & 0   \\
            -1 & 0 & 0 & L_2 \\
            0  & 0 & 0 & 0
        \end{bmatrix}
    }
\end{equation*}
The exponential of the twist matrices can be evaluated to be
\begin{align*}
    \implies
    e^{\widehat{\xi}_{1} \theta_{1}}
     & =
    \begin{bmatrix}
        \cos \theta_{1} & -\sin \theta_{1} & 0 & 0 \\
        \sin \theta_{1} & \cos \theta_{1}  & 0 & 0 \\
        0               & 0                & 1 & 0 \\
        0               & 0                & 0 & 1
    \end{bmatrix}
    \\
    \implies
    e^{\widehat{\xi}_{2} \theta_{2}}
     & =
    \begin{bmatrix}
        \cos(L_1 \theta_2)  & \sin(L_1 \theta_2) & 0 & -\frac{1}{L_1} (1 - \cos(L_1 \theta_2)) \\
        -\sin(L_1 \theta_2) & \cos(L_1 \theta_2) & 0 & -\frac{1}{L_1} \sin(L_1 \theta_2)       \\
        0                   & 0                  & 1 & 0                                       \\
        0                   & 0                  & 0 & 1
    \end{bmatrix}
    \\
    \implies
    e^{\widehat{\xi}_{3} \theta_{3}}
     & =
    \begin{bmatrix}
        \cos(L_2 \theta_3) & -\sin(L_2 \theta_3) & 0 & -\frac{1}{L_2} (1 - \cos(L_2 \theta_3)) \\
        \sin(L_2 \theta_3) & \cos(L_2 \theta_3)  & 0 & \frac{1}{L_2} \sin(L_2 \theta_3)        \\
        0                  & 0                   & 1 & 0                                       \\
        0                  & 0                   & 0 & 1
    \end{bmatrix}
\end{align*}
In the zero configuration, we have
\begin{equation*}
    \boxed{
        g_{e e}(0)
        =
        \begin{bmatrix}
            0  & 0 & 1 & L_1  \\
            0  & 1 & 0 & 0    \\
            -1 & 0 & 0 & -L_2 \\
            0  & 0 & 0 & 1
        \end{bmatrix}
    }
\end{equation*}
which can be written by inspection, based on frame \( \{0\} \), since w.r.t.\@ frame \( \{0\} \), the end-effector frame is translated by \( L_1 \) along the \( x \)-axis and \( -L_2 \) along the \( z \)-axis, and there is a 90 degree rotation about the y-axis. The rotation matrix about the y-axis is given by
\begin{equation*}
    R_y(\theta)
    =
    \begin{bmatrix}
        \cos \theta  & 0 & \sin \theta \\
        0            & 1 & 0           \\
        -\sin \theta & 0 & \cos \theta
    \end{bmatrix}
    \implies
    R_y\left(\frac{\pi}{2}\right)
    =
    \begin{bmatrix}
        0  & 0 & 1 \\
        0  & 1 & 0 \\
        -1 & 0 & 0
    \end{bmatrix}
\end{equation*}
