\subsection*{Extra}

\subsubsection*{When \( \operatorname{tr}(R) = -1 \) or \( a_1^\top a_2 = 1 \equiv \operatorname{tr}(R_1 R_2) = -1 \)}

Defining \( s=\frac{a}{\sqrt{1+a^{T} a}} \), we can see that
\begin{align*}
    \implies
    a
     & =
    s \sqrt{1 + {\| a \|}^2}
    \implies
    a^{T} a
    =
    s^{T} s \left( 1 + {\| a \|}^2 \right)
    \implies
    {\| a \|}^2
    =
    {\| s \|}^2 \left( 1 + {\| a \|}^2 \right)
    \\
    \implies
    {\| s \|}^2
     & =
    \frac{{\| a \|}^2}{1 + {\| a \|}^2}
    \implies
    \frac{1}{{\| s \|}^2}
    =
    1 + \frac{1}{{\| a \|}^2}
    \\
    \implies
    \frac{1}{{\| a \|}^2}
     & =
    \frac{1}{{\| s \|}^2} - 1
    \implies
    {\| a \|}^2
    =
    \frac{{\| s \|}^2}{1 - {\| s \|}^2}
\end{align*}
\begin{align*}
    \implies
    R
     & =
    \frac{\left( 1 - {\| a \|}^2 \right) I + 2 a a^{T} + 2 \widehat{a}}{1 + {\| a \|}^2}
    \\
    \implies
    R
     & =
    \frac{
    \left( 1 - {\| s \|}^2 \left( 1 + {\| a \|}^2 \right) \right) I
    + 2 \left( s \sqrt{1 + {\| a \|}^2} \right) {\left( s \sqrt{1 + {\| a \|}^2} \right)}^{\top}
    + 2 \sqrt{1 + {\| a \|}^2} \widehat{s}
    }
    {1 + {\| a \|}^2}
    \\ & =
    \frac{
    I
    + 2 \sqrt{1 + {\| a \|}^2} \widehat{s}
    }
    {1 + {\| a \|}^2}
    - {\| s \|}^2 I
    + 2 s s^{\top}
\end{align*}

We can then see that
\begin{equation*}
    \implies
    R
    =
    \left(1 - 2\|s\|^2\right) I + 2 ss^{\top} + 2\widehat{s}(1 - \|s\|^2)
\end{equation*}
which does not suffer from the singularity problem.
