\subsubsection*{(d) Composition of rotations}

Let \( a_{1} \) and \( a_{2} \) denote the Cayley parameters for two rotations \( R_{1} \) and \( R_{2} \) respectively. \\
We need to show that for \( R_{3}=R_{1} R_{2} \), the corresponding Cayley parameter \( a_{3} \) is given by
\begin{equation}\label{eq:q1d}
    a_{3}=\frac{a_{1}+a_{2}+\left(a_{1} \times a_{2}\right)}{1-a_{1}^{\top} a_{2}}
\end{equation}
We have \( R_1 = (I - \widehat{a_1}) {(I + \widehat{a_1})}^{-1} \), \( R_2 = (I - \widehat{a_2}) {(I + \widehat{a_2})}^{-1} \).

Now, with the form in equation~\eqref{eq:q1d}, the corresponding rotation for it is
\begin{align*}
    R_{3}
     & =
    \frac{(1 - a_3^{\top} a_3) I + 2 a_3 a_3^{\top} + 2 \widehat{a_3}}{1 + a_3^{\top} a_3}
    \\
    \implies
    a_3^{\top} a_3
     & =
    \frac{a_{1}^\top+a_{2}^\top+\left(a_{1} \times a_{2}\right)^\top}{1-a_{1}^{\top} a_{2}}
    \frac{a_{1}+a_{2}+\left(a_{1} \times a_{2}\right)}{1-a_{1}^{\top} a_{2}}
    \\ & =
    \frac{
    \| a_1 + a_2 \|^2 + \| a_1 \times a_2 \|^2
    }{{(1-a_{1}^{\top} a_{2})}^{2}}
\end{align*}

We can then see that
\begin{align*}
    \implies
    R_{3}
     & =
    \frac{(1 - a_3^{\top} a_3) I + 2 a_3 a_3^{\top} + 2 \widehat{a_3}}{1 + a_3^{\top} a_3}
    \\ & =
    \frac{(1 - a_1^{\top} a_1) I + 2 a_1 a_1^{\top} + 2 \widehat{a_1}}{1 + a_1^{\top} a_1}
    \frac{(1 - a_2^{\top} a_2) I + 2 a_2 a_2^{\top} + 2 \widehat{a_2}}{1 + a_2^{\top} a_2}
    =
    R_1 R_2
\end{align*}

\subsubsection*{Computational complexity of composition of two rotations}

For each element in the composed rotation matrix, we obtain it by 3 multiplications and 2 additions \( \implies \) For the entire rotation matrix, we need \underline{27 multiplications and 18 additions}, since there are 9 elements in the rotation matrix.

Using Cayley representation, for \( a_1^\top a_2 \), we need 3 multiplications and 2 additions, for \( a_1 \times a_2 \), we need 6 multiplications and 3 additions, and for \( a_1 + a_2 \), we need 3 additions, and now for \( (a_1 + a_2) + (a_1 \times a_2) \), we need 3 more additions.
For incorporating \( 1 - a_1^\top a_2 \) in the above sum, we need one addition and one multiplication.
Thereby, in total, we would need
(3 + 6 + 1) multiplications and (2+3+3+3+1) additions i.e., \underline{10 multiplications and 12 additions}.
Here, we considered division taking up one multiplication operation, and similarly subtraction taking up one addition operation.
