\subsubsection*{(b) Case \( k=2 \)}

With \( R \in SO(3) \), we want to show that \( R \) can be computed from
\begin{equation}\label{eq:cayley-k2}
    R
    =
    {(I - \widehat{a})}^{2} {(I + \widehat{a})}^{-2}
    =
    I - 4 \frac{(1 - a^{\top} a)}{{(1 + a^{\top} a)}^{2}} \widehat{a} + \frac{8}{{(1 + a^{\top} a)}^{2}} \widehat{a}^{2}
\end{equation}
for some \( a \in \mathbb{R}^{3} \).
The first half of the equation in~\eqref{eq:cayley-k2} is valid from~\eqref{eq:cayley-k} with \( k = 2 \). \\
Now, we will show the second half of the equation in~\eqref{eq:cayley-k2} is valid.
\begin{align*}
    \text{Let }
    E_0
     & =
    \left[
        I
        - 4 \frac{(1 - a^{\top} a)}{{(1 + a^{\top} a)}^{2}} \widehat{a}
        + \frac{8}{{(1 + a^{\top} a)}^{2}} \widehat{a}^{2}
        \right]
    \text{and }
    E_0
    {(I + \widehat{a})}^{2}
    =
    E_1
    , \text{ say}
    \\
    \implies
    E_1
     & =
    \left[
        I
        - 4 \frac{(1 - a^{\top} a)}{{(1 + a^{\top} a)}^{2}} \widehat{a}
        + \frac{8}{{(1 + a^{\top} a)}^{2}} \widehat{a}^{2}
        \right]
    (I + 2 \widehat{a} + \widehat{a}^{2})
    \\ & =
    I
    + \widehat{a} \left( 2 - 4 \frac{(1 - a^{\top} a)}{{(1 + a^{\top} a)}^{2}} \right)
    + \widehat{a}^{2} \left( 1 - 8 \frac{(1 - a^{\top} a)}{{(1 + a^{\top} a)}^{2}} + \frac{8}{{(1 + a^{\top} a)}^{2}} \right)
    \\ & \qquad
    + \widehat{a}^{3} \left( -4 \frac{(1 - a^{\top} a)}{{(1 + a^{\top} a)}^{2}} + \frac{16}{{(1 + a^{\top} a)}^{2}} \right)
    + \widehat{a}^{4} \left( \frac{8}{{(1 + a^{\top} a)}^{2}} \right)
\end{align*}
Now, using the relation \( \widehat{a}^{3} = - \| a \|^{2} \widehat{a} = - (a^{\top} a) \widehat{a} \), which were proved in previous assignment, thereby \( \widehat{a}^{4} = - (a^{\top} a) \widehat{a}^{2} \) and we get
\begin{align*}
    \implies
    E_1
     & =
    I
    + \widehat{a} \left( 2 - 4 \frac{(1 - a^{\top} a)}{{(1 + a^{\top} a)}^{2}} + 4 (a^{\top} a) \frac{(1 - a^{\top} a)}{{(1 + a^{\top} a)}^{2}} - \frac{16 (a^{\top} a)}{{(1 + a^{\top} a)}^{2}} \right)
    \\ & \qquad
    + \widehat{a}^{2} \left( 1 - 8 \frac{(1 - a^{\top} a)}{{(1 + a^{\top} a)}^{2}} + \frac{8}{{(1 + a^{\top} a)}^{2}} - \frac{8(a^{\top} a)}{{(1 + a^{\top} a)}^{2}} \right)
    \\
    \implies
    E_1
     & =
    I
    + E_2 \widehat{a}
    + E_3 \widehat{a}^{2},
    \text{ say}
    \\
    \implies
    E_2
     & =
    2 - 4 \frac{1 - a^{\top} a}{{(1 + a^{\top} a)}^{2}} + 4 (a^{\top} a) \frac{(1 - a^{\top} a)}{{(1 + a^{\top} a)}^{2}} - \frac{16 (a^{\top} a)}{{(1 + a^{\top} a)}^{2}}
    \\ & =
    \frac{
        2 {(1 + a^{\top} a)}^{2} - 4 (1 - a^{\top} a) + 4 (a^{\top} a) (1 - a^{\top} a) - 16 (a^{\top} a)
    }{{(1 + a^{\top} a)}^{2}}
    \\ & =
    \frac{
        2 + 4 (a^{\top} a) + 2 {(a^{\top} a)}^{2} - 4 + 4 (a^{\top} a) + 4 (a^{\top} a) - 4 {(a^{\top} a)}^{2} - 16 (a^{\top} a)
    }{{(1 + a^{\top} a)}^{2}}
    \\ & =
    \frac{
        -2 - 4 (a^{\top} a) - 4 {(a^{\top} a)}^{2}
    }{{(1 + a^{\top} a)}^{2}}
    =
    -2 \frac{\cancel{{(1 + a^{\top} a)}^{2}}}{\cancel{{(1 + a^{\top} a)}^{2}}}
    =
    -2
    \\
    \implies
    E_3
     & =
    1 - 8 \frac{(1 - a^{\top} a)}{{(1 + a^{\top} a)}^{2}} + \frac{8}{{(1 + a^{\top} a)}^{2}} - \frac{8(a^{\top} a)}{{(1 + a^{\top} a)}^{2}}
    \\ & =
    1 +
    \frac{
        - 8 (1 - a^{\top} a) + 8 - 8 (a^{\top} a)
    }{{(1 + a^{\top} a)}^{2}}
    =
    1 +
    \frac{
        - \cancel{8} + \cancel{8 (a^{\top} a)} + \cancel{8} - \cancel{8 (a^{\top} a)}
    }{{(1 + a^{\top} a)}^{2}}
    =
    1
\end{align*}
\begin{equation*}
    \implies
    E_1
    =
    I - 2 \widehat{a} + \widehat{a}^{2}
    =
    {(I - \widehat{a})}^{2}
    =
    E_0
    {(I + \widehat{a})}^{2}
    \implies
    E_0
    =
    {(I - \widehat{a})}^{2} {(I + \widehat{a})}^{-2}
    =
    R
\end{equation*}
Thereby, we have shown that the second half of the equation in~\eqref{eq:cayley-k2} is valid.

\clearpage
Now, we want to show that the Cayley parameter \( a \) exists and satisfies \( a = -\omega \tan(\theta / 2) \) for some \( \omega \in \mathbb{R}^{3}, \| \omega \| = 1 \) and \( \theta \in \mathbb{R} \).
From the equation~\eqref{eq:cayley-k2}, suppose we had that \( a = -\omega \tan(\theta / 4) \), then we have \( \widehat{a} = -\widehat{\omega} \tan(\theta / 4) \), \( a^\top a = \tan^{2}(\theta / 4) \| \omega \|^{2} = \tan^{2}(\theta / 4) \), and we can see that
\begin{align*}
    R
     & =
    I - 4 \frac{(1 - a^{\top} a)}{{(1 + a^{\top} a)}^{2}} \widehat{a} + \frac{8}{{(1 + a^{\top} a)}^{2}} \widehat{a}^{2}
    \\ & =
    I - 4 \frac{1 - \tan^{2}(\theta / 4)}{{(1 + \tan^{2}(\theta / 4))}^{2}}{(-\widehat{\omega} \tan(\theta / 4))} + \frac{8}{{(1 + \tan^{2}(\theta / 4))}^{2}}{(-\widehat{\omega} \tan(\theta / 4))}^{2}
    \\ & =
    I + E_1 \widehat{\omega} + E_2 \widehat{\omega}^{2}, \text{ say}
    \\
    \implies
    E_1
     & =
    4 \frac{1 - \tan^{2}(\theta / 4)}{{(1 + \tan^{2}(\theta / 4))}^{2}} \tan(\theta / 4)
    =
    4
    \frac{ \cancel{\sec^{2}{(\theta / 4)}} \left[ \cos^{2}{(\theta / 4)} - \sin^{2}{(\theta / 4)} \right] }{ \cancel{\sec^{2}{(\theta / 4)}} \sec^{2}{(\theta / 4)} }
    \tan(\theta / 4)
    \\ & =
    4 \cos{(\theta / 2)} \frac{\sin(\theta / 4)}{\cos(\theta / 4)} \cos^{2}{(\theta / 4)}
    =
    4 \cos{(\theta / 2)} \sin(\theta / 4) \cos{(\theta / 4)}
    \\ & =
    2 \cos{(\theta / 2)} \sin(\theta / 2)
    =
    \sin(\theta)
    \\
    \implies
    E_2
     & =
    \frac{8}{{(1 + \tan^{2}(\theta / 4))}^{2}} \tan^{2}(\theta / 4)
    =
    \frac{8}{{\sec^{4}(\theta / 4)}} \tan^{2}(\theta / 4)
    =
    8 \cos^{4}(\theta / 4) \frac{\sin^{2}(\theta / 4)}{\cos^{2}(\theta / 4)}
    \\ & =
    8 \cos^{2}(\theta / 4) \sin^{2}(\theta / 4)
    =
    2 {\Big[ 2 \cos(\theta / 4) \sin(\theta / 4) \Big]}^{2}
    =
    2 \sin^{2}(\theta / 2)
    =
    1 - \cos(\theta)
    \\
    \implies
    R
     & =
    I + \sin(\theta) \widehat{\omega} + (1 - \cos(\theta)) \widehat{\omega}^{2}
\end{align*}
which is the Rodrigues' formula for \( R \) with \( \theta \) and \( \omega \) as the parameters.

\( \implies \boxed{a = -\omega \tan(\theta / 4)} \) holds.

To see whether this solution is unique, we can see that \( \tan(\theta) \) has a period of \( \pi \), which implies that \( \tan(\theta / 4) \) has a period of \( 4\pi \).
Thus, the solution is \underline{not unique}, and we can have multiple solutions for \( a \) and \( \theta \) for a given \( R \in SO(3) \).
