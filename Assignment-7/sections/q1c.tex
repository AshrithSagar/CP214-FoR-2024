\subsubsection*{(c) Singularity at \( \pi \)}

For \( k \geq 2 \), the singularity at \( \pi \) that exists for the standard Cayley parameters \( (k=1) \) is removed.
The singularity at \( \pi \) is due to the fact that for the standard Cayley parameters with \( k = 1 \), we have that \( {(I + \widehat{a})}^{-1} \) is singular when \( \theta = \pi \), i.e., \( a = -w \tan (\theta/2) \) blows to \( \pm \infty \) near \( \theta = \pi \).
At higher \( k \), say \( k = 2 \), we have \( a = -w \tan (\theta/4) \implies a = -w \tan (\pi/4) = -w \) at \( \theta = \pi \).
The singularity at \( \theta = \pi \) is removed in this case, but we have a singularity at \( \theta = 2\pi \) instead.
If we consider the principal values for \( \theta \in [0, 2\pi) \), the singularity at \( \theta = \pi \) is effectively removed for higher \( k \).

\subsubsection*{Advantages of higher \( k \)}

The advantages of using higher \( k \) are as follows:
\begin{itemize}[noitemsep]
    \item The singularity at \( \pi \) is removed.
    \item The Cayley parameters are more numerically stable. \\
          This is in part due to the behavior of the tangent function.
    \item The Cayley parameters are more robust to numerical errors.
\end{itemize}

\subsubsection*{Disadvantages of higher \( k \)}

The disadvantages of using higher \( k \) are as follows:
\begin{itemize}[noitemsep]
    \item Increased computational complexity. \\
          The Cayley parameters are harder to compute as \( k \) increases.
\end{itemize}
