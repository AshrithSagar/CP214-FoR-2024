\section*{Problem 3}

\textbf{Adjoint Transformation in Planar Motion:}
\begin{figure}[h]
    \centering
    \includegraphics[width=0.75\textwidth]{figures/images/q3.jpg}
    \caption{
        The twist corresponding to the instantaneous motion of the chassis of a three-wheeled vehicle can be visualized as an angular velocity w about the point r.
    }
\end{figure}

In this question, we will have a setup of planar motion as shown in the top view of a car, with a single steerable front wheel driving (kind of like a rickshaw).
The \( \hat{z}_{b} \)-axis of the body frame \( \{ \mathrm{b} \} \) is into the page and the \( \hat{z}_{\mathrm{s}} \)-axis of the fixed frame \( \{ \mathrm{s} \} \) is out of the page.
The angle of the front wheel of the car causes a pure angular motion with \( \mathrm{w}=2 \mathrm{rad} / \mathrm{s} \) about an axis out of the page at the point \( r \) in the plane.
For the sake of this problem we will write r as \( r_{s}=(2,-1,0) \) or \( r_{b}=(2,-1.4,0) \), w as \( \omega_{s}=(0,0,2) \) or \( \omega_{b}=(0,0,-2) \)

Now answer the following:
\begin{enumerate}[label= (\alph*)]
    \item Using the figure and simple geometry (basic relation of \( v=\omega \times r \)), compute \( v_{s}, v_{b} \in \mathbb{R}^{3} \) and thus obtain the twist coordinates \( \xi_{s}, \xi_{b} \in \mathbb{R}^{6} \).
          Verify these results using \( \xi_{s}=\operatorname{Ad}_{g_{s b}} \xi_{b} \), where \( \operatorname{Ad}_{g_{s b}}: \mathbb{R}^{6} \mapsto \mathbb{R}^{6} \) and \( g_{s b} \equiv\left(p_{s b}, R_{s b}\right) \in S E(3) \).
    \item Recall Ques. 3 from Homework 5, where we had a similar problem.
          However, in that case, the transformation was \( g^{\prime} \equiv\left(p^{\prime}, R^{\prime}\right) \in \) \( S E(2) \), and the twists were \( \widehat{\xi}^{\prime} \in s e(2) \) (go through that problem again for exact formulation).
          Now coming to this assignment's question, can we show here that there exists an adjoint transformation used to map twist coordinates in the body frame \( \xi_{b}^{\prime} \in \mathbb{R}^{3} \) into fixed frame \( \xi_{s}^{\prime} \in \mathbb{R}^{3} \).
          If not, why and what changes would be required in the problem formulation to ensure this exists?
    \item After solving (b), show that the adjoint transformation would be of the form,
          \[
              \operatorname{Ad}_{g_{s b}^{\prime}}=\left[\begin{array}{cc}
                      R_{s b}^{\prime} & {\left[\begin{array}{cc}
                                                                0  & 1 \\
                                                                -1 & 0
                                                            \end{array}\right] p_{s b}^{\prime}} \\
                      0                & 1
                  \end{array}\right], \text { where } R_{s b}^{\prime} \in S O(2), p_{s b}^{\prime} \in \mathbb{R}^{2}
          \]
          (Note: To make things distinguishable, we are using ' for all variables in (b) and (c))
\end{enumerate}

\clearpage
\subsection*{Solution}

\subsubsection*{(a) Computing \( v_{s}, v_{b} \) and \( \xi_{s}, \xi_{b} \)}

Given \( r_{s}=(2,-1,0) \), \( r_{b}=(2,-1.4,0) \), \( \omega_{s}=(0,0,2) \) and \( \omega_{b}=(0,0,-2) \), we can compute the linear velocity \( v_{s}, v_{b} \) as
\begin{align*}
    v_{s}
     & =
    \omega_{s} \times r_{s}
    =
    \begin{bmatrix}
        0 \\
        0 \\
        2
    \end{bmatrix}
    \times
    \begin{bmatrix}
        2  \\
        -1 \\
        0
    \end{bmatrix}
    =
    \begin{bmatrix}
        0 & -2 & 0 \\
        2 & 0  & 0 \\
        0 & 0  & 0
    \end{bmatrix}
    \begin{bmatrix}
        2  \\
        -1 \\
        0
    \end{bmatrix}
    \implies
    \boxed{
        v_{s}
        =
        \begin{bmatrix}
            2 \\
            4 \\
            0
        \end{bmatrix}
    }
    \\
    v_{b}
     & =
    \omega_{b} \times r_{b}
    =
    \begin{bmatrix}
        0 \\
        0 \\
        -2
    \end{bmatrix}
    \times
    \begin{bmatrix}
        2    \\
        -1.4 \\
        0
    \end{bmatrix}
    =
    \begin{bmatrix}
        0  & 2 & 0 \\
        -2 & 0 & 0 \\
        0  & 0 & 0
    \end{bmatrix}
    \begin{bmatrix}
        2    \\
        -1.4 \\
        0
    \end{bmatrix}
    \implies
    \boxed{
        v_{b}
        =
        \begin{bmatrix}
            -2.8 \\
            -4   \\
            0
        \end{bmatrix}
    }
\end{align*}
\begin{equation*}
    \implies
    \xi_{s}
    =
    \begin{bmatrix}
        v_{s} \\
        \omega_{s}
    \end{bmatrix}
    \implies
    \boxed{
        \xi_{s}
        =
        \begin{bmatrix}
            2 \\
            4 \\
            0 \\
            0 \\
            0 \\
            2
        \end{bmatrix}
    }
    \quad \text{and} \quad
    \xi_{b}
    =
    \begin{bmatrix}
        v_{b} \\
        \omega_{b}
    \end{bmatrix}
    \implies
    \boxed{
        \xi_{b}
        =
        \begin{bmatrix}
            -2.8 \\
            -4   \\
            0    \\
            0    \\
            0    \\
            -2
        \end{bmatrix}
    }
\end{equation*}
Now, given \( g_{s b} = (R_{s b}, p_{s b}) \), we have it in homogeneous coordinates as
\begin{align*}
    g_{s b}
     & =
    \begin{bmatrix}
        R_{s b} & p_{s b} \\
        0       & 1
    \end{bmatrix}_{4 \times 4}
\end{align*}
Now, from the figure, we can compute \( R_{s b} \) and \( p_{s b} \) as follows.
\( p_{s b} \) would denote the position of the origin of the body frame w.r.t the fixed frame, and we have
\begin{align*}
    p_{s b}
     & =
    \begin{bmatrix}
        2 + 2       \\
        -1 - (-1.4) \\
        0
    \end{bmatrix}
    =
    \begin{bmatrix}
        4   \\
        0.4 \\
        0
    \end{bmatrix}
\end{align*}
Now, for \( R_{s b} \), we can see that the fixed frame is rotated by \( 180^{\circ} \) about the \( \hat{y}_{b} \)-axis, w.r.t the body frame, and we have
\begin{align*}
    R_{s b}
     & =
    \begin{bmatrix}
        \cos(180^{\circ})  & 0 & \sin(180^{\circ}) \\
        0                  & 1 & 0                 \\
        -\sin(180^{\circ}) & 0 & \cos(180^{\circ})
    \end{bmatrix}
    =
    \begin{bmatrix}
        -1 & 0 & 0  \\
        0  & 1 & 0  \\
        0  & 0 & -1
    \end{bmatrix}
\end{align*}
Thus, we have the adjoint transformation \( \operatorname{Ad}_{g_{s b}} \) as
\begin{align*}
    \operatorname{Ad}_{g_{s b}}
     & =
    \begin{bmatrix}
        R_{s b} & - \widehat{p}_{s b} R_{s b} \\
        0       & R_{s b}
    \end{bmatrix}_{6 \times 6}
    \\
    \implies
    - \widehat{p}_{s b} R_{s b}
     & =
    - \begin{bmatrix}
          0    & 0 & 0.4 \\
          0    & 0 & -4  \\
          -0.4 & 4 & 0
      \end{bmatrix}
    \begin{bmatrix}
        -1 & 0 & 0  \\
        0  & 1 & 0  \\
        0  & 0 & -1
    \end{bmatrix}
    =
    \begin{bmatrix}
        0    & 0  & 0.4 \\
        0    & 0  & -4  \\
        -0.4 & -4 & 0
    \end{bmatrix}
\end{align*}
Thus, we have
\begin{align*}
    \implies
    \operatorname{Ad}_{g_{s b}} \xi_{b}
     & =
    \begin{bmatrix}
        R_{s b} & - \widehat{p}_{s b} R_{s b} \\
        0       & R_{s b}
    \end{bmatrix}
    \begin{bmatrix}
        v_b \\
        \omega_b
    \end{bmatrix}
    =
    \begin{bmatrix}
        R_{s b} v_b - \widehat{p}_{s b} R_{s b} \omega_b \\
        R_{s b} \omega_b
    \end{bmatrix}
    \\ \implies
    R_{s b} v_b + (- \widehat{p}_{s b} R_{s b}) \omega_b
     & =
    \begin{bmatrix}
        -1 & 0 & 0  \\
        0  & 1 & 0  \\
        0  & 0 & -1
    \end{bmatrix}
    \begin{bmatrix}
        -2.8 \\
        -4   \\
        0
    \end{bmatrix}
    +
    \begin{bmatrix}
        0    & 0  & 0.4 \\
        0    & 0  & -4  \\
        -0.4 & -4 & 0
    \end{bmatrix}
    \begin{bmatrix}
        0 \\
        0 \\
        -2
    \end{bmatrix}
    \\ & =
    \begin{bmatrix}
        2.8 \\
        -4  \\
        0
    \end{bmatrix}
    +
    \begin{bmatrix}
        -0.8 \\
        8    \\
        0
    \end{bmatrix}
    =
    \begin{bmatrix}
        2 \\
        4 \\
        0
    \end{bmatrix}
    =
    v_s
    \\
    \implies
    R_{s b} \omega_b
     & =
    \begin{bmatrix}
        -1 & 0 & 0  \\
        0  & 1 & 0  \\
        0  & 0 & -1
    \end{bmatrix}
    \begin{bmatrix}
        0 \\
        0 \\
        -2
    \end{bmatrix}
    =
    \begin{bmatrix}
        0 \\
        0 \\
        2
    \end{bmatrix}
    =
    \omega_s
\end{align*}
Thereby,
\begin{align*}
    \implies
    \operatorname{Ad}_{g_{s b}} \xi_{b}
    =
    \begin{bmatrix}
        R_{s b} v_b + (- \widehat{p}_{s b} R_{s b}) \omega_b \\
        R_{s b} \omega_b
    \end{bmatrix}
    =
    \begin{bmatrix}
        v_s \\
        \omega_s
    \end{bmatrix}
    =
    \xi_s
\end{align*}
\begin{equation*}
    \therefore
    \boxed{
    \operatorname{Ad}_{g_{s b}} \xi_{b}
    =
    \xi_s
    }
\end{equation*}
