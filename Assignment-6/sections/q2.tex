\section*{Problem 2}
\setcounter{section}{2}
\setcounter{equation}{0}

If \( \widehat{\xi} \in s e(3) \) is a twist, with twist coordinates \( \xi \in \mathbb{R}^{6} \), then show that for any \( g \in S E(3), g \widehat{\xi} g^{-1} \) is a twist, with twist coordinates \( \operatorname{Ad}_{g} \xi \in \mathbb{R}^{6} \)

\subsection*{Solution}

Given that \( \widehat{\xi} \in s e(3) \) is a twist, with twist coordinates \( \xi \in \mathbb{R}^{6} \), we can write \( \widehat{\xi} \) as
\begin{equation}\label{eq:twist}
    \xi
    =
    \begin{bmatrix}
        v \\
        \omega
    \end{bmatrix}
    \in \mathbb{R}^{6}
    \implies
    \widehat{\xi}
    =
    \begin{bmatrix}
        \widehat{\omega} & v \\
        0                & 0
    \end{bmatrix}
    \in s e(3)
\end{equation}
where \( \omega \in \mathbb{R}^{3} \) and \( v \in \mathbb{R}^{3} \).
Given \( g \in S E(3) \), we can write \( g \) as \( g = (p, R) \), where \( p \in \mathbb{R}^{3} \) and \( R \in S O(3) \).
In homogeneous coordinates, it is
\begin{equation*}
    g
    =
    \begin{bmatrix}
        R & p \\
        0 & 1
    \end{bmatrix}
    \in S E(3)
    \implies
    g^{-1}
    =
    \begin{bmatrix}
        R^{T} & -R^{T} p \\
        0     & 1
    \end{bmatrix}
\end{equation*}
We can now write \( g \widehat{\xi} g^{-1} \) as
\begin{align*}
    g \widehat{\xi} g^{-1}
     & =
    \begin{bmatrix}
        R & p \\
        0 & 1
    \end{bmatrix}
    \begin{bmatrix}
        \widehat{\omega} & v \\
        0                & 0
    \end{bmatrix}
    \begin{bmatrix}
        R^{T} & -R^{T} p \\
        0     & 1
    \end{bmatrix}
    =
    \begin{bmatrix}
        R & p \\
        0 & 1
    \end{bmatrix}
    \begin{bmatrix}
        \widehat{\omega} R^{T} & -\widehat{\omega} R^{T} p + v \\
        0                      & 0
    \end{bmatrix}
    \\
     & =
    \begin{bmatrix}
        R \widehat{\omega} R^{T} & R \widehat{\omega} R^{T} p + R v \\
        0                        & 0
    \end{bmatrix}
\end{align*}
From the form given in equation~\eqref{eq:twist}, we can see that \underline{\( g \widehat{\xi} g^{-1} \) is a indeed a twist}.

Let \( \widehat{\xi}' \) be the twist obtained from \( g \widehat{\xi} g^{-1} \), i.e., \( \widehat{\xi}' = g \widehat{\xi} g^{-1} \), and the correponding twist coordinates be \( \xi' \).
Then, to find \( \xi' \), we can write
\begin{equation*}
    \xi'
    =
    \begin{bmatrix}
        v' \\
        \omega'
    \end{bmatrix}
    \in \mathbb{R}^{6}
    \implies
    \widehat{\xi}'
    =
    \begin{bmatrix}
        \widehat{\omega}' & v' \\
        0                 & 0
    \end{bmatrix}
    \in s e(3)
\end{equation*}
Comparing the two forms, we then have
\begin{align*}
    \widehat{\omega}'
     & =
    R \widehat{\omega} R^{T}
     &
    \implies
    \omega'
     & =
    R \omega
    \\
    v'
     & =
    R \widehat{\omega} R^{T} p + R v
     &
    \implies
    v'
     & =
    R v + R \omega \times p
    =
    R v - R \widehat{p} \omega
\end{align*}
where we've used the property that \( R \omega R^T = \widehat{R \omega} \) and that \( \widehat{\omega} p = \omega \times p = - p \times \omega = - \widehat{p} \omega \).
\begin{equation*}
    \implies
    \xi'
    =
    \begin{bmatrix}
        R v - R \widehat{p} \omega \\
        R \omega
    \end{bmatrix}
    =
    \begin{bmatrix}
        R & \widehat{p} R \\
        0 & R
    \end{bmatrix}_{6\times 6}
    \begin{bmatrix}
        v \\
        \omega
    \end{bmatrix}
    \implies
    \boxed{
        \xi'
        =
        \operatorname{Ad}_{g} \xi
    }
\end{equation*}
