\section*{Problem 1}

Show that the adjoint transformation \( \operatorname{Ad}_{g}: \mathbb{R}^{6} \mapsto \mathbb{R}^{6} \) given below is invertible and compute its inverse.
\[
    \operatorname{Ad}_{g}=\left[\begin{array}{cc}
            R & \widehat{p} R \\
            0 & R
        \end{array}\right], \text { where, } g \equiv(p, R) \in SE(3)
\]

\subsection*{Solution}

Given \( g \in SE(3) \), we can write \( g \) as \( g = (p, R) \), where \( p \in \mathbb{R}^{3} \) and \( R \in SO(3) \). The adjoint transformation \( \operatorname{Ad}_{g}: \mathbb{R}^{6} \mapsto \mathbb{R}^{6} \) is defined by
\begin{equation*}
    \operatorname{Ad}_{g}
    =
    \begin{bmatrix}
        R & \widehat{p} R \\
        0 & R
    \end{bmatrix}_{6 \times 6}
\end{equation*}

To show that \( \operatorname{Ad}_{g} \) is invertible, we need to show that \( \operatorname{Ad}_{g} \) is a square matrix and that its determinant is non-zero.

\begin{lemma}
    (Schur's formula)
    If \( A \) is invertible, then the determinant of the block matrix \( M \) is given by
    \begin{equation*}
        \det M
        =
        \det
        \begin{bmatrix}
            A & B \\
            C & D
        \end{bmatrix}
        =
        \det(A)\det \left(D-CA^{-1}B\right)
    \end{equation*}
    where \( \left(D-CA^{-1}B\right) \) is the Schur complement of \( A \) in \( M \).
\end{lemma}
\begin{proof}
    Since \( A^{-1} \) exists, left-multiply \( M \) by the block matrix
    \(
    P = \begin{bmatrix}
        I        & 0 \\
        -CA^{-1} & I
    \end{bmatrix}
    \)
    \[
        \implies
        PM
        =
        \begin{bmatrix}
            I        & 0 \\
            -CA^{-1} & I
        \end{bmatrix}
        \begin{bmatrix}
            A & B \\
            C & D
        \end{bmatrix}
        =
        \begin{bmatrix}
            A & B            \\
            0 & D - CA^{-1}B
        \end{bmatrix}
    \]
    Now, the determinant of \( PM \) is given by
    \begin{align*}
        \det(PM)
         & =
        \det(P)\det(M)
        =
        1 \cdot \det(M)
        \\
        \implies
        \det(M)
         & =
        \det(PM)
        =
        \det(A) \det(D - CA^{-1}B)
    \end{align*}
\end{proof}

The matrix \( \operatorname{Ad}_{g} \) is a \( 6 \times 6 \) matrix, and its determinant is given by
\begin{equation*}
    \det(\operatorname{Ad}_{g})
    =
    \det
    \begin{bmatrix}
        R & \widehat{p} R \\
        0 & R
    \end{bmatrix}
    =
    \det( R ) \det \left( R - 0 \cdot R^\top \widehat{p} R \right)
    =
    \det( R ) \det( R )
    =
    1
\end{equation*}
Thereby, the \underline{inverse of \( \operatorname{Ad}_{g} \) exists}, and is given by
\begin{equation*}
    \boxed{
    \operatorname{Ad}_{g}^{-1}
    =
    \begin{bmatrix}
        R^{-1} & -R^{-1} \widehat{p} \\
        0      & R^{-1}
    \end{bmatrix}_{6 \times 6}
    =
    \begin{bmatrix}
        R^{T} & -R^{T} \widehat{p} \\
        0     & R^{T}
    \end{bmatrix}
    }
\end{equation*}

We can indeed verify that this is the inverse, by
\begin{align*}
    \operatorname{Ad}_{g}^{-1}
    \operatorname{Ad}_{g}
     & =
    \begin{bmatrix}
        R^{T} & -R^{T} \widehat{p} \\
        0     & R^{T}
    \end{bmatrix}
    \begin{bmatrix}
        R & \widehat{p} R \\
        0 & R
    \end{bmatrix}
    =
    \begin{bmatrix}
        R^{T} R & R^{T} \widehat{p} R - R^{T} \widehat{p} R \\
        0       & R^{T} R
    \end{bmatrix}
    =
    \begin{bmatrix}
        I_{3 \times 3} & 0              \\
        0              & I_{3 \times 3}
    \end{bmatrix}
    =
    I_{6 \times 6}
    \\
    \operatorname{Ad}_{g}
    \operatorname{Ad}_{g}^{-1}
     & =
    \begin{bmatrix}
        R & \widehat{p} R \\
        0 & R
    \end{bmatrix}
    \begin{bmatrix}
        R^{T} & -R^{T} \widehat{p} \\
        0     & R^{T}
    \end{bmatrix}
    =
    \begin{bmatrix}
        R R^{T} & -R R^{T} \widehat{p} + \widehat{p} R R^T \\
        0       & R R^{T}
    \end{bmatrix}
    \\ & =
    \begin{bmatrix}
        I_{3 \times 3} & - \widehat{p} + \widehat{p} \\
        0              & I_{3 \times 3}
    \end{bmatrix}
    =
    \begin{bmatrix}
        I_{3 \times 3} & 0              \\
        0              & I_{3 \times 3}
    \end{bmatrix}
    =
    I_{6 \times 6}
\end{align*}
