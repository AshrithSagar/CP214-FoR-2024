\section*{Problem 4}

A robot arm AB is rotating about point A with \( \omega_{AB} = 5 \text{rad/s} \) and \( \dot\omega_{AB} = 2 \text{rad/s} \).
Meanwhile, the forearm BC is rotating at a constant angular speed with respect to AB of \( \omega_{BC/AB} = 3 \text{rad/s} \).
Gravity cannot be neglected.
At the instant shown, find the net force acting on the object P, which has mass \( m = 1\text{kg} \).

\begin{figure*}[h]
    \centering
    \includegraphics[width=0.5\linewidth]{figures/images/q4.png}
\end{figure*}

\subsection*{Solution}

The robot arm AB is rotating about point A with an angular velocity \( \omega_{AB} = 5 \text{rad/s} \) and an angular acceleration \( \dot\omega_{AB} = 2 \text{rad/s} \).
The forearm BC is rotating at a constant angular speed with respect to AB of \( \omega_{BC/AB} = 3 \text{rad/s} \).
The object P has mass \( m = 1\text{kg} \).

The net force acting on the object P can be determined by considering the forces acting on the object.
The forces acting on the object P are the gravitational force \( m\vec{\mathbf{g}} \) and the centrifugal force \( m\vec{\mathbf{a}} \).
The centrifugal force acts radially outward from the axis of rotation.

The centrifugal force acting on the object P can be determined by considering the relative angular velocity of the forearm BC with respect to AB.
The relative angular velocity of the forearm BC with respect to AB is given by:
\[
    \omega_{BC/P} = \omega_{BC/AB} + \omega_{AB} = 3 + 5 = 8 \text{rad/s}
\]

The centrifugal acceleration of the object P is given by:
\[
    a = r\omega^2 = 0.5 \times 8^2 = 32 \text{m/s}^2
\]

The centrifugal force acting on the object P is given by:
\[
    F_{\text{centrifugal}} = ma = 1 \times 32 = 32 \text{N}
\]

The gravitational force acting on the object P is given by:
\[
    F_{\text{gravity}} = mg = 1 \times 9.81 = 9.81 \text{N}
\]

The net force acting on the object P is given by:
\[
    F_{\text{net}} = F_{\text{centrifugal}} + F_{\text{gravity}} = 32 + 9.81 = 41.81 \text{N}
\]

Therefore, the net force acting on the object P is \( 41.81 \text{N} \).
