\section*{Problem 4}

A robot arm AB is rotating about point A with \( \omega_{AB} = 5 \text{rad/s} \) and \( \dot\omega_{AB} = 2 \text{rad/s} \).
Meanwhile, the forearm BC is rotating at a constant angular speed with respect to AB of \( \omega_{BC/AB} = 3 \text{rad/s} \).
Gravity cannot be neglected.
At the instant shown, find the net force acting on the object P, which has mass \( m = 1\text{kg} \).

\begin{figure*}[h]
    \centering
    \includegraphics[width=0.5\linewidth]{figures/images/q4.png}
\end{figure*}

\subsection*{Solution: Parametric method}

\begin{figure*}[htb]
    \centering
    \includegraphics[page=1, width=0.4\linewidth]{figures/q4/_}
    \caption{
        Parametric form of the solution
    }
\end{figure*}

Consider the above parametric form of the problem setup, with parameters \(\theta_1\) and \(\theta_2\), and with \(l_1\) and \(l_2\) being the fixed lengths of the robot arm AB and the forearm BC, respectively.
The position vector of the object P is given by
\[
    \vec{\mathbf{r}}_P =
    \begin{bmatrix}
        l_1 \cos\theta_1 + l_2 \cos\theta_2 \\
        l_1 \sin\theta_1 + l_2 \sin\theta_2 \\
    \end{bmatrix}
\]
Differentiating the position vector w.r.t.\ time, we get the velocity vector as
\[
    \vec{\mathbf{v}}_P =
    \begin{bmatrix}
        -l_1 \dot\theta_1 \sin\theta_1 - l_2 \dot\theta_2 \sin\theta_2 \\
        l_1 \dot\theta_1 \cos\theta_1 + l_2 \dot\theta_2 \cos\theta_2  \\
    \end{bmatrix}
\]
Differentiating the velocity vector w.r.t.\ time, we get the acceleration vector as
\[
    \vec{\mathbf{a}}_P =
    \begin{bmatrix}
        -l_1 \ddot\theta_1 \sin\theta_1 - l_1 \dot\theta_1^2 \cos\theta_1 - l_2 \ddot\theta_2 \sin\theta_2 - l_2 \dot\theta_2^2 \cos\theta_2 \\
        l_1 \ddot\theta_1 \cos\theta_1 - l_1 \dot\theta_1^2 \sin\theta_1 + l_2 \ddot\theta_2 \cos\theta_2 - l_2 \dot\theta_2^2 \sin\theta_2  \\
    \end{bmatrix}
\]

From the problem statement, we have the following values for the parameters
\[
    \begin{aligned}
        l_1           & = 2\text{ m}                                                              \\
        l_2           & = 1\text{ m}                                                              \\
        \theta_1      & = 30^\circ = \frac{\pi}{6}\text{ rad}                                     \\
        \theta_2      & = \theta_{BC} = 0^\circ = 0\text{ rad}                                    \\
        \dot\theta_1  & = \omega_{AB} = 5\text{ rad/s}                                            \\
        \dot\theta_2  & = \omega_{BC} = \omega_{AB} + \omega_{BC/AB} = 8\text{ rad/s}             \\
        \ddot\theta_1 & = \dot\omega_{AB} = 2\text{ rad/s}^2                                      \\
        \ddot\theta_2 & = \dot\omega_{BC} = \dot\omega_{AB} + \dot\omega_{BC/AB} = 2\text{ rad/s} \\
    \end{aligned}
\]

Substituting the above values into the acceleration vector, we get, for the given instant
\begin{align*}
    \vec{\mathbf{a}}_P
     & =
    \begin{bmatrix}
        -2 \cdot 2 \cdot \sin\frac{\pi}{6} - 2 \cdot 5^2 \cdot \cos\frac{\pi}{6} - 1 \cdot 0 \cdot \sin 0 - 1 \cdot 8^2 \cdot \cos 0 \\
        2 \cdot 2 \cdot \cos\frac{\pi}{6} - 2 \cdot 5^2 \cdot \sin\frac{\pi}{6} + 1 \cdot 2 \cdot \cos 0 - 1 \cdot 8^2 \cdot \sin 0  \\
    \end{bmatrix} \\
     & =
    \begin{bmatrix}
        -2 - 25\sqrt{3} - 64 \\
        2\sqrt{3} - 25 + 2   \\
    \end{bmatrix}                                                                                                         \\
     & =
    \begin{bmatrix}
        - 25\sqrt{3} - 66 \\
        2\sqrt{3} - 23    \\
    \end{bmatrix}                                                                                                            \\
\end{align*}

The force can be found out using Newton's laws as
\begin{equation*}
    \vec{\mathbf{F}}_P
    =
    m \vec{\mathbf{a}}_P = 1 \cdot
    \begin{bmatrix}
        - 25\sqrt{3} - 66 \\
        2\sqrt{3} - 23    \\
    \end{bmatrix}
\end{equation*}

Hence, the force experienced by the object P is given by
\begin{align*}
    \vec{\mathbf{F}}_P
     & =
    \boxed{
        (- 25\sqrt{3} - 66) \vec{\mathbf{\imath}} + (2\sqrt{3} - 23) \vec{\mathbf{\jmath}}
    }
    \\ & =
    \boxed{
        -109.30 \vec{\mathbf{\imath}} -19.54 \vec{\mathbf{\jmath}}
    }
\end{align*}
\begin{equation*}
    \boxed{
        \lVert \vec{\mathbf{F}}_P \rVert = 111.04 \text{ N}
    }
\end{equation*}

% \newpage
% \subsection*{Solution-2: Pseudo forces}

% \begin{figure*}[htb]
%     \centering
%     \includegraphics[page=2, width=0.4\linewidth]{figures/q4/_}
%     \caption{
%         Free body diagram
%     }
% \end{figure*}
