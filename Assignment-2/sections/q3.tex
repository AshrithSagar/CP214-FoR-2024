\section*{Problem 3}

An autonomous car needs to maintain a constant speed \( v \) as it moves along the curved surface of a hill, which can be approximated as a circle with radius \( R \).
The car starts from point O at the base of the hill.
The car's ``AI'' must ensure it does not lift off the ground as it reaches the top of the hill.
Therefore, determine the maximum speed \( v_{\max} \) the AI should set to prevent the car from leaving the surface at the hill's peak.

\begin{figure*}[h]
    \centering
    \includegraphics[width=0.5\linewidth]{figures/images/q3.png}
\end{figure*}

\subsection*{Solution}

The car is moving along a circular path with radius \( R \) and speed \( v \).
The car is subjected to two forces: the gravitational force \( m\vec{\mathbf{g}} \) and the normal reaction force \( \vec{\mathbf{N}} \).
The gravitational force acts vertically downwards, and the normal reaction force acts perpendicular to the surface of the hill.
The normal reaction force provides the centripetal force required to keep the car moving in a circular path.

The gravitational force can be resolved into two components: \( m\vec{\mathbf{g}}_x \) and \( m\vec{\mathbf{g}}_y \).
The component \( m\vec{\mathbf{g}}_x \) is balanced by the normal reaction force \( \vec{\mathbf{N}} \), and the component \( m\vec{\mathbf{g}}_y \) is balanced by the centripetal force.
The centripetal force is given by \( m\vec{\mathbf{a}} = m\frac{v^2}{R} \), where \( \vec{\mathbf{a}} \) is the acceleration of the car.

The gravitational force \( m\vec{\mathbf{g}} \) can be resolved into two components:
\begin{align*}
    m\vec{\mathbf{g}}_x & = -m\sin(\theta)g \\
    m\vec{\mathbf{g}}_y & = -m\cos(\theta)g
\end{align*}

The normal reaction force \( \vec{\mathbf{N}} \) is equal in magnitude and opposite in direction to the component \( m\vec{\mathbf{g}}_x \).
Therefore, the normal reaction force is given by:
\[
    \vec{\mathbf{N}} = m\sin(\theta)g
\]

The centripetal force required to keep the car moving in a circular path is given by:
\[
    m\frac{v^2}{R} = m\cos(\theta)g
\]

The maximum speed \( v_{\max} \) can be determined by substituting \( \theta = 0 \) in the above equation:
\[
    v_{\max} = \sqrt{Rg}
\]
