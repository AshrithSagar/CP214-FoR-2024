\section*{Problem 3}

An autonomous car needs to maintain a constant speed \( v \) as it moves along the curved surface of a hill, which can be approximated as a circle with radius \( R \).
The car starts from point O at the base of the hill.
The car's ``AI'' must ensure it does not lift off the ground as it reaches the top of the hill.
Therefore, determine the maximum speed \( v_{\max} \) the AI should set to prevent the car from leaving the surface at the hill's peak.

\begin{figure*}[h]
    \centering
    \includegraphics[width=0.5\linewidth]{figures/images/q3.png}
\end{figure*}

\subsection*{Solution}

The FBD of the car is as follows:
\begin{figure*}[h]
    \centering
    \includegraphics[width=0.3\linewidth]{figures/q3/_.pdf}
    \caption{
        Free Body Diagram of the car.
    }
\end{figure*}

There are two forces acting on the car, the gravitational force \( m\vec{\mathbf{g}} \) and the normal reaction force \( \vec{\mathbf{N}} \).
At the hill's peak, the normal reaction force is perpendicular to the gravitational force.
For the car to not lift off the ground at the hill's peak, we should have that the normal reaction force at that point is just zero, and the gravitational force is sufficient to balance the centripetal reaction force at that point.

Thereby, we have, at the hill's peak,
\begin{align*}
    \vec{\mathbf{F}}_{\textbf{centripetal}}
     & =
    m\vec{\mathbf{g}}
    \\ \implies
    m\frac{v^2}{R}
     & =
    mg
\end{align*}

Solving for \( v \), we get
\begin{align*}
    v^2 & = gR        \\
    \implies
    v   & = \sqrt{gR}
\end{align*}

Thereby, we obtain a threshold velocity above which the gravitational force is insufficient to balance the centripetal reaction force at the hill's peak, and the car will lift off the ground in this situation.
Thereby, this is the maximum velocity that can be set by the AI to prevent the car from leaving the surface at the hill's peak.
\[
    \boxed{v_{\max} = \sqrt{gR}}
\]
