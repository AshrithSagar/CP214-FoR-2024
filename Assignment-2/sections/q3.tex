\section*{Problem 3}

An autonomous car needs to maintain a constant speed \( v \) as it moves along the curved surface of a hill, which can be approximated as a circle with radius \( R \).
The car starts from point O at the base of the hill.
The car's ``AI'' must ensure it does not lift off the ground as it reaches the top of the hill.
Therefore, determine the maximum speed \( v_{\max} \) the AI should set to prevent the car from leaving the surface at the hill's peak.

\begin{figure*}[h]
    \centering
    \includegraphics[width=0.5\linewidth]{figures/images/q3.png}
\end{figure*}

\subsection*{Solution}

The FBD of the car is as follows:
\begin{figure*}[h]
    \centering
    \includegraphics[width=0.3\linewidth]{figures/q3/_.pdf}
    \caption{
        Free Body Diagram of the car.
    }
\end{figure*}

At the peak, the car should have zero acceleration in the vertical direction.
The two forces that act in this case are the gravitational force and the centripetal force, which should balance out to ensure that the car doesn't lift off the ground at the hill's peak.
Thereby, we have, at the hill's peak,
\begin{align*}
    \vec{\mathbf{F}}_{\textbf{centripetal}} + m\vec{\mathbf{g}}
     & =
    \vec{\mathbf{0}} \\
    \implies
    m\frac{v^2}{R} - mg
     & =
    0
\end{align*}

Solving for \( v \), we get
\begin{align*}
    v^2 & = gR        \\
    \implies
    v   & = \sqrt{gR}
\end{align*}

Therefore, the maximum speed \( v_{\max} \) the AI should set to prevent the car from leaving the surface at the hill's peak is \( \boxed{v_{\max} = \sqrt{gR}} \).
