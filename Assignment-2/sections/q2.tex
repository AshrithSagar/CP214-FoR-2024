\section*{Problem 2}

For the bicycle shown in the figure, assume the mass of the bicycle (and possibly the rider) to be a point mass located at C.
A vertical downward force \( \vec{F} \) is applied on the front pedal.

\begin{enumerate}[label = (\alph*)]
    \item Draw a free-body diagram of the front wheel.
    \item Draw a free-body diagram of the back wheel.
    \item Draw a free-body diagram of the entire bicycle.
\end{enumerate}

\begin{figure*}[h]
    \centering
    \includegraphics[width=0.45\linewidth]{figures/images/q2.png}
\end{figure*}

\subsection*{Solution}

The bicycle is moving forward due to the action of the pedal force \( \vec{\mathbf{F}} \).
The friction generated due to the movement of the front wheel tries to \textit{oppose} it, and acts in a direction opposite to the direction of rotation of the front wheel, i.e., towards the right.
Similarly, the friction generated in the back wheel is responsible for \textit{causing} the rotation in the back wheel.

\begin{figure}[htbp]
    \centering
    \includegraphics[page=1, width=0.45\linewidth]{figures/q2/_}
    \caption{
        Free-body diagram of the front wheel.
    }
\end{figure}

\begin{figure}[htbp]
    \centering
    \includegraphics[page=2, width=0.3\linewidth]{figures/q2/_}
    \caption{
        Free-body diagram of the back wheel.
    }
\end{figure}

\begin{figure}[htbp]
    \centering
    \includegraphics[page=3, width=0.5\linewidth]{figures/q2/_}
    \caption{
        Free-body diagram of the entire bicycle.
    }
\end{figure}
