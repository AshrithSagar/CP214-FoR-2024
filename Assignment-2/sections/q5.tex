\section*{Problem 5}

A humanoid robot has to balance a stick AC by moving its hands horizontally (only in x direction).
For this problem, it models the hand contact with the stick as a hinge at point A.
The stick has all the mass concentrated at point C.
Answer the following questions

\begin{enumerate}[label = (\alph*)]
    \item \textbf{Equation of motion -} Given the acceleration of its hand (horizontal only), the current stick angle \( \phi \), and the rate of the stick's angle \( \dot\phi \), find the stick's angular acceleration.
    \item \textbf{Control -} Can you find a hand acceleration in terms of \( \phi \) and \( \dot\phi \) that will make the stick balance upright?
\end{enumerate}

\begin{figure*}[h]
    \centering
    \includegraphics[width=0.45\linewidth]{figures/images/q5.png}
\end{figure*}

\subsection*{Solution}

\subsubsection*{(a) Equation of motion}

Let the horizontal acceleration of the hand be \( \mathbf{\ddot x} \) in the positive \(x\)-direction, the current angle of the stick be \( \phi_0 \), and the rate of change of the stick's angle be \( \dot\phi_0 \).
Considering the FBD of the point C w.r.t.\ the inertial frame C, we have the FBD as given in Figure~\ref{fig:q5a-fbd}.

\begin{figure*}[htb]
    \centering
    \includegraphics[width=0.6\linewidth]{figures/q5/_}
    \caption{
        Free body diagram of the stick.
    }\label{fig:q5a-fbd}
\end{figure*}

The gravitational force acts downwards at point C.
The mass at point C also experiences a pseudo force towards the left due to the acceleration of the hand, which is an inertial frame of reference.
The direction of this pseudo-force is opposite to the direction of the acceleration of the hand, and it's magnitude is given by \( m\ddot x \).
Thereby, we can calculate the torque at point C as
\begin{equation}
    \vec{\boldsymbol{\tau}} = mL\sin\phi \; \vec{\mathbf{g}} - mL\cos\phi \; \vec{\ddot{\mathbf{x}}} \\
    \tag{5a.1}
\end{equation}

Since, C is a point mass, we have the mass moment of interia as \( I = mL^2 \), giving
\begin{equation}
    \vec{\boldsymbol{\tau}} = I \vec{\boldsymbol{\alpha}} = mL^2 \vec{\ddot{\boldsymbol{\phi}}}
    \tag{5a.2}
\end{equation}

Equating equations~(5a.1) and (5a.2), we get
\begin{align*}
    mL^2 \vec{\ddot{\boldsymbol{\phi}}}
     & =
    mL\sin\phi \; \vec{\mathbf{g}} - mL\cos\phi \; \vec{\ddot{\mathbf{x}}} \\
    \implies
    L \vec{\ddot{\boldsymbol{\phi}}}
     & =
    \sin\phi \; \vec{\mathbf{g}} - \cos\phi \; \vec{\ddot{\mathbf{x}}}
\end{align*}

Hence, the equation of motion is given by
\begin{equation}
    \boxed{
        \ddot{\phi} = \frac{g}{L} \sin\phi - \frac{\ddot x}{L} \cos\phi
    }
    \tag{5a.3}
\end{equation}

\subsubsection*{(b) Control}

To balance the stick upright, we need to have \( \ddot\phi = 0 \).

Thereby, from Equation~(5a.3), we have
\begin{align*}
    \frac{\ddot x}{L} \cos\phi
     & =
    \frac{g}{L} \sin\phi
    \\ \implies
        \ddot x
     & =
        g \tan\phi
\end{align*}
