\section*{Problem 5}

A humanoid robot has to balance a stick AC by moving its hands horizontally (only in x direction).
For this problem, it models the hand contact with the stick as a hinge at point A.
The stick has all the mass concentrated at point C.
Answer the following questions

\begin{enumerate}[label = (\alph*)]
    \item \textbf{Equation of motion -} Given the acceleration of its hand (horizontal only), the current stick angle \( \phi \), and the rate of the stick's angle \( \dot\phi \), find the stick's angular acceleration.
    \item \textbf{Control -} Can you find a hand acceleration in terms of \( \phi \) and \( \dot\phi \) that will make the stick balance upright?
\end{enumerate}

\begin{figure*}[h]
    \centering
    \includegraphics[width=0.45\linewidth]{figures/images/q5.png}
\end{figure*}

\subsection*{Solution}

\begin{enumerate}[label = (\alph*)]
    \item \textbf{Equation of motion -} The stick is balanced by the forces acting on it.
          The forces acting on the stick are the gravitational force \( m\vec{\mathbf{g}} \) and the force exerted by the hand at point A.
          The force exerted by the hand at point A can be resolved into two components: \( F_x \) and \( F_y \).
          The force \( F_x \) is responsible for the angular acceleration of the stick.

          The gravitational force acting on the stick is given by:
          \[
              m\vec{\mathbf{g}} = -mg\hat{\mathbf{j}}
          \]

          The force exerted by the hand at point A is given by:
          \[
              \vec{\mathbf{F}} = F_x\hat{\mathbf{i}} + F_y\hat{\mathbf{j}}
          \]

          The torque acting on the stick about point A is given by:
          \[
              \vec{\mathbf{\tau}} = \vec{\mathbf{r}} \times \vec{\mathbf{F}} = -l\hat{\mathbf{k}} \times (F_x\hat{\mathbf{i}} + F_y\hat{\mathbf{j}}) = -lF_y\hat{\mathbf{i}} + lF_x\hat{\mathbf{j}}
          \]

          The torque acting on the stick about point A is equal to the moment of inertia of the stick about point C times the angular acceleration of the stick:
          \[
              -lF_y\hat{\mathbf{i}} + lF_x\hat{\mathbf{j}} = I_C\ddot{\phi}\hat{\mathbf{k}}
          \]

          The moment of inertia of the stick about point C is given by:
          \[
              I_C = ml^2
          \]

          Therefore, the equation of motion for the stick is:
          \[
              -lF_y = ml^2\ddot{\phi}
          \]

    \item \textbf{Control -} To make the stick balance upright, the stick's angular acceleration \( \ddot{\phi} \) should be zero.
          Therefore, the force exerted by the hand at
          point A should be zero in the y-direction:
          \[
              F_y = 0
          \]

          The force exerted by the hand at point A in the x-direction is responsible for the angular acceleration of the stick.
          Therefore, the force exerted by the hand at point A in the x-direction should be such that it balances the gravitational force acting on the stick:
          \[
              F_x = mg
          \]

          Therefore, the hand acceleration in terms of \( \phi \) and \( \dot\phi \) that will make the stick balance upright is:
          \[
              \vec{\mathbf{a}} = \ddot{\vec{\mathbf{r}}} = \ddot{\phi}l\hat{\mathbf{i}} = 0
          \]
\end{enumerate}
