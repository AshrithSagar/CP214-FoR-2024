\section*{Problem 6}

\textbf{Drone Orientation Future Prediction:}
In this problem, we model the drone as a single rigid body whose current orientation is represented by a rotation matrix \(R_{c} \in S O(3)\).
While the drone is moving, its orientation changes depending on the thrust forces generated.
Using some math (which you will learn about by the end of the course), we can derive the following rule to predict the future orientations of the drone:
\begin{equation*}
    R_{k+1}=R_{k} \exp \left(\omega_{k} \Delta t\right)
\end{equation*}
where \(R_{k}\) is the rotation at the \(k^{\text {th }}\) time step
(Note: For \(k=1\), we have \(\left.R_{k}=R_{c}\right), \Delta t\)
is a discretization constant, and \(\omega_{k}\) is the drone's angular velocity at the \(k^{\text {th }}\) time step
(Note: It has its own update rule, which you need not worry about in this problem).
The operator \(\exp (\cdot)\) is the matrix exponential from \(s o(3)\) to \(S O(3)\) (also known as Rodrigues' rotation formula).

The use case of such a rule would be in algorithms like Model Predictive Control (MPC), which make future predictions to determine the best decisions/controls/actions.
However, working with the above-stated rule is tedious due to the nonlinear relationship between \(R_{k+1}\) and \(\omega_{k}\).
Therefore, show that using certain assumptions, we can obtain the following simpler form, which is easier to work with:
\begin{equation*}
    R_{k+1}=R_{c}\left(I+\Delta t \sum_{j=1}^{k} \omega_{j}\right)
\end{equation*}

\subsection*{Solution}

Given the single rigid model of the drone whose current orientation is represented by the rotation matrix \( R_c \in \text{SO}(3) \), we can rewrite the given relation as follows:
\begin{equation*}
    R_{k+1}
    =
    R_k \exp(\omega_k \Delta t)
    \implies
    \frac{R_{k+1}}{R_k}
    =
    \exp(\omega_k \Delta t)
\end{equation*}
Given that \( R_{1} = R_c \), we can arrive at the following by multiplying as a telescoping series:
\begin{equation*}
    \implies
    \frac{R_{k+1}}{R_1}
    =
    \frac{R_{k+1}}{\cancel{R_k}}
    \frac{\cancel{R_k}}{\cancel{R_{k-1}}}
    \cdots
    \frac{\cancel{R_2}}{R_1}
    =
    \exp(\omega_k \Delta t)
    \exp(\omega_{k-1} \Delta t)
    \cdots
    \exp(\omega_1 \Delta t)
\end{equation*}
\begin{equation*}
    \implies
    R_{k+1}
    =
    R_1 \prod_{j=1}^{k} \exp(\omega_j \Delta t)
    =
    R_1 \exp \Big( \sum_{j=1}^{k} \omega_j \Delta t \Big)
    =
    R_c \exp \Big( \Delta t \sum_{j=1}^{k} \omega_j \Big)
\end{equation*}
since \( \Delta t \) is a constant, as given.

We can see that this is the non-linear equation for the orientation of the drone.
To linearise it, we will use the Taylor (Maclaurin) series expansion and truncate the higher order terms.
The Taylor series expansion is given by:
\begin{equation*}
    \exp(a \Delta t)
    =
    I + a \Delta t + \frac{1}{2!} (a \Delta t)^2 + \frac{1}{3!} (a \Delta t)^3 + \cdots
    \approx
    I + a \Delta t
\end{equation*}
This approximation is valid for small values of \( a \Delta t \), so that the higher order terms become negligible.
Thereby, we can write the linearised equation as:
\begin{align*}
    R_{k+1}
     & =
    R_c \exp \Big( \Delta t \sum_{j=1}^{k} \omega_j \Big)
    \\
    \implies
    R_{k+1}
     & \approx
    R_c \Big( I + \Delta t \sum_{j=1}^{k} \omega_j \Big)
\end{align*}

This simplified linear form is much easier to work with than the non-linear form.
