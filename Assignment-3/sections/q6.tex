\section*{Problem 6}

Drone Orientation Future Prediction:
In this problem, we model the drone as a single rigid body whose current orientation is represented by a rotation matrix \(R_{c} \in S O(3)\).
While the drone is moving, its orientation changes depending on the thrust forces generated.
Using some math (which you will learn about by the end of the course), we can derive the following rule to predict the future orientations of the drone:
\begin{equation*}
    R_{k+1}=R_{k} \exp \left(\omega_{k} \Delta t\right)
\end{equation*}
where \(R_{k}\) is the rotation at the \(k^{\text {th }}\) time step
(Note: For \(k=1\), we have \(\left.R_{k}=R_{c}\right), \Delta t\)
is a discretization constant, and \(\omega_{k}\) is the drone's angular velocity at the \(k^{\text {th }}\) time step
(Note: It has its own update rule, which you need not worry about in this problem).
The operator \(\exp (\cdot)\) is the matrix exponential from \(s o(3)\) to \(S O(3)\) (also known as Rodrigues' rotation formula).

The use case of such a rule would be in algorithms like Model Predictive Control (MPC), which make future predictions to determine the best decisions/controls/actions.
However, working with the above-stated rule is tedious due to the nonlinear relationship between \(R_{k+1}\) and \(\omega_{k}\).
Therefore, show that using certain assumptions, we can obtain the following simpler form, which is easier to work with:
\begin{equation*}
    R_{k+1}=R_{c}\left(I+\Delta t \sum_{j=1}^{k} \omega_{j}\right)
\end{equation*}

\subsection*{Solution}
