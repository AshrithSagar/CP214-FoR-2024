\section*{Problem 6}

\textbf{Drone Orientation Future Prediction:}
In this problem, we model the drone as a single rigid body whose current orientation is represented by a rotation matrix \(R_{c} \in S O(3)\).
While the drone is moving, its orientation changes depending on the thrust forces generated.
Using some math (which you will learn about by the end of the course), we can derive the following rule to predict the future orientations of the drone:
\begin{equation*}
    R_{k+1}=R_{k} \exp \left(\omega_{k} \Delta t\right)
\end{equation*}
where \(R_{k}\) is the rotation at the \(k^{\text {th }}\) time step
(Note: For \(k=1\), we have \(\left.R_{k}=R_{c}\right), \Delta t\)
is a discretization constant, and \(\omega_{k}\) is the drone's angular velocity at the \(k^{\text {th }}\) time step
(Note: It has its own update rule, which you need not worry about in this problem).
The operator \(\exp (\cdot)\) is the matrix exponential from \(s o(3)\) to \(S O(3)\) (also known as Rodrigues' rotation formula).

The use case of such a rule would be in algorithms like Model Predictive Control (MPC), which make future predictions to determine the best decisions/controls/actions.
However, working with the above-stated rule is tedious due to the nonlinear relationship between \(R_{k+1}\) and \(\omega_{k}\).
Therefore, show that using certain assumptions, we can obtain the following simpler form, which is easier to work with:
\begin{equation*}
    R_{k+1}=R_{c}\left(I+\Delta t \sum_{j=1}^{k} \omega_{j}\right)
\end{equation*}

\subsection*{Solution}

Given the single rigid body model of the drone whose current orientation is represented by the rotation matrix \( R_c \in \text{SO}(3) \), we have
\begin{equation*}
    R_{k+1}
    =
    R_k \exp(\omega_k \Delta t)
\end{equation*}
where \( R_k \) are rotations matrices at the \( k^{\text{th}} \) time step, with \( R_{1} = R_c \), and \( \omega_k \) are the drone's angular velocities at the \( k^{\text{th}} \) time step.
The \( \omega_k \) terms are written as a skew-symmetric matrix \( \hat\omega \) such that they are in \( so(3) \).
The Rodrigues rotation formula provides a way to compute the exponential map \( \exp(.) \) from \( \hat \omega \in so(3)\) to \( R \in SO(3)\) for some \( \theta \in \mathbb{R} \) as
\[
    R = I +(\sin \theta )\hat\omega +(1-\cos \theta )\hat\omega^2
\]
Given that \( R_{1} = R_c \), we can observe the following:
\begin{align*}
    R_{2}
     & =
    R_{c} \exp(\omega_1 \Delta t)
    \\
    R_{3}
     & =
    R_{2} \exp(\omega_2 \Delta t)
    \\ &
    \vdots
    \\
    R_{k+1}
     & =
    R_{k} \exp(\omega_k \Delta t)
    \\
\end{align*}
\begin{align*}
    \implies
    R_{k+1}
     & =
    R_c \exp(\omega_1 \Delta t) \exp(\omega_2 \Delta t) \cdots \exp(\omega_k \Delta t)
    \\
    \implies
    R_{k+1}
     & =
    R_c \prod_{j=1}^{k} \exp(\omega_j \Delta t)
\end{align*}
where on the right hand side, we have a product of matrix exponentials.
Note the order of the terms, since matrix multiplication is in general not commutative.

We can see that this is the non-linear equation for the orientation of the drone.
To linearise it, we will use the Taylor (Maclaurin) series expansion and truncate the higher order terms.
The Taylor series expansion for \( \exp(x) \) for a scalar \( x \) is given by:
\begin{equation*}
    \exp(x)
    =
    1 + x + \frac{1}{2!} x^2 + \frac{1}{3!} x^3 + \cdots
    =
    \sum_{k=0}^{\infty} \frac{x^k}{k!}
\end{equation*}
which is an analytic function since the power series converges.
Thereby, we can write the matrix analogue of the above equation as
\begin{equation*}
    \exp(A)
    =
    I + A + \frac{1}{2!} A^2 + \frac{1}{3!} A^3 + \cdots
    =
    \sum_{k=0}^{\infty} \frac{1}{k!} A^k
\end{equation*}
With the assumption that the \underline{discretization constant \( \Delta t \) is small}, we can approximate the exponential map as a linear function by a first-order approximation.
For some \( \omega \in so(3)\), we have
\begin{align*}
    \exp(\omega \Delta t)
     & =
    I + \omega \Delta t + \frac{1}{2!} {(\omega \Delta t)}^2 + \frac{1}{3!} {(\omega \Delta t)}^3 + \cdots
    \\ & =
    I + \omega \Delta t + \frac{{(\Delta t)}^2}{2!} \omega^2 + \frac{{(\Delta t)}^3}{3!} \omega^3 + \cdots
    \\ & =
    I + \omega \Delta t + O({\Delta t}^2)
    \\
     & \approx
    I + \omega \Delta t
\end{align*}
where we have that the higher-order terms are of order \( O({\Delta t}^2) \) and go to zero as \( \Delta t \) goes to zero.
This approximation is valid for small values of \( \Delta t \) and small changes of the angular velocity.
Thereby, we can write the linearised equation as
\begin{align*}
    R_{k+1}
     & =
    R_c \exp(\omega_1 \Delta t) \exp(\omega_2 \Delta t) \cdots \exp(\omega_k \Delta t)
    \\ & \approx
    R_c (I + \omega_1 \Delta t) (I + \omega_2 \Delta t) \cdots (I + \omega_k \Delta t)
    \\ & =
    R_c \left( I + \Delta t \sum_{j=1}^{k} \omega_j + O({\Delta t}^2) \right)
\end{align*}
\begin{equation*}
    \implies
    \boxed{
        R_{k+1}
        \approx
        R_c \left( I + \Delta t \sum_{j=1}^{k} \omega_j \right)
    }
\end{equation*}

This simplified linear form is much easier to work with than the non-linear form.
