\section*{Problem 1}

Prove that given \(R \in S O(3)\) and \(v, w \in \mathbb{R}^{3}\), the following properties hold:
\begin{align*}
    R(v \times w)   & =(R v) \times(R w) \\
    R \hat{w} R^{T} & =\widehat{R w}
\end{align*}

\subsection*{Solution}

\subsubsection*{(a) \( R(v \times w) =(R v) \times(R w) \)}

This property illustrates the rotational invariance of cross product.

We know that the special orthogonal group \( SO(3) \) is defined as
\[
    SO(3) = \{ R \in \mathbb{R}^{3 \times 3} \mid R R^{T} = I, \det(R) = +1 \}
\]

We will show that the rotation matrix \( R \) preserves length, the dot product and the cross product of two vectors.
Let \( v, w \in \mathbb{R}^{3} \) be two vectors.
\begin{align*}
    \implies
    \left \| R v \right \| ^{2}
        = {(R v)}^{T} (R v)
    = (v^{T} R^{T}) (R v)
    = v^{T} (R^{T} R) v
    = v^{T} v
    = \left \| v \right \| ^{2}
\end{align*}
Thereby, transformation by the rotation matrix \( R \) \underline{preserves the length} of a vector.

We can show that any length-preserving transformation preserves the dot product, by using the fact that the dot product can completely be expressed in terms of the lengths of the vectors as seen in the following identity, which is by the law of cosines:
\begin{align*}
    v \cdot w
     & = \frac{1}{2} \left( \left \| v + w \right \| ^{2} - \left \| v \right \| ^{2} - \left \| w \right \| ^{2} \right)
    \\ \implies
    (R v) \cdot (R w)
     & = \frac{1}{2} \left( \left \| R v + R w \right \| ^{2} - \left \| R v \right \| ^{2} - \left \| R w \right \| ^{2} \right)
    \\ & = \frac{1}{2} \left( \left \| R (v + w) \right \| ^{2} - \left \| R v \right \| ^{2} - \left \| R w \right \| ^{2} \right)
    \\ & = \frac{1}{2} \left( \left \| v + w \right \| ^{2} - \left \| v \right \| ^{2} - \left \| w \right \| ^{2} \right)
    = v \cdot w
\end{align*}
Thereby, transformation by the rotation matrix \( R \) \underline{preserves the dot product} of two vectors, and thereby also \underline{preserves the angle} between the vectors.

We can see that the cross product of two vectors can be expressed in terms of length and the angle between the vectors, since
\[
    v \times w = \left \| v \right \| \left \| w \right \| \sin(\theta) \hat{n}
\]
where \( \hat{n} \) is the unit vector perpendicular to the plane containing \( v \) and \( w \), and \( \theta \) is the angle between the vectors.
The convention is to use the right-hand rule to determine the direction of the cross product.

Let \( u, v, w \in \mathbb{R}^{3} \) be three vectors.
Now, consider the scalar triple product \( u \cdot (v \times w) \), which can be written as a determinant as follows:
\[
    u \cdot (v \times w)
    =
    \begin{vmatrix}
        u_{1} & v_{1} & w_{1} \\
        u_{2} & v_{2} & w_{2} \\
        u_{3} & v_{3} & w_{3}
    \end{vmatrix}
    =
    \det
    \begin{bmatrix}
        u & v & w
    \end{bmatrix}_{3 \times 3}
\]
From this, we can see that after a rotation by \( R \), we would have
\begin{align*}
    R u \cdot (R v \times R w)
     & =
    \det
    \begin{bmatrix}
        R u & R v & R w
    \end{bmatrix}_{3 \times 3}
    \\ & =
    \det \Big(
    R \begin{bmatrix}
          u & v & w
      \end{bmatrix}_{3 \times 3}
    \Big)
    \\ & =
    (\det R)
    \Big( \det
    \begin{bmatrix}
        u & v & w
    \end{bmatrix}_{3 \times 3} \Big)
    \\ & =
    1 \cdot
    \det
    \begin{bmatrix}
        u & v & w
    \end{bmatrix}_{3 \times 3}
    \\ & =
    u \cdot (v \times w)
\end{align*}
But now since the dot product is preserved, as shown above, we have that, by considering \( (v \times w) \) as one quantity,
\[
    u \cdot (v \times w)
    =
    R u \cdot R (v \times w)
\]
Thereby,
\[
    R u \cdot R (v \times w)
    =
    R u \cdot (R v \times R w)
\]
Now, since the dot product is preserved as shown previously and since the above equation holds for all \( u, v, w \in \mathbb{R}^{3} \), we get
\[
    R (v \times w)
    =
    R v \times R w
\]
Hence, \( \boxed{ R(v \times w) =(R v) \times(R w) } \).

\subsubsection*{(b) \( R \hat{w} R^{T} =\widehat{R w} \)}

We can show this by the above property.

The hat map is defined as \( \hat{(.)}: \mathbb{R}^{3} \rightarrow \mathbb{R}^{3 \times 3} \) such that
\[
    \hat{v} = \begin{bmatrix}
        0      & -v_{3} & v_{2}  \\
        v_{3}  & 0      & -v_{1} \\
        -v_{2} & v_{1}  & 0
    \end{bmatrix}
\]
for \( v = [v_{1}, v_{2}, v_{3}]^{T} \in \mathbb{R}^{3} \).

Now, for a vector \( v \in \mathbb{R}^{3} \), we have
\[
    R \hat{w} R^{T} v
    =
    R (\hat{w} (R^{T} v))
    =
    R (w \times (R^{T} v))
\]
\[
    \implies
    R (w \times (R^{T} v))
    =
    R w \times R(R^{T} v)
    =
    R w \times (RR^{T}) v
    =
    R w \times v
    =
    (R w) \times v
    =
    \widehat{R w} v
\]
where we used the property that \( R R^{T} = I \), and the above property (a).

Hence, \( \boxed{ R \hat{w} R^{T} = \widehat{R w} } \).
