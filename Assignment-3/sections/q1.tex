\section*{Problem 1}

Prove that given \(R \in S O(3)\) and \(v, w \in \mathbb{R}^{3}\), the following properties hold:
\begin{align*}
    R(v \times w)   & =(R v) \times(R w) \\
    R \hat{w} R^{T} & =\widehat{R w}
\end{align*}

\subsection*{Solution}

\subsubsection*{(a) \( R(v \times w) =(R v) \times(R w) \)}

We have
\[
    SO(3) = \{ R \in \mathbb{R}^{3 \times 3} \mid R R^{T} = I, \det(R) = +1 \}
\]

We will show that the rotation matrix \( R \) preserves length, the dot product and the cross product of two vectors.
Let \( v, w \in \mathbb{R}^{3} \) be two vectors.
\begin{align*}
    \implies
    \left \| R v \right \| ^{2}
        = {(R v)}^{T} (R v)
    = (v^{T} R^{T}) (R v)
    = v^{T} (R^{T} R) v
    = v^{T} v
    = \left \| v \right \| ^{2}
\end{align*}
Thereby, transformation by the rotation matrix \( R \) \underline{preserves the length} of a vector.

We can show that any length-preserving transformation preserves the dot product, by using the fact that the dot product can completely be expressed in terms of the lengths of the vectors as seen in the following identity, which is the law of cosines:
\begin{align*}
    v \cdot w
     & = \frac{1}{2} \left( \left \| v + w \right \| ^{2} - \left \| v \right \| ^{2} - \left \| w \right \| ^{2} \right)
    \\ \implies
    (R v) \cdot (R w)
     & = \frac{1}{2} \left( \left \| R v + R w \right \| ^{2} - \left \| R v \right \| ^{2} - \left \| R w \right \| ^{2} \right)
    \\ & = \frac{1}{2} \left( \left \| R (v + w) \right \| ^{2} - \left \| R v \right \| ^{2} - \left \| R w \right \| ^{2} \right)
    \\ & = \frac{1}{2} \left( \left \| v + w \right \| ^{2} - \left \| v \right \| ^{2} - \left \| w \right \| ^{2} \right)
    = v \cdot w
\end{align*}
Thereby, transformation by the rotation matrix \( R \) \underline{preserves the dot product} of two vectors, and thereby also \underline{preserves the angle} between the vectors.

We can see that the cross product of two vectors can be expressed in terms of length and the angle between the vectors, since
\[
    v \times w = \left \| v \right \| \left \| w \right \| \sin(\theta) \hat{n}
\]
where \( \hat{n} \) is the unit vector perpendicular to the plane containing \( v \) and \( w \), and \( \theta \) is the angle between the vectors.

\subsubsection*{(b) \( R \hat{w} R^{T} =\widehat{R w} \)}

We can show this by the above property.

The hat map is defined as \( \hat{(.)}: \mathbb{R}^{3} \rightarrow \mathbb{R}^{3 \times 3} \) such that
\[
    \hat{v} = \begin{bmatrix}
        0      & -v_{3} & v_{2}  \\
        v_{3}  & 0      & -v_{1} \\
        -v_{2} & v_{1}  & 0
    \end{bmatrix}
\]
for \( v = [v_{1}, v_{2}, v_{3}]^{T} \in \mathbb{R}^{3} \).

Now, for a vector \( v \in \mathbb{R}^{3} \), we have
\[
    R \hat{w} R^{T} v
    =
    R (\hat{w} (R^{T} v))
    =
    R (w \times (R^{T} v))
\]
\[
    \implies
    R (w \times (R^{T} v))
    =
    R w \times R(R^{T} v)
    =
    R w \times (RR^{T}) v
    =
    R w \times v
    =
    (R w) \times v
    =
    \widehat{R w} v
\]
where we used the property that \( R R^{T} = I \).

Hence, \( \boxed{ R \hat{w} R^{T} = \widehat{R w} } \).
