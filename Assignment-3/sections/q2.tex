\section*{Problem 2}

Prove that \(R \in S O(3)\) is a rigid body transformation, i.e \(R\) preserves distance and orientation.

\subsection*{Solution}

A \textit{rigid body transformation} is a transformation that preserves the distance between any two points and the orientation between any two vectors, i.e., it is a mapping \( g: \mathbb{R}^{3} \rightarrow \mathbb{R}^{3} \) such that the distance and cross product are preserved:
\begin{enumerate}
    \item \( \lVert g(p) - g(q) \rVert = \lVert p - q \rVert \) for all points \( p, q \in \mathbb{R}^{3} \)
    \item The cross product is preserved, i.e., \( g_{*}(v \times w) = g_{*}(v) \times g_{*}(w) \) for all vectors \( v, w \in \mathbb{R}^{3} \)
\end{enumerate}

\subsubsection*{Preservation of Distance}

Since \( R \) preserves the length, as shown in the previous question, we can show that the distance between any two points is preserved under the transformation \( R \).

Let \(v, w \in \mathbb{R}^{3}\) be two vectors. The distance between \(v\) and \(w\) is given by
\[
    \lVert v - w \rVert = \sqrt{{(v - w)}^{T}(v - w)}
\]

Now, let \(v' = R v\) and \(w' = R w\). The distance between \(v'\) and \(w'\) is given by
\begin{align*}
    \lVert v' - w' \rVert
     & = \sqrt{{(v' - w')}^{T}(v' - w')}
    \\ & = \sqrt{{((R v) - (R w))}^{T}((R v) - (R w))}
    \\ & = \sqrt{(v^{T} R^{T} - w^{T} R^{T})(R v - R w)}
    \\ & = \sqrt{((v^{T} - w^{T})R^{T}) (R(v - w))}
    \\ & = \sqrt{{(v - w)}^{T} (R^{T} R) (v - w)}
    \\ & = \sqrt{{(v - w)}^{T} (v - w)}
    \\ & = \lVert v - w \rVert
\end{align*}

Hence, the \underline{distance between \(v\) and \(w\) is preserved} under the transformation \(R\).

\subsubsection*{Preservation of Angle}

Since the dot product is preserved, as seen in the previous question, we can see that the angle between any two vectors is preserved under the transformation \( R \).

Let \(v, w \in \mathbb{R}^{3}\) be two vectors. The angle between \(v\) and \(w\) satisfies
\[
    \cos(\theta) = \frac{v^{T} w}{\lVert v \rVert \lVert w \rVert}
\]

Now, let \(v' = R v\) and \(w' = R w\). The angle between \(v'\) and \(w'\) satisfies
\begin{align*}
    \cos(\theta')
     & = \frac{{(R v)}^{T} (R w)}{\lVert R v \rVert \lVert R w \rVert}
    \\ & = \frac{v^{T} R^{T} R w}{\lVert R \rVert \lVert v \rVert \lVert R \rVert \lVert w \rVert}
    \\ & = \frac{v^{T} R^{T} R w}{\lVert R \rVert ^ 2 \lVert v \rVert \lVert w \rVert}
    \\ & = \frac{v^{T} w}{\lVert v \rVert \lVert w \rVert}
    \\ & = \cos(\theta)
\end{align*}
since \(R^{T} R = I \implies \lVert R^T \rVert \lVert R \rVert = \lVert I \rVert \implies \lVert R \rVert ^2 = 1\).

From here, we can see that \( \theta' = \theta \) since we define the angle between two vectors as being th principle value, i.e., the angle between the vectors is in the range \([0, 2\pi)\).
Hence, the \underline{angle between \(v\) and \(w\) is preserved} under the transformation \(R\).

\subsubsection*{Preservation of Orientation}

For the matrix \( R \in S O(3) \), we have shown in the previous question that it satisfies the property \( R(v \times w) = (R v) \times (R w) \).
Hence, the rotation matrix \underline{preserves the cross product} of two vectors, and thereby the orientation between any two vectors.

Therefore, since \( R \) preserves distance and the cross product as required, we have that \underline{ \(R \in S O(3)\) is a rigid body transformation }.
