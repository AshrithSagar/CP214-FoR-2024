\section*{Problem 4}

Let \(R \in S O(3)\) be a rotation matrix generated by rotating about a unit vector \(\omega \) by \(\theta \) radians.
That is, \(R\) satisfies \(R=\exp (\hat{\omega} \theta)\).
\begin{enumerate}[label= (\alph*)]
    \item Show that the eigenvalues of \(\hat{\omega}\) are \(0, i\) and \(-i\), where \(i=\sqrt{-1}\).
    \item Show that the eigenvalues of \(R\) are \(1, e^{i \theta}\), and \(e^{-i \theta}\).
\end{enumerate}

\subsection*{Solution}

\subsubsection*{(a) Eigenvalues of \(\hat{\omega}\)}

The skew-symmetric matrix \(\hat{\omega}\) can be written as
\[
    \hat{\omega} = \begin{bmatrix}
        0           & -\omega_{3} & \omega_{2}  \\
        \omega_{3}  & 0           & -\omega_{1} \\
        -\omega_{2} & \omega_{1}  & 0
    \end{bmatrix}
\]
The characteristic equation of \(\hat{\omega}\) is given by
\[
    \det(\hat{\omega} - \lambda I) = 0
    \implies
    \begin{vmatrix}
        -\lambda    & -\omega_{3} & \omega_{2}  \\
        \omega_{3}  & -\lambda    & -\omega_{1} \\
        -\omega_{2} & \omega_{1}  & -\lambda
    \end{vmatrix}
    = 0
\]
\[
    \implies
    -\lambda (\lambda^{2} + \omega_{1}^{2}) + \omega_{3} (-\omega_{3} \lambda - \omega_{1} \omega_{2}) + \omega_{2} (\omega_{3} \omega_{1} - \omega_{2} \lambda) = 0
\]
\[
    \implies
    \lambda^{3} + \lambda (\omega_{1}^{2} + \omega_{2}^{2} + \omega_{3}^{2}) = 0
\]
Since \( \omega \) is a unit vector, we have that \( \omega_{1}^{2} + \omega_{2}^{2} + \omega_{3}^{2} = 1^2 = 1 \).
\[
    \implies
    \lambda^{3} + \lambda = 0
    \implies
    \lambda (\lambda^{2} + 1) = 0
    \implies
    \lambda \in \{ 0, i, -i \}
\]
where \( i = \sqrt{-1} \).

Hence, the \underline{eigenvalues of \( \hat{\omega} \) are \( 0, i, -i \)}.

\subsubsection*{(b) Eigenvalues of \( R \)}

We will show that \( \exp(\lambda) \) is an eigenvalue of \( \exp(A) \), where \( A \) is a matrix and \( \lambda \) is an eigenvalue of \( A \).
The result then follows from the previous part and by \( R = \exp(\hat{\omega} \theta) \).

The matrix exponential of a matrix \( A \) is defined as, which is a power series,
\[
    \exp(A)
    = I + A + \frac{1}{2!} A^{2} + \frac{1}{3!} A^{3} + \ldots
    = \sum_{n=0}^{\infty} \frac{1}{n!} A^{n}
\]
Let \( \lambda \) be an eigenvalue of \( A \) with eigenvector \( v \).
Then, we have
\[
    A v = \lambda v
\]
Consider \( A^k v \), where
\begin{align*}
    \implies
    A^k v
    =
    (A^{k-1} A) v
    =
    A^{k-1}(A v) = A^{k-1} (\lambda v)
    =
    \lambda A^{k-1} v
\end{align*}
from which we can see that upon recursively applying, we have \( A^k v = \lambda^k v \) for all \( k \in \mathbb{N} \).
Hence, we can write
\begin{align*}
    \exp(A) v
     & = \Big(I + A + \frac{1}{2!} A^{2} + \frac{1}{3!} A^{3} + \ldots\Big) v
    \\ & =
    I v + A v + \frac{1}{2!} A^{2} v + \frac{1}{3!} A^{3} v + \ldots
    \\ & =
    v + \lambda v + \frac{1}{2!} \lambda^{2} v + \frac{1}{3!} \lambda^{3} v + \ldots
    \\ & =
    (1 + \lambda + \frac{1}{2!} \lambda^{2} + \frac{1}{3!} \lambda^{3} + \ldots) v
    \\ & =
    \exp(\lambda) v
\end{align*}
which implies that the eigenvalues of \( \exp(A) \) are \( \exp(\lambda) \) where \( \lambda \) are the eigenvalues of \( A \).

From the previous part, we have that the eigenvalues of \( \hat{\omega} \) are \( 0, i, -i \).
Since \( R = \exp(\hat{\omega} \theta) \), the eigenvalues of \( R \) are thereby, \( \exp(0), \exp(i \theta), \exp(-i \theta) \).

Hence, the \underline{eigenvalues of \( R \) are \( 1, e^{i \theta}, e^{-i \theta} \)}.
