\section*{Problem 5}

\textbf{Cayley parameters}:
Another parameterization of \(S O(3)\), which does not involve transcendental functions is Cayley's parameterization.
Let \(a\) be a vector in \(\mathbb{R}^{3}\) and let \(\hat{a}\) be the associated \(3 \times 3\) skew-symmetric matrix.
\begin{enumerate}[label= (\alph*)]
    \item Show that \(R_{a}=(I-\hat{a})^{-1}(I+\hat{a})\) satisfies the properties of \(S O(3)\)
    \item Given a rotation matrix \(R\), compute the Cayley parameters \(a\).
\end{enumerate}

\subsection*{Solution}

\subsubsection*{(a) \( R_{a}=(I-\hat{a})^{-1}(I+\hat{a}) \) satisfies the properties of \( SO(3) \)}

The special orthogonal group \( SO(3) \) is defined as
\[
    SO(3) = \{ R \in \mathbb{R}^{3 \times 3} \mid R R^{T} = I, \det(R) = +1 \}
\]

First, we will try to show that \( R_{a} \) is an orthogonal matrix, i.e., \( R_{a} R_{a}^{T} = R_{a}^{T} R_{a} = I \).\\
Since \( \hat a \) is a skew-symmetric matrix, we have that \( \hat a^{T} = -\hat a \).
\begin{align*}
    \implies
    R_a^T
     & =
    \left( (I - \hat{a})^{-1} (I + \hat{a}) \right)^T
    =
    (I + \hat{a})^T (I - \hat{a})^{-T}
    \\ & =
    (I^T + \hat{a}^T) (I^T - \hat{a}^T)^{-1}
    =
    (I - \hat{a}) (I + \hat{a})^{-1}
    \\
    \implies
    R_a R_a^T
     & =
    (I - \hat{a})^{-1} (I + \hat{a}) (I - \hat{a}) (I + \hat{a})^{-1}
\end{align*}

Now, we can observe that \( (I + \hat{a}) (I - \hat{a}) = I^2 - \hat{a}^2 = (I - \hat{a}) (I + \hat{a}) \).\\
Thereby commutativity holds for \( (I - \hat{a}) \) and \( (I + \hat{a}) \).
\begin{align*}
    \implies
    R_a R_a^T
     & =
    (I - \hat{a})^{-1} (I + \hat{a}) (I - \hat{a}) (I + \hat{a})^{-1}
    \\ & =
    (I - \hat{a})^{-1} (I - \hat{a}) (I + \hat{a}) (I + \hat{a})^{-1}
    \\ & =
    \Big((I - \hat{a})^{-1} (I - \hat{a}) \Big) \Big( (I + \hat{a}) (I + \hat{a})^{-1} \Big)
    \\ & =
    (I)(I)
    = I
\end{align*}
Thereby, \( R_a R_a^T = I \).
Now, we have to show that \( \det(R_a) = +1 \).
\begin{align*}
    \implies
    \det(R_a)
     & = \det \Big( (I - \hat{a})^{-1} (I + \hat{a}) \Big)
    \\ & = \det \left( (I - \hat{a})^{-1} \right) \det (I + \hat{a})
    \\ & = \frac{1}{\det(I - \hat{a})} \det(I + \hat{a})
\end{align*}

Now, we observe that for a skew-symmetric matrix \( \hat{a} \), it's non-zero eigenvalues are of the form \( \pm i \lambda \), where \( \lambda \) is a real number.
Hence, the eigenvalues of \( I - \hat{a} \) will be of the form \( 1 \pm i \lambda \) and the eigenvalues of \( I + \hat{a} \) will be of the form \( 1 \mp i \lambda \), thereby implying that they have the same eigenvalues.
Since the determinant of a matrix is the product of its eigenvalues, we have that \( \det(I - \hat{a}) = \det(I + \hat{a}) \), which gives us that \( \det(R_a) = 1 \).

Hence, \( \boxed{ R_a \in SO(3) } \).

\subsection*{Computing Cayley parameters given \( R \) }

Given \( R \in SO(3) \), we have that
\begin{align*}
    R
     & =
    (I - \hat{a})^{-1} (I + \hat{a})
    \\
    \implies
    (I - \hat{a}) R
     & =
    (I + \hat{a})
    \\
    \implies
    R - \hat{a} R
     & =
    I + \hat{a}
    \\
    \implies
    R - I
     & =
    \hat{a} R + \hat{a}
    \\
    \implies
    R - I
     & =
    \hat{a} (R + I)
    \\
    \implies
    \hat{a}
     & =
    (R - I)(R + I)^{-1}
\end{align*}
provided that \( R + I \) is invertible, which is possible when (-1) is not an eigenvalue of \( R \).

Hence, the Cayley parameters can be computed from \( R \) using \( \boxed{ \hat{a} = (R - I){(R + I)}^{-1} } \), provided (-1) is not an eigenvalue of \( R \).
Note that using Cayley parameters for rotation matrices having an eigenvalue of (-1) requires careful handling, such as for rotations by \( \pi \) about an axis, since \( (R + I) \) is singular in such cases.
