\section*{Problem 1}

\textbf{Drone Perching:}
A drone needs to attach a thin rod to the surface of a wall to perform an inspection.
Assume that the effect of gravity is completely canceled by the drone's thrust (essentially, treat gravity as \(g = 0\)) and that the drone's mass is negligible compared to the rod.
The rod has a mass \(m\), length \(l\) with inertia \(I_G = ml^2/12\) and is initially tumbling with a constant angular velocity \(\omega_0\) in a counterclockwise direction.
At the same time, the rod's center of mass is moving with a constant speed \(v_0\), as shown in the diagram below.
The end A of the rod then attaches to the rigid peg at O.
Just before the attaching, the rod's center of mass velocity is perpendicular to the rod.
\begin{enumerate}[label= (\alph*)]
    \item What is the new angular speed \(\omega_f\)?
    \item What is the loss in total kinetic energy after attaching?
\end{enumerate}
[Hint: Use conservation of angular momentum]

\begin{figure*}[h]
    \centering
    \includegraphics[width=0.7\linewidth]{figures/images/q1.png}
\end{figure*}

\subsection*{Solution}
