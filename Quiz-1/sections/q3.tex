\section*{Problem 3}

There are three forces acting on the circular plate shown below.
They act at the corners of a square which is concentric with the plate.
Find an equivalent force-couple system acting at point C.

\begin{figure*}[h]
    \centering
    \includegraphics[width=0.4\linewidth]{figures/images/q3.png}
\end{figure*}

\subsection*{Solution}

\begin{figure}[htb]
    \centering
    \begin{minipage}{0.375\textwidth}
        \centering
        \includegraphics[page=1, width=\linewidth]{figures/q3/_}
        \caption*{
            (a) Original plate with forces
        }
    \end{minipage}\hfill
    \begin{minipage}{0.3\textwidth}
        \centering
        \includegraphics[page=2, width=\linewidth]{figures/q3/_}
        \caption*{
            (b) Net force acting at O
        }
    \end{minipage}\hfill
    \begin{minipage}{0.3\textwidth}
        \centering
        \includegraphics[page=3, width=\linewidth]{figures/q3/_}
        \caption*{
            (c) Net torque acting at O
        }
    \end{minipage}
    \caption{
        Forces acting on the plate
    }\label{fig:q3}
\end{figure}

The forces acting on the plate are shown in Figure~\ref{fig:q3}, which can be broken down into a net force and a net torque acting at point O.
For an equivalent force-couple system, we should have
\begin{align*}
    \vec{F}_{\text{net}}
     & =
    \sum \vec{F}_i,
    \\
    \vec{M}_{\text{net}}
     & =
    \sum \vec{r}_i \times \vec{F}_i,
\end{align*}
where \(\vec{F}_{\text{net}}\) is the net force acting at point O, and \(\vec{M}_{\text{net}}\) is the net torque acting at point O.

\begin{align*}
    \implies
    \vec{F}_{\text{net}}
     & =
    \frac{3}{2} F \hat\imath - F \hat\imath - F \hat\jmath
    =
    \frac{1}{2} F \hat\imath - F \hat\jmath,
    \\
    \implies
    \lVert \vec{F}_{\text{net}} \rVert
     & =
    \frac{5}{\sqrt{2}}
\end{align*}
