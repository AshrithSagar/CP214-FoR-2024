\section*{Problem 4}

A Mars rover moves with a velocity \(v\) on a flat ground as shown below.
The crawler belt of the rover has mass \(m\).
Determine the kinetic energy of the crawler belt, assuming its length is much larger than that of the radius of rover wheels. [Hint: Rover wheels are under pure rolling condition]

\begin{figure*}[h]
    \centering
    \includegraphics[width=0.4\linewidth]{figures/images/q4.png}
\end{figure*}

\subsection*{Solution}

\begin{figure}[htb]
    \centering
    \includegraphics[page=1, width=0.7\linewidth]{figures/q4/_}
    \caption{
        Side view of the rover
    }\label{fig:q4-belt}
\end{figure}

We can see that under pure rolling condition, the bottom of the crawler belt is stationary with respect to the ground, and the entire top of the crawler belt moves with a velocity \(2v\), as shown by the movement of the wheel in Figure~\ref{fig:q4-wheel}.

\begin{figure}[htb]
    \centering
    \includegraphics[page=2, width=0.3\linewidth]{figures/q4/_}
    \caption{
        Side view of the wheel
    }\label{fig:q4-wheel}
\end{figure}

The kinetic energy of the crawler belt is given by
\begin{align*}
    \text{KE}_{\text{belt}}
     & = \frac{1}{2} \int_{\text{belt}} v^2 \, \text{d}m
    \\ & =
    \int_{\substack{\text{top}
    \\ \text{belt}}}
    + \int_{\substack{\text{bottom}
    \\ \text{belt}}}
    + \int_{\substack{\text{left}
    \\ \text{belt}}}
    + \int_{\substack{\text{right}
    \\ \text{belt}}}
    \\ & =
    \frac{1}{2} \int_{\substack{\text{top}
    \\ \text{belt}}} {(2v)}^2 \, \text{d}m
    + \cancel{\frac{1}{2} \int_{\substack{\text{bottom}
    \\ \text{belt}}} {(0)}^2 \, \text{d}m}
    + \frac{1}{2} \int_{\substack{\text{left}
    \\ \text{belt}}} v^2 \, \text{d}m
    + \frac{1}{2} \int_{\substack{\text{right}
    \\ \text{belt}}} v^2 \, \text{d}m
\end{align*}

We can observe that the last two terms can be ignored, as the radius of the rover wheels is much smaller than the length of the crawler belt.
This can be seen as follows, by first considering the density of the belt, as given by \( \lambda = m/l \), following which we can see that
\[
    \text{d}m = \lambda \text{d}r
\]
\begin{equation*}
    \frac{1}{2} \int_{
        \substack{\text{left} \\ \text{belt}}}
    v^2 \, \text{d}m
    \approx
    \frac{1}{2} v_0^2 \int_{
        \substack{\text{left} \\ \text{belt}}}
    \text{d}m
    =
    \frac{1}{2} v_0^2 \int_{
        \substack{\text{left} \\ \text{belt}}}
    \lambda \text{d}r
\end{equation*}
