\section*{Problem 2}
\setcounter{section}{2}
\setcounter{equation}{0}

Consider a cast iron dumbbell consisting of a cylinder connecting two solid spheres at either end of the cylinder.
The density of the dumbbell is \( 7500 \mathrm{~kg} / \mathrm{m}^{3} \).
The cylinder has a diameter of 4 cm and a length of 20 cm.
Each sphere has a diameter of 20 cm.
\begin{enumerate}[label= (\alph*)]
    \item Find the approximate Rotational Inertia Matrix \( \mathcal{I}^{b} \) in a frame \( \{b\} \) at the center of mass with axes aligned with the principal axes of inertia of the dumbbell.

    \item Write down the full Mass Matrix \( \mathcal{M}^{b}=\operatorname{BlockDiag}\left(m I, \mathcal{I}^{b}\right) \).

    \item For this part no need to compute your answer numerically.
          Show that in another frame \( \{\mathrm{a}\} \), we have \( \mathcal{M}^{a}=\operatorname{Ad}_{g_{a b}}^{T} \mathcal{M}^{b} \operatorname{Ad}_{g_{a b}} \).
          Is this a generalized version of the parallel axis theorem you learned at the beginning of this course? Yes/No?

          [Hint: Which scalar quantity remains invariant under frame transformation?]
\end{enumerate}

\subsection*{Solution}

\subsubsection*{(a) Rotational Inertia Matrix \( \mathcal{I}^{b} \) in frame \( \{b\} \)}

The Rotational Inertia Matrix \( \mathcal{I}^{b} \) in frame \( \{b\} \) is given by
\begin{equation}
    \mathcal{I}^{b}=\begin{bmatrix}
        I_{xx} & 0      & 0      \\
        0      & I_{yy} & 0      \\
        0      & 0      & I_{zz}
    \end{bmatrix}
\end{equation}
where \( I_{xx} \), \( I_{yy} \), and \( I_{zz} \) are the principal moments of inertia of the dumbbell.

The principal moments of inertia of the dumbbell are given by
\begin{align}
    I_{xx} & =\frac{1}{12} m\left(3 r_{1}^{2}+l^{2}\right) \\
    I_{yy} & =\frac{1}{12} m\left(3 r_{1}^{2}+l^{2}\right) \\
    I_{zz} & =\frac{1}{2} m r_{1}^{2}
\end{align}
where
\begin{align*}
    m     & =\rho V=\rho\left(\pi r_{1}^{2} l+4 / 3 \pi r_{1}^{3}\right) \\
    r_{1} & =\frac{d_{1}}{2}=\frac{4}{2} \times 10^{-2} \mathrm{m}       \\
    l     & =20 \times 10^{-2} \mathrm{m}
\end{align*}
Substituting the values of \( m \), \( r_{1} \), and \( l \) in equations (2), (3), and (4), we get
\begin{align*}
    I_{xx} & =\frac{1}{12} \times 7500\left(3\left(\frac{4}{2} \times 10^{-2}\right)^{2}+\left(20 \times 10^{-2}\right)^{2}\right) \\
           & =0.0016 \mathrm{~kg} \cdot \mathrm{m}^{2}                                                                             \\
    I_{yy} & =\frac{1}{12} \times 7500\left(3\left(\frac{4}{2} \times 10^{-2}\right)^{2}+\left(20 \times 10^{-2}\right)^{2}\right) \\
           & =0.0016 \mathrm{~kg} \cdot \mathrm{m}^{2}                                                                             \\
    I_{zz} & =\frac{1}{2} \times 7500\left(\frac{4}{2} \times 10^{-2}\right)^{2}                                                   \\
           & =0.0012 \mathrm{~kg} \cdot \mathrm{m}^{2}
\end{align*}

Therefore, the Rotational Inertia Matrix \( \mathcal{I}^{b} \) in frame \( \{b\} \) is
\begin{equation}
    \mathcal{I}^{b}=\begin{bmatrix}
        0.0016 & 0      & 0      \\
        0      & 0.0016 & 0      \\
        0      & 0      & 0.0012
    \end{bmatrix}
\end{equation}

\subsubsection*{(b) Mass Matrix \( \mathcal{M}^{b} \)}

The full Mass Matrix \( \mathcal{M}^{b} \) is given by
\begin{equation}
    \mathcal{M}^{b}=\operatorname{BlockDiag}\left(m I, \mathcal{I}^{b}\right)
\end{equation}
where
\begin{align*}
    m & =7500\left(\pi\left(\frac{4}{2} \times 10^{-2}\right)^{2} \times 20 \times 10^{-2}+\frac{4}{3} \pi\left(\frac{4}{2} \times 10^{-2}\right)^{3}\right) \\
      & =7500\left(\pi \times 0.0016 \times 0.2+\frac{4}{3} \pi \times 0.0003\right)                                                                         \\
      & =7500\left(0.00032+0.0004\right)                                                                                                                     \\
      & =7500 \times 0.00072                                                                                                                                 \\
      & =5.4 \mathrm{~kg}
\end{align*}
\begin{equation}
    \implies
    m I=\begin{bmatrix}
        5.4 & 0   & 0   \\
        0   & 5.4 & 0   \\
        0   & 0   & 5.4
    \end{bmatrix}
\end{equation}

Therefore, the full Mass Matrix \( \mathcal{M}^{b} \) is
\begin{equation}
    \mathcal{M}^{b}=\begin{bmatrix}
        5.4 & 0   & 0   & 0      & 0      & 0      \\
        0   & 5.4 & 0   & 0      & 0      & 0      \\
        0   & 0   & 5.4 & 0      & 0      & 0      \\
        0   & 0   & 0   & 0.0016 & 0      & 0      \\
        0   & 0   & 0   & 0      & 0.0016 & 0      \\
        0   & 0   & 0   & 0      & 0      & 0.0012
    \end{bmatrix}
\end{equation}

\subsubsection*{(c) Mass Matrix \( \mathcal{M}^{a}=\operatorname{Ad}_{g_{a b}}^{T} \mathcal{M}^{b} \operatorname{Ad}_{g_{a b}} \)}

The Mass Matrix \( \mathcal{M}^{a} \) in frame \( \{\mathrm{a}\} \) is given by
\begin{equation}
    \mathcal{M}^{a}=\operatorname{Ad}_{g_{a b}}^{T} \mathcal{M}^{b} \operatorname{Ad}_{g_{a b}}
\end{equation}
where \( g_{a b} \) is the transformation matrix from frame \( \{b\} \) to frame \( \{\mathrm{a}\} \).

This is indeed a generalized form of the parallel axis theorem, which relates the moments of inertia in two frames that are offset by a vector.
The scalar quantity that remains invariant under frame transformation is the total mass \( m_{\text{total}} \).
