\section*{Problem 2}
\setcounter{section}{2}
\setcounter{equation}{0}

Consider a cast iron dumbbell consisting of a cylinder connecting two solid spheres at either end of the cylinder.
The density of the dumbbell is \( 7500 \mathrm{~kg} / \mathrm{m}^{3} \).
The cylinder has a diameter of 4 cm and a length of 20 cm.
Each sphere has a diameter of 20 cm.
\begin{enumerate}[label= (\alph*)]
    \item Find the approximate Rotational Inertia Matrix \( \mathcal{I}^{b} \) in a frame \( \{b\} \) at the center of mass with axes aligned with the principal axes of inertia of the dumbbell.

    \item Write down the full Mass Matrix \( \mathcal{M}^{b}=\operatorname{BlockDiag}\left(m I, \mathcal{I}^{b}\right) \).

    \item For this part no need to compute your answer numerically.
          Show that in another frame \( \{\mathrm{a}\} \), we have \( \mathcal{M}^{a}=\operatorname{Ad}_{g_{a b}}^{T} \mathcal{M}^{b} \operatorname{Ad}_{g_{a b}} \).
          Is this a generalized version of the parallel axis theorem you learned at the beginning of this course? Yes/No?

          [Hint: Which scalar quantity remains invariant under frame transformation?]
\end{enumerate}

\subsection*{Solution}

\subsubsection*{Given}

\begin{align*}
    \rho
     & =
    7500\mathrm{~kg} / \mathrm{m}^{3}
    \\
    r_{c}
     & =
    \frac{d_{c}}{2}
    =
    4 / 2 \times 10^{-2} \mathrm{~m}
    =
    0.02 \mathrm{~m}
    \\
    l_c
     & =
    20 \times 10^{-2} \mathrm{~m}
    =
    0.2 \mathrm{~m}
    \\
    r_{s}
     & =
    \frac{d_{s}}{2}
    =
    20 / 2 \times 10^{-2} \mathrm{~m}
    =
    0.1 \mathrm{~m}
\end{align*}

\subsubsection*{(a) Rotational Inertia Matrix \( \mathcal{I}^{b} \) in frame \( \{b\} \)}

Since the frame \( \{b\} \) axis is given along the principal axis of inertia of the dumbbell, we have that the Rotational Inertia Matrix \( \mathcal{I}^{b} \) in frame \( \{b\} \) is diagonal, and is given by
\begin{equation*}
    \mathcal{I}^{b}=\begin{bmatrix}
        I_{xx} & 0      & 0      \\
        0      & I_{yy} & 0      \\
        0      & 0      & I_{zz}
    \end{bmatrix}
\end{equation*}
where \( I_{xx} \), \( I_{yy} \), and \( I_{zz} \) are the principal moments of inertia of the dumbbell.
\begin{align*}
    \implies
    I_{xx}
     & =
    \frac{1}{12} m\left(3 r_{c}^{2}+l_c^{2}\right)
    \\
    \implies
    I_{yy}
     & =
    \frac{1}{12} m\left(3 r_{c}^{2}+l_c^{2}\right)
    \\
    \implies
    I_{zz}
     & =
    \frac{1}{2} m r_{c}^{2}
\end{align*}
where
\begin{align*}
    m
     & =
    \rho V
    =
    \rho
    \left(\pi r_{1}^{2} l+4 / 3 \pi r_{1}^{3}\right)
\end{align*}
\begin{align*}
    \implies
    I_{xx}
     & =
    \frac{1}{12} \times 7500\left(3\left(\frac{4}{2} \times 10^{-2}\right)^{2}+\left(20 \times 10^{-2}\right)^{2}\right)
    \\ & =
    0.0016 \mathrm{~kg} \cdot \mathrm{m}^{2}
    \\
    I_{yy}
     & =
    \frac{1}{12} \times 7500\left(3\left(\frac{4}{2} \times 10^{-2}\right)^{2}+\left(20 \times 10^{-2}\right)^{2}\right)
    \\ & =
    0.0016 \mathrm{~kg} \cdot \mathrm{m}^{2}
    \\
    I_{zz}
     & =
    \frac{1}{2} \times 7500\left(\frac{4}{2} \times 10^{-2}\right)^{2}
    \\ & =
    0.0012 \mathrm{~kg} \cdot \mathrm{m}^{2}
\end{align*}

Therefore, the Rotational Inertia Matrix \( \mathcal{I}^{b} \) in frame \( \{b\} \) is
\begin{equation*}
    \boxed{
        \mathcal{I}^{b}
        =
        \begin{bmatrix}
            0.0016 & 0      & 0      \\
            0      & 0.0016 & 0      \\
            0      & 0      & 0.0012
        \end{bmatrix}
    }
\end{equation*}

\subsubsection*{(b) Mass Matrix \( \mathcal{M}^{b} \)}

The full Mass Matrix \( \mathcal{M}^{b} \) is given by
\begin{equation*}
    \mathcal{M}^{b}=\operatorname{BlockDiag}\left(m I, \mathcal{I}^{b}\right)
\end{equation*}
where
\begin{align*}
    m & =
    7500\left(\pi\left(0.02\right)^{2} \times 0.2+\frac{4}{3} \pi\left(0.02\right)^{3}\right)
    \\ & =
    7500\left(\pi\left(0.02\right)^{2} \times 0.2+\frac{4}{3} \pi\left(0.02\right)^{3}\right)
    \\ & =
    5.4\mathrm{~kg}
\end{align*}
\begin{equation*}
    \implies
    m I=\begin{bmatrix}
        5.4 & 0   & 0   \\
        0   & 5.4 & 0   \\
        0   & 0   & 5.4
    \end{bmatrix}
\end{equation*}

Therefore, the full Mass Matrix \( \mathcal{M}^{b} \) is
\begin{equation*}
    \boxed{
        \mathcal{M}^{b}
        =
        \begin{bmatrix}
            5.4 & 0   & 0   & 0      & 0      & 0      \\
            0   & 5.4 & 0   & 0      & 0      & 0      \\
            0   & 0   & 5.4 & 0      & 0      & 0      \\
            0   & 0   & 0   & 0.0016 & 0      & 0      \\
            0   & 0   & 0   & 0      & 0.0016 & 0      \\
            0   & 0   & 0   & 0      & 0      & 0.0012
        \end{bmatrix}
    }
\end{equation*}

\subsubsection*{(c) Mass Matrix \( \mathcal{M}^{b}=\operatorname{Ad}_{g_{a b}}^{\top} \mathcal{M}^{a} \operatorname{Ad}_{g_{a b}} \)}

The Mass Matrix \( \mathcal{M}^{a} \) in frame \( \{a\} \) is given by
\begin{equation*}
    \mathcal{M}^{b}
    =
    \operatorname{Ad}_{g_{a b}}^{\top}
    \mathcal{M}^{a}
    \operatorname{Ad}_{g_{a b}}
\end{equation*}
where \( g_{a b} \) is the transformation matrix from frame \( \{b\} \) to frame \( \{\mathrm{a}\} \).
The scalar quantity that remains invariant under frame transformation is the total mass \( m_{\text{total}} \).
We can see this by doing block operations on the \( 3 \times 3 \) blocks in the expressions.

For \( g_{ab} = \begin{bmatrix}
    R & p \\
    0 & 1
\end{bmatrix}
\) and \( \mathcal{M}^{a} = \operatorname{BlockDiag}\left(m I, \mathcal{I}^{a}\right) \), we have
\begin{align*}
    \implies
    \operatorname{Ad}_{g_{a b}}
     & =
    \begin{bmatrix}
        R & \widehat{p} R \\
        0 & R
    \end{bmatrix}
    \implies
    \operatorname{Ad}_{g_{a b}}^{\top}
    =
    \begin{bmatrix}
        R^{\top}              & 0        \\
        - R^{\top}\widehat{p} & R^{\top}
    \end{bmatrix}
    \\
    \implies
    \mathcal{M}^{b}
     & =
    \operatorname{Ad}_{g_{a b}}^{\top} \mathcal{M}^{a} \operatorname{Ad}_{g_{a b}}
    \\ & =
    \begin{bmatrix}
        R^{\top}              & 0        \\
        - R^{\top}\widehat{p} & R^{\top}
    \end{bmatrix}
    \begin{bmatrix}
        mI & 0               \\
        0  & \mathcal{I}^{a}
    \end{bmatrix}
    \begin{bmatrix}
        R & \widehat{p} R \\
        0 & R
    \end{bmatrix}
    =
    \begin{bmatrix}
        R^{\top}               & 0        \\
        - R^{\top} \widehat{p} & R^{\top}
    \end{bmatrix}
    \begin{bmatrix}
        m R & m \widehat{p} R   \\
        0   & \mathcal{I}^{a} R
    \end{bmatrix}
    \\ & =
    \begin{bmatrix}
        m R^{\top} R              & mR^{\top} \widehat{p} R                                              \\
        -m R^{\top} \widehat{p} R & (-m R^{\top} \widehat{p} \widehat{p} R + R^{\top} \mathcal{I}^{a} R)
    \end{bmatrix}
    \\
    \because
    R^{\top} \mathcal{I}^{a} R
     & =
    \mathcal{I}^{b}
    \\
    \implies
    \mathcal{M}^{b}
     & =
    \begin{bmatrix}
        m R^{\top} R                       & \cancel{mR^{\top} \widehat{p} R}                                   \\
        \cancel{-m R^{\top} \widehat{p} R} & (\cancel{-m R^{\top} \widehat{p} \widehat{p} R} + \mathcal{I}^{b})
    \end{bmatrix}
    =
    \begin{bmatrix}
        mI & 0               \\
        0  & \mathcal{I}^{b}
    \end{bmatrix}
    =
    \operatorname{BlockDiag}\left(m I, \mathcal{I}^{b}\right)
\end{align*}
as required.
This is indeed a generalized form of the parallel axis theorem, which relates the moments of inertia in two frames that are offset by a vector.

\subsubsection*{Alternate}

The kinetic energy of the system in frame \( \{b\} \) is given by
\begin{equation*}
    T
    =
    \frac{1}{2} \dot{\xi_{b}}^{\top} \mathcal{M}^{b} \dot{\xi_{b}}
\end{equation*}
where \( \mathcal{M}^{b} \) is the Mass Matrix in frame \( \{b\} \), and we have that the kinetic energy remains invariant under frame transformations, thereby
\begin{align*}
    T
     & =
    \frac{1}{2} \dot{\xi_{b}}^{\top} \mathcal{M}^{b} \dot{\xi_{b}}
    =
    \frac{1}{2} \dot{\xi_{a}}^{\top} \mathcal{M}^{a} \dot{\xi_{a}}
\end{align*}
and since \( \xi_{a} = \operatorname{Ad}_{g_{a b}} \xi_{b} \), we have that
\begin{align*}
    \dot{\xi_{a}}
     & =
    \frac{d}{dt} \operatorname{Ad}_{g_{a b}} \xi_{b}
    =
    \operatorname{Ad}_{g_{a b}} \dot{\xi_{b}}
    \\
    \implies
    \frac{1}{2} \dot{\xi_{a}}^{\top} \mathcal{M}^{a} \dot{\xi_{a}}
     & =
    \frac{1}{2} \dot{\xi_{b}}^{\top} \operatorname{Ad}_{g_{a b}}^{\top} \mathcal{M}^{a} \operatorname{Ad}_{g_{a b}} \dot{\xi_{b}}
    =
    \frac{1}{2} \dot{\xi_{b}}^{\top} \mathcal{M}^{b} \dot{\xi_{b}}
\end{align*}
and since \( \xi_{b} \) is arbitrary, we have that
\begin{equation*}
    \boxed{
    \mathcal{M}^{b}
    =
    \operatorname{Ad}_{g_{a b}}^{\top} \mathcal{M}^{a} \operatorname{Ad}_{g_{a b}}
    }
\end{equation*}
