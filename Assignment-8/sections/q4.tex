\section*{Problem 4}
\setcounter{section}{4}
\setcounter{equation}{0}

\textbf{Lie Bracket:} \\
The generalization a of the cross product between two 6 dimensional twists \( \xi_{1}=\left(v_{1}, \omega_{1}\right) \) and \( \xi_{2}=\left(v_{2}, \omega_{2}\right) \) is called the Lie bracket of \( \xi_{1} \) and \( \xi_{2} \).
It is commonly written as \( \left[\widehat{\xi}_{1}, \widehat{\xi}_{2}\right] \), and is defined as follows,
\[
    \left[\widehat{\xi}_{1}, \widehat{\xi}_{2}\right]=\widehat{\xi}_{1} \widehat{\xi}_{2}-\widehat{\xi}_{2} \widehat{\xi}_{1} \in s e(3)
\]
The use of this quantity comes in trying to compactly write down the equations of motion using Newton's Laws (force \( = \) change of momentum) and Euler's Equations (torque \( = \) change of angular momentum) or in trying to efficiently compute Coriolis Matrices.

Now prove the following Lie bracket identity (called the Jacobi identity) for arbitrary twists \( \xi_{1}, \xi_{2}, \xi_{3} \):
\[
    \left[\widehat{\xi}_{1},\left[\widehat{\xi}_{2}, \widehat{\xi}_{3}\right]\right]+\left[\widehat{\xi}_{3},\left[\widehat{\xi}_{1}, \widehat{\xi}_{2}\right]\right]+\left[\widehat{\xi}_{2},\left[\widehat{\xi}_{3}, \widehat{\xi}_{1}\right]\right]=0
\]

\subsection*{Solution}
