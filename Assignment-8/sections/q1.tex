\section*{Problem 1}
\setcounter{section}{1}
\setcounter{equation}{0}

Show that the forward kinematics map for a manipulator is independent of the order in which rotations and translations are performed.

\subsection*{Solution}

The forward kinematics of a manipulator is described by a mapping \( g_{st} : Q \to SE (3) \)which describes the end-effector configuration as a function of the robot joint variables, where \( Q \) is the joint space/configuration space of the manipulator consisting of all possible values of the joint variables of the robot.
For open-chain manipulators with \( p \) revolute joints and \( r \) prismatic joints, the configuration has \( p+r \) degrees of freedom and we have that \( Q \subseteq \mathbb{T}^{p} \times \mathbb{R}^{r} \), where \( \mathbb{T} \) is the \(p\)-torus, i.e., \( \mathbb{T}^{p} = \mathbb{S}^{1} \times \cdots \times \mathbb{S}^{1} \) (p times), with \( \mathbb{S}^{1} \) being the unit circle in the plane and \( \times \) denoting the Cartesian product/direct product of groups, and \( \mathbb{R}^{r} \) is the \(r\)-dimensional Euclidean space.
The general form of the forward kinematics map can be written as
\begin{equation}
    g_{st}(\theta)
    =
    g_{s l_1}(\theta_1)
    g_{l_1 l_2}(\theta_2)
    \cdots
    g_{l_{n-1} l_n}(\theta_n)
    g_{l_n t}
\end{equation}
where the frames are denoted successively by \( S \) (base frame), \( L_i \) (link \(i\) frame), and \( T \) (tool frame), and the relative transformation matrices between the frames are denoted by \( g_{ij} \) where the subscripts \( i \) and \( j \) denote the frames between which the transformation is defined.
In terms of twists, the forward kinematics map is given by the product of exponentials formula,
\begin{equation}
    g_{st}(\theta)
    =
    e^{\widehat{\xi}_1 \theta_1}
    e^{\widehat{\xi}_2 \theta_2}
    \cdots
    e^{\widehat{\xi}_n \theta_n}
    g_{st}(0)
\end{equation}
where \( \xi_{i} \) is the twist corresponding to the \(i\)-th joint axis in the reference \( (\theta = 0) \) configuration.
To show that the product of exponentials formula is invariant to the order in which the rotations and translations are performed, we can prove it by mathematical induction.
Below, we show the proof for the case of a manipulator with two degrees of freedom, and the proof can be extended to manipulators with more degrees of freedom.

\newpage
\begin{figure}[htb]
    \centering
    \includegraphics[width=0.6\textwidth]{figures/images/q1.png}
    \caption{
        Two degree of freedom manipulator, with two revolute joints.
    }\label{fig:2dof-manipulator}
\end{figure}

To illustrate this, consider a simple example of a manipulator with two degrees of freedom, an example of which is shown in Figure~\ref{fig:2dof-manipulator}, with two revolute joints with joint variables \( \theta_1 \) and \( \theta_2 \).
Keeping \( \theta_1 \) fixed and varying \( \theta_2 \) gives the configuration of the tool frame as a function of \( \theta_2 \) only, as
\begin{equation*}
    g_{st}(\theta_2)
    =
    e^{\widehat{\xi}_2 \theta_2}
    g_{st}(0)
\end{equation*}
Now, fixing \( \theta_2 \) and varying \( \theta_1 \) gives the overall configuration of the tool frame as
\begin{equation}\label{eq:varying-theta2-first}
    g_{st}(\theta_1, \theta_2)
    =
    e^{\widehat{\xi}_1 \theta_1}
    g_{st}(\theta_2)
    =
    e^{\widehat{\xi}_1 \theta_1}
    e^{\widehat{\xi}_2 \theta_2}
    g_{st}(0)
\end{equation}
Now consider the case where the order is reversed, i.e., first varying \( \theta_1 \) and then varying \( \theta_2 \).
Keeping \( \theta_2 \) fixed and varying \( \theta_1 \), we have
\begin{equation*}
    g_{st}(\theta_1)
    =
    e^{\widehat{\xi}_1 \theta_1}
    g_{st}(0)
\end{equation*}
This motion moves the axis of \( \theta_2 \), and the new axis is given by
\(
\displaystyle
\xi_2'
=
\operatorname{Ad}_{e^{\widehat{\xi}_1 \theta_1}} \xi_2
\)
, and the overall motion is then given by
\begin{equation*}
    g_{st}(\theta_1, \theta_2)
    =
    e^{\widehat{\xi}_2' \theta_2}
    g_{st}(\theta_1)
    =
    e^{\widehat{\xi}_2' \theta_2}
    e^{\widehat{\xi}_1 \theta_1}
    g_{st}(0)
\end{equation*}
Now, by properties of the adjoint transformation, we know that \( e^{\widehat{\operatorname{Ad}_X Y}} = e^{X Y X^{-1}} = X e^{Y} X^{-1} \), where \( X \in SE (3) \) and \( Y \in se (3) \), and thus we have
\begin{align*}
    e^{\widehat{\xi}_2' \theta_2}
     & =
    e^{\widehat{\xi}_1 \theta_1}
    e^{\widehat{\xi}_2 \theta_2}
    e^{-\widehat{\xi}_1 \theta_1}
    \\
    \implies
    g_{st}(\theta_1, \theta_2)
     & =
    e^{\widehat{\xi}_1 \theta_1}
    e^{\widehat{\xi}_2 \theta_2}
    \cancel{e^{-\widehat{\xi}_1 \theta_1}}
    \cancel{e^{\widehat{\xi}_1 \theta_1}}
    g_{st}(0)
    \nonumber
    \\ & =
    e^{\widehat{\xi}_1 \theta_1}
    e^{\widehat{\xi}_2 \theta_2}
    g_{st}(0)
\end{align*}
which is the same as the result obtained in~\eqref{eq:varying-theta2-first}.
Thus, we have the result.
