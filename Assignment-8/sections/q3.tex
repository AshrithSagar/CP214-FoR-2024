\section*{Problem 3}
\setcounter{section}{3}
\setcounter{equation}{0}

\textbf{Passivity of Robot Dynamics:} \\
Let \( \mathcal{H}(\theta)=\mathcal{T}(\theta)+\mathcal{V}(\theta) \) be the total energy for a rigid robot.
Show that if \( \dot{\mathcal{M}}-2 \mathcal{C} \) is skew-symmetric, then energy is conserved, i.e., \( \dot{\mathcal{H}}=\dot{\theta}^{\top} \tau \).
(Basically total energy only changes due to the input work applied to the system.).

\subsection*{Solution}

With
\begin{equation*}
    \mathcal{H}(\theta)
    =
    \mathcal{T}(\theta)
    +
    \mathcal{V}(\theta)
\end{equation*}
as the total energy for a rigid robot, we have the kinetic and potential energies as
\begin{equation*}
    \mathcal{T}(\theta)
    =
    \frac{1}{2} \dot{\theta}^{\top} \mathcal{M}(\theta) \dot{\theta}
    \quad \text{and} \quad
    \mathcal{V}
    =
    \mathcal{C}(\theta) \dot{\theta}
\end{equation*}
where \( \mathcal{M}(\theta) \) is the inertia matrix and \( \mathcal{C}(\theta) \) is the Coriolis matrix.
\begin{align*}
    \implies
    \dot{\mathcal{T}}
     & =
    \frac{d}{dt} \left( \frac{1}{2} \dot{\theta}^{\top} \mathcal{M}(\theta) \dot{\theta} \right)
    =
    \frac{1}{2} \left( \ddot{\theta}^{\top} \mathcal{M}(\theta) \dot{\theta} + \dot{\theta}^{\top} \mathcal{M}(\theta) \ddot{\theta} \right)
    =
    \dot{\theta}^{\top} \mathcal{M}(\theta) \ddot{\theta}
    \\
    \implies
    \dot{\mathcal{V}}
     & =
    \frac{d}{dt} \left( \mathcal{C}(\theta) \dot{\theta} \right)
    =
    \dot{\mathcal{C}}(\theta) \dot{\theta} + \mathcal{C}(\theta) \ddot{\theta}
\end{align*}
Given that \( \dot{\mathcal{M}}-2 \mathcal{C} \) is skew-symmetric, we have
\begin{equation*}
    {\left( \dot{\mathcal{M}}-2 \mathcal{C} \right)}^{\top}
    =
    -\left( \dot{\mathcal{M}}-2 \mathcal{C} \right)
\end{equation*}
\begin{align*}
    \implies
    \dot{\mathcal{H}}
     & =
    \dot{\mathcal{T}} + \dot{\mathcal{V}}
    =
    \dot{\theta}^{\top} \mathcal{M}(\theta) \ddot{\theta} + \dot{\mathcal{C}}(\theta) \dot{\theta} + \mathcal{C}(\theta) \ddot{\theta}
    \\ & =
    \dot{\theta}^{\top} \left( \mathcal{M}(\theta) \ddot{\theta} + \mathcal{C}(\theta) \dot{\theta} \right) + \dot{\mathcal{C}}(\theta) \dot{\theta}
    \\ & =
    \dot{\theta}^{\top} \left( \tau - \mathcal{V}(\theta) \right) + \dot{\mathcal{C}}(\theta) \dot{\theta}
    =
    \dot{\theta}^{\top} \tau - \dot{\theta}^{\top} \mathcal{V}(\theta) + \dot{\mathcal{C}}(\theta) \dot{\theta}
    \\ & =
    \dot{\theta}^{\top} \tau - \dot{\theta}^{\top} \mathcal{C}(\theta) \dot{\theta} + \dot{\mathcal{C}}(\theta) \dot{\theta}
    =
    \dot{\theta}^{\top} \tau - \dot{\theta}^{\top} \mathcal{C}(\theta) \dot{\theta} + \dot{\mathcal{C}}(\theta) \dot{\theta}
    =
    \dot{\theta}^{\top} \tau
\end{align*}
Thus, the total energy only changes due to the input work applied to the system.
